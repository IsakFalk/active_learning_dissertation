\chapter{Literature Review}
\label{ch:lit_rev}

In this chapter, we review and compile the work done on active learning for
regression based on RKHS theory. We divide this into three parts: the first
focusing on the theoretical foundation of RKHS theory, the second active
learning theory and previous work using RKHS to do active learning for
regression, and finally we give an overview of Kernel Herding (hereafter
\textit{KH}) and its link to the Frank Wolfe (hereafter \textit{FW}) algorithm
for optimising the maximum mean discrepancy efficiently.

\section{Reproducing Kernel Hilbert Space Theory}
\label{sec:rkhs-theory}

\subsection{A brief history of the what and the why of Reproducing Kernel
Hilbert Spaces}
\label{sec:a-brief-hisory-RKHS} Kernels have a long history and the field was
arguably started with the work of \cite{aronszajn50_theor_reprod_kernel}. Since
then it has found use, to name a few, in geostatistics in the form of Kriging
\cite{cressie90_origin_krigin}, machine learning with support vector machines
\cite{evgeniou99_suppor} and gaussian processes \cite{williams96_gauss}.

A first question might be why we are interested in RKHS's when in many other
areas of mathematics and the sciences we consider \(L^{2}\) spaces. One of the
first obstacles is that the space of \(L^{2}(\X, \rho_{\X})\) is actually not a
space of functions, but equivalence classes of functions which agree except on a
set of measure zero with respect to \(\rho_{\X}\). In particular for specific
choices of kernel functions, for any compact subset \(\mathcal{Z}\) of the input
space \(\X\), the set of \(\overline{\textrm{span}(K_x, x \in \mathcal{Z})}\) is
dense in \(\cont{}(\mathcal{Z})\), the space of all continuous functions on
\(\mathcal{Z}\) with respect to the maximum norm, showing the power of RKHS's.

A second reason is that by choosing an RKHS as the hypothesis space turns the
regularised empirical risk minimisation objective, when using a convex loss,
into a nice convex program through the use of the representer theorem
\cite["Large" Reproducing Kernel Hilbert
Spaces]{gretton18_advan_topic_machin_learn}. This also allows us to turn
optimisation over a space of functions to a dual problem over a
finite-dimiensional vector space, reducing it to the domain of linear algebra
and giving us access to the tools therein.

However, there are drawbacks of using kernels for learning. It is well-known
that most kernel algorithms require inversion of the kernel matrix, including
kernel ridge regression, which scales cubically in the number of datapoints
\cite{saunders98_ridge_regres_learn_algor_dual_variab} making it prohibitively
expensive except for small datasets. Luckily this goes very well with active
learning since we are actually interested in making the size of the dataset
small while retaining close-to-optimal performance. This means that we are not
really giving up scalability as we are already making the aim to label as few
instances in the original dataset as possible. Hence RKHS is an ideal space to
work with as it gives us nice theoretical properties, a well-developed theory
and when used in active learning, few of the drawbacks usually encountered.

\subsection{Reproducing Kernel Hilbert Spaces} The field of RKHS theory is
well-established and there has been much written about it, we will follow the
course on RKHS theory taught at UCL 2018-2019
\cite{gretton18_advan_topic_machin_learn} together with introductions of
\cite{manton15_primer_reprod_kernel_hilber_spaces,fasshauer11_posit_defin_kernel}.

A Hilbert space is a tuple \((\Hc, \langle \cdot, \cdot \rangle)\) where \(\Hc\)
is a vector space and \(\langle \cdot, \cdot \rangle\) is an inner product
defined on some set \(\X\) such that this space is complete with respect to the
metric \(\|\cdot\|\) induced by the inner product. An RKHS is a specific type of
Hilbert space with the following property:

\begin{definition}
\label{def:RKHS} Let \(\X\) be a non-empty set and let \(\rkhs{H}\) be a Hilbert
space of functions \(f: \X \to \R\), then \(\rkhs{H}\) is said to be an RKHS if
for any \(x \in \X\) the evaluation functional \(\delta_x: f \mapsto f(x)\) is
continuous or equivalently bounded.
\end{definition}

The above definition is often too abstract in practice and we will show how to
equivalently construct RKHs's through the use of specific concepts of
\emph{kernels, reproducing kernel} and \emph{positive definite functions},
starting with the definition a kernel,
\begin{definition}
\label{def:kernel} Let \(\X\) be a non-empty set and \(K\) a function \(K: \X
\times \X \to \R\). Then \(K\) is called a \textit{kernel} on \(\X\) if there
exists some Hilbert space \(\rkhs{H}\) and a \textit{feature map} \(\phi : \X
\to \rkhs{H}\) such that for any \(x, x' \in \X\), \(K(x, x') =
\scal{\phi(x)}{\phi(x')}_{\rkhs{H}}\).
\end{definition}

Secondly the concept of a \emph{reproducing kernel},
\begin{definition}
\label{def:reproducing-kernel} Let \(\X\) be a non-empty set and let
\(\rkhs{H}\) be a Hilbert space of functions \(f: \X \to \R\). A function \(K:
\X \times \X \to \R\) is called a \textit{reproducing kernel} of \(\rkhs{H}\) if
it satisfies the following properties
\begin{align} \forall x \in \X, & \quad K_x = K(\cdot, x) \in
\rkhs{H}, \label{al:reproducer-in-rkhs}\\ \forall x \in \X, & \: \forall f \in
\rkhs{H}, \quad \scal{f}{K_x}_{\rkhs{H}} =
f(x), \label{al:reproducing-property-of-rkhs}
\end{align} that is, all functions \(K_x\) are in the space and they have the so
called \textit{reproducing property}.
\end{definition} In particular, we have that
\begin{corollary}
  \label{cor:reproducing-property-kxy} For any \(x, x' \in \X\)
  \begin{equation*} K(x, x') = \scal{K(\cdot, x)}{K(\cdot, x')}_{\rkhs{H}}.
  \end{equation*}
\end{corollary} In fact, every reproducing kernel is a kernel
\begin{corollary}
 \label{cor:reproducing-kernel-is-a-kernel} Every reproducing kernel is a kernel
with explicit feature map \(\phi : x \mapsto K(\cdot, x) \in \rkhs{H}\).
\end{corollary}

Finally, we have the following theorems which state that an RKHS is
characterised by its reproducing kernel function and it is unique:
\begin{theorem}
\label{th:rep-kernel-is-unique} If it exists, a reproducing kernel is unique.
\end{theorem}

\begin{theorem}
\label{th:rep-kernel-defines-rkhs} \(\rkhs{H}\) is an RKHS if and only if it has
a reproducing kernel.
\end{theorem}

A reproducing kernel is useful since it enables us to use machinery from linear
algebra to bound quantities. For example, we can use cauchy-schwartz (from here
on denoted \emph{CS}) to bound \(f(x)\) using \(\kappa = \sup_{x \in \X}
\sqrt{K(x, x)}\),
\begin{example}
\label{ex:cs-rkhs-bound-on-fx} When \(f \in \rkhs{H}\), an RKHS with reproducing
kernel \(K\), then we have the following
  \begin{align*} \abs{f(x)} & = \abs{\scal{f}{K(\cdot, x)}_{\rkhs{H}}} \\ & \leq
\norm{f}_{\rkhs{H}} \norm{K(\cdot, x)}_{\rkhs{H}} \\ & \leq \norm{f}_{\rkhs{H}}
\sqrt{K(x, x)} \\ & \leq \norm{f}_{\rkhs{H}} \sup_{x \in \X} \sqrt{K(x, x)} \\ &
= \norm{f}_{\rkhs{H}} \cdot \kappa
  \end{align*}
\end{example} where we have used the CS inequality since \(\rkhs{H}\) is a
Hilbert space. Thus we have
\begin{equation}
\label{eq:cs-rkhs-bound-on-fx} f(x) \leq \norm{f}_{\rkhs{H}} \cdot \kappa,
\end{equation} which shows that if a function is in an RKHS it is enough to
control the norm \(\norm{f}_{\rkhs{H}}\) in order to control \(f\) at any point
\(x \in \X\).

The final part we need is that of \emph{positive semi-definiteness},
\begin{definition}
  \label{def:pos-semi-definite-function} Let \(\X\) be a non-empty set then \(K
: \X \times \X \to \R\) is a \textit{positive semi-definite} function if for any
\(n \geq 1\), for any \((\alpha_i)_{i=1}^n \in \R^n\), for any \((x_i)_{i=1}^n
\in \X^n\),
\begin{equation*} \sum_{i, j}^n \alpha_i \alpha_j K(x_i, x_j) \geq 0,
\end{equation*} and we say that \(K\) is \textit{positive definite} if for
mutually distinct \(x_i\), the inequality is strict for any vector
\((\alpha_i)_{i=1}^n \neq \mat{0}\).
\end{definition}

Every inner product is a positive semi-definite function, which implies that
\begin{corollary}
\label{cor:kernels-are-positive-semi-definite} Every kernel \(K\) is a positive
semi-definite function.
\end{corollary}

At this point we have shown that for any reproducing kernel \(K\) of an RKHS
\(\rkhs{H}\), it is also a kernel and furthermore is positive semi-definite,
finally we also have that every positive semi-definite function \(K\) is the
reproducing kernel for some RKHS \(\rkhs{H}\),
\begin{theorem}[\cite{aronszajn50_theor_reprod_kernel}]
\label{thm:moore-aronszajn-pos-def-function-is-rep-kernel} Let \(\X\) be a
non-empty set and let \(K: \X \times \X \to \R\) be a positive semi-definite
function. Then there exists a unique RKHS \(\rkhs{H} \subset \{f : \X \to \R, f
\: \textrm{is measureable}\}\) such that \(K\) is the reproducing kernel of this
space.
\end{theorem}

Looking back, we see that using an RKHS as the space of functions under
consideration in machine learning is useful as it enables us to use the
\emph{kernel trick}: if an algorithm is defined only in terms of elementary
operations over inner products between data vectors, we can replace each inner
product \(\scal{x}{x'}\) with a kernel \(K(x, x')\) in order to work implicitly
in a (potentially) infinite dimensional space.

There are many different kernels that can be used and can be combined in various
ways according to rules we won't go into here, yielding a sort of calculus of
kernels. We will only consider the so called Gaussian kernel,
\begin{definition}
  \label{def:gaussian-kernel} Let \(\X \subseteq \R^d\) for some \(d > 0\), the
gaussian kernel \(K_{\sigma} : \X \times \X \to \R\) has the form of
  \begin{equation}
    \label{eq:gaussian-kernel} K_{\sigma}(x, x') = \exp(-\frac{\norm{x -
x'}_2^2}{2\sigma^2}),
  \end{equation} where \(\sigma > 0\).
\end{definition}

The Gaussian kernel is ubiqituos in machine learning and other fields which use
RKHS's, in particular it has the following properties
\begin{theorem}
  \label{th:gaussian-kernel-properties} For any \(\sigma > 0\), the Gaussian
kernel is
  \begin{itemize}
  \item Characteristic
  \item Universal
  \item Stationary
  \end{itemize} .
\end{theorem}

\subsection{Kernel Mean Embedding} Kernel Mean Embedding (hereafter \emph{KME})
generalises the reproducer for \(x \in \X\), \(K_{x}\), to that of a
distribution \(\rho \in \prob(\X)\). We will follow the review of
\cite{muandet17_kernel_mean_embed_distr} and the notes of
\cite{gretton18_advan_topic_machin_learn}.

\begin{definition}
  \label{def:kernel-mean-embedding} Let \(\X\) be a non-empty set and let
\(\prob(X)\) be the space of distributions on a measurable space \((\X,
\Sigma)\). The kernel mean embedding of a distribution \(\rho \in \prob(X)\)
into a RKHS \(\rkhs{H}\) endowed with a reproducing kernel \(K: \X \times \X \to
\R\) is defined by a mapping
  \begin{equation}
    \label{eq:kernel-mean-embedding} \mu : \prob(\X) \to \rkhs{H}, \quad
\mu(\rho) = \int_{\X} K_{x} \dd \rho(x).
  \end{equation}
\end{definition} We will use the shorthand \(\mu_{\rho} \coloneqq \mu(\rho)\).
Note that this generalises the definition of \(K_{x}\) since a dirac delta
distribution \(\delta_{x'}\) gives us \(\mu(\delta_{x'}) = \int_{\X}K_{x} \dd
\delta_{x'} = K_{x'}\). In this way \(\mu\) can be seen as the analogue to
\(\phi\) when working with distributions rather than points.

The following lemma gives necessary condition for the element to be in
\(\rkhs{H}\). We reproduce the proof as it exposes techniques we will be using
later.

\begin{theorem} Let \(\rkhs{H}\) be an RKHS with reproducing kernel \(K: \X
\times \X \to \R\). If \(\E_{\rho}[\sqrt{K(X, X)}] < \infty\) and \(f \in
\rkhs{H}\), then \(\mu_{\rho} \in \rkhs{H}\) and \(\E_{\rho}[f(X)] =
\scal{f}{\mu_{\rho}}_{\rkhs{H}}\).
\end{theorem}

\begin{proof} Let \(L_{\rho}\) be the linear operator defined as \(L_{\rho}(f) =
\E_{\rho}[f(X)]\). Then
  \begin{align*} \abs{ L_{\rho}(f) } & = \abs{\E_{\rho}[f(X)]} \\ & \leq
\E_{\rho}[\abs{f(X)}] \\ & = \E_{\rho}[\abs{\scal{f}{K_{X}}}] \\ & \leq
\E_{\rho}[\sqrt{K(X, X)} \norm{f}_{\rkhs{H}}]
  \end{align*} where we first apply Jensen's inequality then the CS inequlity.
By Riesz representation theorem, there exists an element \(\lambda_{\rho} \in
\rkhs{H}\) such that \(L_{\rho}(f) = \scal{\lambda_{\rho}}{f}_{\rkhs{H}}\).
Letting \(f = K_{x}\) for some \(x \in \X\), then \(\lambda_{\rho}(x) =
L_{\rho}(K_{x}) = \int_{\X} K(x, x') \dd \rho(x')\) which shows that
\(\lambda_{\rho} = \mu_{\rho}\).
\end{proof}

Trivially we have the following corollary
\begin{corollary} If \(\kappa = \sup_{x \in \X}\sqrt{K(X, X)} < +\infty\) then
for any \(\rho \in \prob(\X)\), \(\mu_{\rho} \in \rkhs{H}\).
\end{corollary}

\begin{proof} Since \(\E_{\rho}[\sqrt{K(X, X)} \norm{f}_{\rkhs{H}}] \leq \kappa
\norm{f}_{\rkhs{H}}\) we are done.
\end{proof}

We finish by considering the KME of the empirical distribution of a dataset. Let
\(S = (x_{i})_{i=1}^{n}\) be a sequence of \(n\) datapoints from some set
\(\X\), then the empirical distribution of \(S\), call this \(\rho_{S}\), is
defined to be
\begin{equation}
\label{eq:empirical-distribution} \rho_{S} = \frac{1}{n}\sum_{i=1}^{n}
\delta_{x_{i}}
\end{equation} and the corresponding mean embedding is \(\mu_{\rho_{S}} =
\int_{S} K_{x} \dd \rho_{S} = \frac{1}{n} \sum_{i=1}^{n}K_{x_{i}}\). If \(S\) is
the set of \(n\) samples sampled \(\iid\) from some distribution \(\rho\), then
we say
\begin{equation}
\label{eq:empirical-kernel-mean-embedding} \hat{\mu}_{\rho} \coloneqq
\mu_{\rho_{S}}
\end{equation} as it is an \emph{empirical estimate} of the mean embedding
\(\mu_{\rho}\).

\subsection{Maximum Mean Discrepancy} Integral Probability Metrics (hereafter
\emph{IPM}) \cite{mueller97_integ_probab_metric_their_gener_class_funct} are
distance-like functions defined in the following way
\begin{definition}
\label{def:IPM} Let \(\Fc\) be a space of real-valued bounded measurable
functions on a non-empty set \(\X\). Given two probability measures \(\rho,
\xi\) with support of \(\X\), then
  \begin{equation}
    \label{eq:IPM} \mathrm{IPM}_{\Fc}(\xi, \rho) = \sup_{f \in \Fc}\left(
\int_{\X}f(x) \dd \xi(x) - \int_{\X}f(x) \dd \rho(x) \right).
  \end{equation}
\end{definition}

Any space \(\Fc\) defines an IPM, we will only consider the case when \(\Fc\) is
a unit ball in an RKHS, in which case we call this metric the Maximum Mean
Discrepancy (hereafter \emph{MMD}). It's simple to show that in this case the
MMD takes the following simple form,
\begin{theorem}
\label{th:mmd-is-norm-between-feature-maps} Let \(B\) be the unit ball of an
RKHS with reproducing kernel \(K\), then we can express the MMD as
\begin{equation}
\label{eq:mmd-is-norm-between-feature-maps} \MMD{\xi}{\rho}{B} = \norm{\mu_{\xi}
- \mu_{\rho}}_{\rkhs{H}}.
\end{equation}
\end{theorem} This leads to the following expression of the MMD as the
expectation over kernels,
\begin{equation}
\label{eq:mmd-expectation-over-kernels} \MMD{\xi}{\rho}{B}^{2} = \E_{X, X' \sim
\xi}[K(X, X')] - 2 \E_{X \sim \xi, Y \sim \rho}[K(X, Y)] + \E_{Y, Y' \sim
\rho}[K(Y, Y')],
\end{equation} where the notation \(\E_{X, X' \sim \xi}[K(X, X')]\) indicates
that we draw two independent identically distributed copies form \(\xi\). When
the distributions are discrete, this turns into sums over kernel evaluations,
\begin{corollary}
\label{cor:discrete-dists-MMD-sum} When \(\xi = \sum_{i=1}^{n}\alpha_{i}
\delta_{x_{i}}\) and \(\rho = \sum_{j=1}^{m} \beta_{j} \delta_{y_{j}}\), where
we collect \((\alpha_{i})_{i=1}^{n}, (\beta_{j})_{j=1}^{m}\) into vectors
\(\mat{\alpha}, \mat{\beta}\), then we have
  \begin{align} \MMD{\xi}{\rho}{B}^{2} & = \sum_{i, i'}^{n} \alpha_i
\alpha_{i'}K(x_i, x_{i'}) - 2 \sum_{i, j}^{n, m} \alpha_i \beta_j K(x_i, y_j) +
\sum_{j, j'}^{n} \beta_j \beta_{j'}K(y_j, y_{j'}) \\ & = \mat{\alpha}^T
\mat{K}_{nn} \mat{\alpha} - 2 \mat{\alpha}^T \mat{K}_{nm} \mat{\beta} +
\mat{\beta}^T \mat{K}_{mm} \mat{\beta},
  \end{align} where \(\mat{K}_{nn} \in \R^{n \times n}, \mat{K}_{nm} \in \R^{n
\times m}, \mat{K}_{mm} \in \R^{m \times m}\) are matrices such that
\((\mat{K}_{nn})_{lt} = K(x_l, x_t), (\mat{K}_{nm})_{lt} = K(x_l, y_t),
(\mat{K}_{mm})_{lt} = K(y_l, y_t)\).
\end{corollary}

When the kernel is characteristic, the MMD is a true metric
\begin{theorem}
\label{th:when-characteristic-kernel-mmd-is-metric} When \(K\) is a
characteristic kernel, MMD is a metric on distributions in \(\prob(\X)\).
\end{theorem}

\section{Active Learning} We introduce the necessary concepts of Active Learning
by first introducing Supervised Learning (hereafter \emph{SL}) under the
Statistical Learning Theory (hereafter \emph{SLT}) model following the book
\cite{shalev-shwartz14_under} and the notes from the Learning Theory part of the
course in Advanced Topics in Machine Learning given at UCL 2019
\cite{ciliberto18_advan_topic_machin_learn}. After this we review active
learning as a field, looking at the historical developments and sub-divisions.
We will then focus on the so called \emph{pool-based active learning}, showing
how it relates to supervised learning and critically analyse the works using
RKHS theory. We lean on the comprehensive introduction of
\cite{settles12_activ_learn}.

\subsection{Statistical Learning Theory and Supervised Learning} SLT aims to
introduce a framework grounded on probability theory that answers such questions
as \emph{what does it mean for an algorithm to learn} and \emph{How do we design
good learning algorithms} \cite[Lecture
1]{ciliberto18_advan_topic_machin_learn}.

The SLT framework introduces the following concepts,

\begin{description}
\item[{Domain set}] An arbitrary set, \(\X\). The set of objects that we wish to
find the output for. We also refer to this set as the \emph{input space} and
call a point \(x \in \X\) an \emph{instance} or \emph{input}. We assume that
\(\X \subseteq \R^{D}\) for some \(D \in \N\).
\item[{Co-domain set}] An arbitrary set, \(\Y\). Our goal is to find a good
mapping from \(\X\) to \(\Y\) in the sense that we should be able to predict the
output in \(\Y\) for an arbitrary instance \(x \in \X\), hence why it's called
the \emph{output space} or \emph{target space}, occasionally we will call \(y\)
a \emph{label}. We will focus on regression, assuming that \(\Y \subseteq \R\).
\item[{Train set}] \(S = (x_{i}, y_{i})_{i=1}^{n}\) is a finite sequence of
pair, where each pair \((x_{i}, y_{i}) \in \X \times \Y\).
\item[{Hypothesis space}] A space of functions \(\mathcal{H} \subseteq \{h : \X
\to \Y, \: h \: \mathrm{measurable}\}\) meant to represent all of the possible
input-output rules that we consider. Note that we have used \(\mathcal{H}\) for
both an RKHS and hypothesis space, in practice this doesn't matter as we will
always use an RKHS as the hypothesis space.
\item[{Learning algorithm}] A function that takes as input a train set and maps
  to a hypothesis space,
  \begin{equation}
    \label{eq:sl-algorithm}
    \algo : \cup_{i=1}^{\infty} (\X \times \Y)^{n} \to \mathcal{H}.
  \end{equation}
\item[{Data generating distribution}] We assume there exists some distribution
\(\rho\) with support on \(\X \times \Y\) such that \(\rho(x, y) = \rho(y |
x)\rho_{\X}(x)\). The train set \(S = (x_{i}, y_{i})_{i=1}^{n}\) is sampled
\(\iid\) from \(\rho\), that is, each pair \((x_{i}, y_{i}) \in S\) is sampled
independently from \(\rho\) and we write \(S \sim \rho^{n}\). We do not have
access to \(\rho\) directly, but only through \(S\).
\item[{Measure of performance}] We choose a \emph{loss function}, \(\ell : \Y
\times \Y \to \R_{+}\) that quantifies the loss of predicting \(\hat{y}\) when
the output is \(y\) through \(\ell(y, y')\). The goal of SLT is then to minimize
the \emph{risk}, which is a function of an estimator \(h\),
\begin{equation}
\label{eq:def-risk} \risk{}{h} = \E_{\rho}[\ell(h(x), y)],
\end{equation} and we will be writing \(\err{\rho}{h}\) to indicate the risk
with respect to what measure.
\end{description}

The goal of SLT is to find an algorithm \(\algo\) that with high probability
over the sampled train set \(S\) is close in performance to the optimal
estimator in the hypothesis space. This is formalised by the concept of
\emph{excess risk},
\begin{definition}
  \label{def:excess-risk} Given the above, we define the excess risk to be the
quantity
  \begin{equation}
    \label{eq:excess-risk} \err{\rho}{h} - \err{\rho}{h^{\ast}},
  \end{equation} where \(h^{\ast} \in \argmin_{h' \in \rkhs{H}}\err{\rho}{h'}\).
\end{definition} The excess risk measures the additional risk that we take on by
using \(h\) rather than the optimal estimator in the hypothesis space. In
practice, the estimator \(h\) will be random since it will be the output of an
algorithm \(\algo\), which maps from a train set \(S\) to \(h\)\footnote{When
the function \(h\) is the output of an algorithm based on a train set \(S\) of
size \(n\) we denote \(\hat{h}_{n} \coloneqq \algo(S)\). If the size of \(S\) is
obvious we suppress the \(n\) and simply write \(\hat{h}\).}, making the excess
risk itself random due to the random sampling of \(S \sim \rho^{n}\).

In order to control the excess risk in probability, statements are made of the
following form
\begin{example}
  \label{ex:pac-like-excess-risk} With probability larger than \(1 - \delta\)
taken over the sampling of the train set \(S \sim \rho^n\), for any \(\delta \in
[0, 1]\), we have that the excess risk is bounded,
  \begin{equation*} \err{\rho}{\hat{h}_n} - \err{\rho}{h^{\ast}} \leq \epsilon
  \end{equation*}
\end{example} and these bounds are derived using error decomposition and
controlling for the error in probabiliy using assumptions on the hypothesis
class, data generating distribution etc. One famous example of this is Empirical
Risk Minimization (hereafter \emph{ERM})\cite{vapnik92_princ} where the
algorithm \(\algo\) is defined as the minimizer of the empirical risk, which in
our setting reduces to the following definition,
\begin{definition}
  \label{def:erm} Given an input and an output space \(\X, \Y\), an RKHS
\(\rkhs{H}\) acting as our hypothesis space, with loss \(\ell\) and a dataset
\(S = (x_i, y_i)_{i=1}^n\), Empirical Risk Minimization is an algorithm
  \begin{equation} \algo(S) = \argmin_{h \in \mathcal{H}} \frac{1}{n}
\sum_{i=1}^n \ell(h(x_i), y_i).
  \end{equation}
\end{definition} In this work, we will focus on Regularised Empirical Risk
Minimization (herafter \emph{RERM}) specialised to our setting,
\begin{definition}
  \label{def:rerm} Given input and output space \(\X, \Y\), an RKHS \(\rkhs{H}\)
acting as our hypothesis space, with loss \(\ell\) and a dataset \(S = (x_i,
y_i)_{i=1}^n\), Regularised Empirical Risk Minimization is an algorithm
  \begin{equation} \algo(S) = \argmin_{h \in \mathcal{H}} \frac{1}{n}
\sum_{i=1}^n \ell(h(x_i), y_i) + \lambda \norm{h}_{\rkhs{H}}^2.
  \end{equation}
\end{definition}

Kernel Ridge Regression (hereafter \emph{KRR}) is a generalisation of Ridge
Regression \cite{hoerl70_ridge_regres} where all of the inner products are
lifted to the feature space through the use of the kernel function. KRR can be
seen as the solution to the RERM when we have an RKHS \(\rkhs{H}\) and mean
squared error \(\ell(y, y') = (y - y')^{2}\) with \(\X \subseteq \R^{D}\) and
\(\Y \subseteq \R\), so that
\begin{equation}
  \label{eq:krr-rerm-form} \algo(S) = \argmin_{h \in \mathcal{H}}
\frac{1}{n}\sum_{i=1}^n (h(x_i) - y_i)^2 + \lambda \norm{h}_{\rkhs{H}}^2.
\end{equation}

It can be shown that KRR has the following solution
\begin{equation}
\label{eq:krr-equation} \algo(S)(x) = \mat{k}_{x}^{T} \mat{\alpha}_{\ast} =
\mat{k}_{x}^{T} (\mat{K}_{nn} + \lambda n \mat{I}_{n})^{-1}\mat{Y},
\end{equation} where
\begin{equation}
\label{eq:k-one-x} \mat{k}_{x} =
\begin{bmatrix} K(x, x_{1}) \\ \vdots \\ K(x, x_{n}), 
\end{bmatrix}
\end{equation} and
\begin{equation}
  \label{eq:y-mat} \mat{Y} =
  \begin{bmatrix} y_1 \\ \vdots \\ y_n, 
  \end{bmatrix}
\end{equation}
.

We judge the quality of an algorithm with respect to its \emph{error bound}. An
error bound gives us probabilistic guarantees on the magnitude of the excess
risk as defined in \ref{def:excess-risk},
\begin{definition}
  \label{def:error-bound-sl} A supervised learning error bound \(\epsilon(n,
\delta)\) is a function depending on an algorithm \(\algo\) that given a
\(\delta \in [0, 1]\) assures that the excess risk will be less than
\(\epsilon\) with probability greater than \(1 - \delta\). Or as a mathematical
expression,
  \begin{equation} \Pr_{S \sim \rho^n}(\err{}{\algo(S)} - \err{}{h^{\ast}} \leq
\epsilon(n, \delta)) \geq 1 - \delta.
  \end{equation}
\end{definition}

\subsection{Active Learning Theory Review} Active learning generalises the
setting of supervised learning by introducing an aspect of cost to acquire
labels of instances. According to Settles
\begin{quote} Active learning is any form of learning in which the learning
program has some control over the inputs on which it trains
\cite{settles12_activ_learn}.
\end{quote} In the SL setting we are \emph{given} a train set \(S\) consisting
of instance and output-pairs sampled from \(\rho\). Compare this to a setting
where we are only given a large set of \emph{instances} but attaining the
outputs of an instances is costly, it seems possible that we could attain an
acceptable level of accuracy or risk by just asking for a few outputs rather
than all of them. The latter example can be formalised as the setting of Active
Learning and was hinted at in the \hyperref[sec:background]{Background} section.

From literature there are proposed criteria for what a strategy for choosing
instances to label should consider and allows us to classify algorithms
according to how they implement each of these.
\cite{wu18_pool_based_sequen_activ_learn_regres} argue that a strategy should
take into account the \emph{informativeness}, \emph{representativeness} and
\emph{diversity} of the instances chosen. This corresponds well with how the
objective for choosing what instance to label often decompose into terms that
can be interpreted as encouraging informative instances but discouraging instances similar to instances already labeled (for
example \cite[Equation 4]{guo08_discr} and \cite[Equation
7]{chattopadhyay13_batch_mode_activ_sampl_based}).

It is possible to define an active learning algorithm by specifying an
\emph{active learning tuple} \((\algo, \querstrat)\) where \(\algo\) is the
algorithm of SL as defined in \ref{eq:sl-algorithm} and \(\querstrat\) is the strategy for how to choose instances
to query. The various setting of active learning then depends on what we need
\(\querstrat\) to do and according to what objective we want to have good
performance on. This leads to the subsettings of active learning.

Active learning can be split up into three subsettings, being \emph{Query
Synthesis}, \emph{Stream-Based Selective Sampling} and \emph{Pool-based Active
Learning} \cite{settles12_activ_learn}. These can be further divided into
sequential and batch-mode active learning where the first considers instances
one-by-one while the second instead presents the learner with sets of instances
\cite{guo08_discr}. The dominating stream of the two is that of sequential
learning due to the simplicity of deriving results but also due to the
combinatorial aspect of choosing subsets from a mother set leading to explosion
in computational complexity unless using approximations. Furthermore, these
approximations can sometimes be shown to be close-to-optimal.
\cite{krause08_near_optim_sensor_placem_gauss_proces} shows that in an active
learning setting where the goal is to optimally place sensors in predefined
locations and casting this as a problem of maximising the Mutual Information
between the sensor locations and the rest of the locations, there is a
polynomial-time greedy procedure that is within \((1 - 1/e)\) of optimum of the
NP-hard combinatorial objective problem. All this indicates that the
batch-setting is underexplored as the benefits are small over sequential active
learning.

All active learning theory from here on will be considered in the sequential
setting, and note that all batch-mode active learning can be rendered sequential
by making the set of size 1. Active learning then operates in a loop where \(t\)
is the time step at the start of the iteration, and the active learning
algorithm gets an input to the querying strategy \(\querstrat\) after which
\(\querstrat\) chooses what to do depending on the current setting, thereafter
the algorithm is free to predict or do whatever needed for \(\querstrat\) to
choose the next point at time \(t+1\). The reason for the flexibility is that in
general the instance chosen at \(t+1\) may depend on the output and predicted
output of the trained model up to time \(t\), that is the full history up until \(t\).

\subsection{Query synthesis} When the algorithm is allowed to synthesis
instances \emph{de novo} so that the instance chosen by \(\querstrat\) at time
\(t\) may be any point in \(\X\) we call this query synthesis, or learning with
membership queries \cite{angluin88_queries_concep_learn}. One prominent issue
with this is that we do not have access to \(\rho_{\X}\) at all and thus cannot
hope to minimise any risk unless this is supposed to be known a-priori.

\cite{cohn96_activ_learn_with_statis_model} does query synthesis by assuming
that \(\rho_{\X}\) is uniform to make the problem tractable, learning the
dynamics of an idealised robot arm. This can be a fruitful direction to take if
the actual goal is about learning a concept in itself rather than aiming for low
risk. However, there are failure cases when the instances are actually from some
underlying distribution \(\rho_{\X}\), which is expanded upon in
\cite{baum92_query} where they show that using humans to label synthesised
images of MNIST lead to strange artifacts as the algorithm does not take into
account the fact that the true marginal distribution on image only has support
for a small subspace of all possible images of digits. This makes it problematic
since we are trying to reduce the risk and need access to samples from
\(\rho_{\X}\) in order to learn about the marginal distribution.

\subsection{Stream-Based active learning} This is also called selective sampling
\cite{cohn94_improv_gener_with_activ_learn} due to the querying strategy
deciding at each time step whether to query an instance, which arrive in an
on-line fashion, or discard it and query a later instance. The assumption is
that the source of the instances produce these virtually for free so cost only
enters through querying. The example of learning a threshold concept in
\hyperref[sec:background]{Background} can be cast in this setting, where we reject any instance
shown to us to be less than the biggest instance currently labeled \(0\). In
this way, the querying strategy is a function \(\querstrat: \X \to \{0, 1\}\)
where \(0\) represents a decision to not label and \(1\) to label the current
instance streamed.

Approaches of when to query often corresponds to the criteria of an instance
being informative, representative and to add to the diversity of the already
queried instances so far. The most straightforward approach is to define a
measure of utility of information content and bias the sampling towards
instances with high information content. A second approach is that of using a
version space \cite{tong01_suppor_vector_machin_activ_learn}.

This setting contains many theoretically grounded algorithms, for example
\cite{dasgupta08}. They also point out some common problems with general active
learning algorithms; they rely upon computations which are not feasible in
practice or requires solving problems which are intractable in practice.

\subsection{Pool-based active learning} When given a set of datapoints \(S \sim
\rho^{n}\) we may assume that we know the instances and that the labels exist,
but are hidden from us until we query them for the corresponding outputs. When
the querying comes with a cost we want to only query as many instances as needed
to reach some level of performance before stopping. This is the pool-based
active learning setting \cite{lewis94} and it differs from query synthesis in
that the query strategy \(\querstrat\) depends on the dataset \(S\) of which it
only has access to the instances and labels from previously queried instances to
pick which instance to query next.

Pool-based active learning has a lot of overlap with the information retrieval
and data mining communities, due to the straightforward representation of
databases with instances such as documents as a pool set where labels need to be
produced by human experts
\cite{lewis94,tong01_suppor_vector_machin_activ_learn}. \cite{settles08} go into
how various utility based ways to choose what instance to label. Most of these
depend on the output of the current best hypothesis (or in the case of Query By
Committee, a set of hypotheses), which we will see is unnecessary in the setting
we will consider.

One limitation of pool-based active learning is that if the pool of samples is
big, the querying strategy theoretically need to calculate the utility of each
instance and then pick the instance which maximises the utility. In practice,
this is less of a concern as it's possible to approximate this scheme by for
example first subsample a smaller unlabeled dataset and label the most
informative instance in this. This poses a problem for us though, as unless we
can show that making this approximation leads to guarantees in probability, it's
not simple to see how this can be overcome.

We now define the necessary components for Pool-based active learning for
regression. For a set of input-output pairs \(\poolset = (x_{i},
y_{i})_{i=1}^{n}\) we denote the set of only instance as \(\poolsetx =
(x_{i})_{i=1}^{n}\). Pool-based active learning operates in a loop and we let
\(t\) denote the current time-step, incremented at the end of each loop. We have
the following

\begin{description}
\item[{Pool set}] A set of instance-output pairs sampled from \(\rho\) before
the start of the active learning loop, denoted by \(\poolset_{0} = (x_{i},
y_{i})_{i=1}^{n} \sim \rho^{n}\) and the corresponding set of only instances as
\(\poolsetx_{0}\). The pool set acts as the pool of available instances that
the algorithm can query at each \(t\), so that \(\poolsetx_{t}\) are the
instances not labeled at start of iteration \(t\). At time \(t\) the algorithm
only have access to \(\poolsetx_{t}\) and will ask for labels of an instance
from \(\poolsetx_{t}\).
\item[{Oracle}] The oracle is a function which takes as an input an instance and
samples from the conditional distribution \(\rho(Y | X = x)\). We will assume no
output noise in our derivation, so the conditional distribution is
deterministic. Formally we define \(\oracle: \X \to \Y\) where \(\oracle(x) =
f(x)\) where \(f(x)\) is the true relationship.
\item[{Querying strategy}] The strategy of how we decide what instance to label
at time \(t\) which we denote \(\querstrat\). We will make the assumption that
\(\querstrat\) chooses instances at \(t\) through a non-adaptive utility
function\footnote{This simply means that the querying strategy is independent of
the history of outputs from the queried instances.} \(\upsilon: \X \to \R\),
which implicitly depends on the instances already queried although we omit this
for notational ease, and that \(\querstrat\) chooses an instance \(x_{t} \in
\argmax_{x \in \poolsetx_{t}} \nu(x)\). We call the chosen instance \(x^{q}_{t}\) and send it
to the oracle to label giving us \(y^{q}_{t} = \oracle(x^{q}_{t})\).
\item[{Train set}] This is the analogue of the train set used in supervised
learning. We denote this by \(\trainset_{t}\) and the corresponding set of
instances as \(\trainsetx_{t}\). The reason for including \(t\) as a subscript
is that \(\trainset_{t}\) is not static but is grown by one input-output pair at
the end of each iteration.
\end{description}

In this way a pool-based active learning algorithm is defined by a tuple
\((\algo, \querstrat)\) where \(\algo\) has the same role as in supervised
learning, outputting a hypothesis \(\hat{h}_{t}\) depending on the current train
set \(S_{t}\), but which is now built depending on the querying strategy
\(\querstrat\).
\begin{algorithm}
  \caption{Pool based active learning}\label{alg:active-learning}
  \begin{algorithmic}[1] \Procedure{ActiveLearning}{$\poolset_{0}$, $\algo$,
$\querstrat$} \State $\trainset_0 \gets \emptyset$ \State $n \gets
|\poolset_{0}|$ \State $t \gets 1$ \While{$t \leq n$} \State $x^q_t \gets
\querstrat(\poolsetx_{t})$ \State $y^q_t \gets \oracle(x^q_t)$ \State
$\trainset_{t} \gets \trainset_{t-1} \cup \{(x^q_t, y^q_t)\}$ \State $\poolset_t
\gets \poolset_t \setminus \{(x^q_t, y^q_t)\}$ \State $t \gets t + 1$ \EndWhile
\EndProcedure
  \end{algorithmic}
\end{algorithm}

The performance of an active learning algorithm will be considered to be its
\emph{error bound} which is defined similarly as in \ref{def:error-bound-sl},
but we now also let this depend on the size of \(\trainset_{t}\) in addition to
the size original pool set
\begin{definition}
  \label{def:error-bound-al} An active learning error bound \(\epsilon(n, t,
\delta)\) is a function depending on an algorithm \(\algo\) and a querying
strategy \(\querstrat\) that given a \(\delta \in [0, 1]\) assures that the
excess risk will be less than \(\epsilon\) with probability greater than \(1 -
\delta\). Or as a mathematical expression, let \(\trainset_{t}\) be the train
set at the end of time \(t\),
  \begin{equation}
  \label{eq:error-bound-al} \Pr_{S \sim \rho^n}(\err{}{\algo(S_{t})} -
\err{}{h^{\ast}} \leq \epsilon(n, t, \delta)) \geq 1 - \delta.
  \end{equation}
\end{definition} This gives us a way to compare active learning algorithms to
each other, and note that this also gives us a way to compare supervised
learning to active learning by specifying \((\algo, \querstrat)\) and comparing
the active learning error bound \(\epsilon(n_{al}, t, \delta)\) to the
supervised learning bound \(\epsilon(n_{sl}, \delta)\) where \(n_{al} =
|\poolset_{0}|\) is the number of original samples in the pool and \(n_{sl} =
|\trainset|\) is the number of samples in the train set of supervised learning.

For active learning there has been several works based on generalisation bounds
\cite{ganti12_upal,xu19_towar_effic_evaluat_risk_via_herdin,gu12_towar} but
these are aimed at classification or graph-based prediction. For the setting of
regression, where a normal choice is to consider square loss and kernel ridge
regresion there is recently some promising results of
\cite{viering17_nuclear_discr_activ_learn} where they derive a bound of the form
\begin{equation}
\label{eq:viering-mmd-gen-bound} \err{\poolset_{0}}{h} \leq
\err{\trainset_{t}}{h} + \MMD{\poolsetx_{0}}{\trainsetx_{t}}{} + \eta_{MMD}
\end{equation} and propose to base the querying strategy of optimising the MMD
term. By making certain assumptions on the space for the MMD it can be shown
that \(\eta_{MMD} = 0\) and that the active learning bound decomposes into term
which may be interpreted as the empirical fit and the drift from the instance
set \(\trainsetx_{t}\) to the original set of instances \(\poolsetx_{0}\). This
makes it very clear that active learning in this setting is closely related to
domain adaptation and domain drift. \cite{cortes19_adapt_based_gener_discr} show
other measures in addition to MMD which may be used to control this domain drift
through upper bounding the empirical risk in similar ways, but MMD is an
appealing choice since it has an analytical form and analysing it comes down to
qualities of the kernel matrix.

From a domain adaptation perspective, active learning under no noise is
equivalent to the case of \emph{covariate shift}
\cite{gretton09_covar_shift_by_kernel_mean_match}.

\section{Herding and Conditional Gradient Methods in RKHS's}
\subsection{Kernel Herding}
Optimising the MMD term globally in the sense of
picking \(k\) instances out of \(\poolsetx_{0}\) can be shown to reduce to a
quadratic binary programming problem and is thus NP-hard in general
\cite{chaovalitwongse09_quadr_integ_progr}. In
\cite{chattopadhyay13_batch_mode_activ_sampl_based} they relax this problem in
two ways to choose instances, one through solving a convex quadratic programming
problem and one through solving a linear programming problem, both which can be
solved efficiently. While they consider classification, since the algorithm is
non-adaptive and based on controlling the same MMD term as in
\ref{eq:viering-mmd-gen-bound} it is directly comparable to what we are trying
to do. We will insted consider optimising this term through the use of
Frank-Wolfe \cite{frank56_algor_quadr_progr,jaggi13_revis_frank_wolfe} of which
herding can be shown to be a special case
\cite{bach12_equiv_between_herdin_condit_gradien_algor}.

Herding started as an approach to generate deterministic \emph{pseudo-samples}
from observed moments of a posited model, without needing to first find the
maximum likelihood estimates of the model parameters, but was limited in that
\(\X\) had to be finite \cite{welling09_herdin}. The Herding algorithm was later
generalised to arbitrary domains \(\X\) by
\cite{chen12_super_sampl_from_kernel_herdin} noting that we can let \(\mat{w}\),
the parameters of the model, be an element of a \textbf{finite-dimensional} RKHS
\(\rkhs{H}\) with \(\phi\) being the feature map of the corresponding
reproducing kernel \(K\). Given a probability distribution \(\rho_{\X}\) on
\(\X\), kernel herding is the following
\begin{algorithm}
  \caption{KernelHerding}\label{alg:kernel-herding}
  \begin{algorithmic}[1] \Procedure{KernelHerding}{$\rkhs{H}$, $\rho_{\X}$,
$\mat{w}_{0} \in \rkhs{H}$} \State $\mu_{\rho_{\X}} = \E_{X \sim
\rho_{\X}}[\phi(X)]$ \State $t \gets 1$ \While{$t < \infty$} \State $x_{t} \gets
\argmax_{x \in \X} \scal{\mat{w}_{t-1}}{\phi(x)}_{\rkhs{H}}$ \State $\mat{w}_{t}
\gets \mat{w}_{t-1} + \hat{\mu} - \phi(x_{t})$ \State $t \gets t + 1$ \EndWhile
\EndProcedure
  \end{algorithmic}
\end{algorithm}

While this is very similar to the original herding algorithm, they also show
that we can view the herding algorithm as a way of sequentially minimizing the
squared error
\begin{equation}
\label{eq:squared-error-herding} \norm{\mu_{\rho_{\X}} - \frac{1}{t}
\sum_{j=1}^{t} \phi(x_{j})}_{\rkhs{H}}^{2}.
\end{equation} If we look at the form of this, we can see that by
\ref{eq:mmd-is-norm-between-feature-maps} this is the same as
\begin{equation} \MMD{\rho_{\X}}{\hat{\rho}_{t}}{\rkhs{H}}^{2},
\end{equation} where \(\hat{\rho}_{t} = \frac{1}{t} \sum_{j=1}^{t}
\delta_{x_{j}}\). Furthermore, while Monte-Carlo sampling of \(\hat{\rho}_{t}\), that
is sampling \(\iid\) from \(\rho_{\X}\), decreases the squared MMD as
\(O(\frac{1}{t})\), KH improves upon this with \(O(\frac{1}{t^{2}})\) which can
be explained by the KH algorithm using \emph{negative} auto-correlations to
explore parts of the space where samples haven't been drawn from before, linking
it to Quasi Monte Carlo methods and Low Discrepancy Sequences
\cite{chen12_super_sampl_from_kernel_herdin,bach12_equiv_between_herdin_condit_gradien_algor}.
They state the following which relates how herding and KH can be used in active
learning,
\begin{corollary}[\citep{chen12_super_sampl_from_kernel_herdin}, Corollary 6] An
active learning algorithm selecting labels in accordance with the herding
algorithm has guaranteed rate of convergence in terms of \(O(\frac{1}{t})\).
Moreover, the submodular greedy algorithm of
\cite{krause08_near_optim_sensor_placem_gauss_proces} has therefor also at least
the same approximation rate since it is within a constant fraction \((1 -
e^{-1})\) of optimality.
\end{corollary}
which is in essence what we will be doing in this dissertation by building
\(S_t\) using KH and FW, but trying to extend
it to the generalisation error instead of considering the empirical excess risk.

\subsection{Frank-Wolfe} Frank Wolfe
\cite{frank56_algor_quadr_progr} is an algorithm for solving general constrained
convex optimization problems of the form
\begin{equation}
\label{eq:constrainted-convex-opt-problem} \min_{z \in \mathcal{C}} J(z),
\end{equation} where \(\mathcal{C}\) is some compact convex subset of a Hilbert
space \(\mathcal{Z}\) and \(J\) is assumed to be convex over \(\mathcal{C}\) and
continuously differentiable. FW sets itself apart as it only requires access to
the gradient \(\nabla J\) on \(\mathcal{C}\) and be able to find the minimiser
of quantities of the form \(\scal{s}{\nabla J(z)}_{\mathcal{Z}}\), and does not
require any projection steps while converging to \(J(z^{\ast}) = \min_{z
\in \mathcal{C}} J(z)\) as \(O(\frac{1}{t})\) given certain regularity
conditions \cite{jaggi13_revis_frank_wolfe}.

\begin{algorithm}
  \caption{FrankWolfe}\label{alg:frank-wolfe}
  \begin{algorithmic}[1] \Procedure{FrankWolfe}{$\mathcal{Z}$, $\mathcal{C}$, $J$,
$\rho_{t}$} \State $t \gets 1$ \While{$t \leq T$} \State $\tilde{g}_{t} \gets
\argmin_{z \in \mathcal{Z}} \scal{z}{\nabla J(x_{t-1})}_{\X}$ \State $g_{t}
\gets (1 - \rho_{t-1})g_{t-1} + \rho_{t-1}\tilde{g}_t$ \State $t \gets t + 1$
\EndWhile \EndProcedure
  \end{algorithmic}
\end{algorithm}

By rewriting the Kernel Herding algorithm,
\cite{bach12_equiv_between_herdin_condit_gradien_algor} showed that KH is a
subset of a family of Frank-Wolfe algorithms over an RKHS \(\rkhs{H}\). This is
the setting we will work in and we describe the assumptions they use which will
also be ours.

\begin{assumption}
\label{as:fw-kernel-herding} We have a set \(\X\) and a mapping \(\phi: \X \to
\rkhs{H}\), where \(\rkhs{H}\) is an RKHS with reproducing kernel \(K\). We
assume that the data is uniformly bounded in the feature space, which means that
for any \(x \in \X\), \(\norm{\phi(x)}_{\rkhs{H}} \leq R\) for some \(R > 0\).
\end{assumption}

We denote \(\mathcal{M}\) to be the marginal polytope, which is the
\hyperref[def:conv-hull]{convex hull} of all vectors \(\phi(x)\) for \(x \in X\)
a finite set \(X \subseteq \X\). For any \(f \in \rkhs{H}\) the following holds
\begin{fact}
\label{fact:optimal-element-is-extremum} The supremum of \(f(x)\) over
\(\mathcal{C}\) is attained by an extremum,
\begin{equation}
\label{eq:optimal-element-is-extremum} \sup_{x \in \X} f(x) = \sup_{g \in
\mathcal{M}} \scal{f}{g}_{\rkhs{H}}.
\end{equation}
\end{fact} the authors then show equivalence between KH and FW by considering
the optimization problem where we equate \(\rkhs{H}\) with \(\mathcal{Z}\) and
\(\mathcal{M}\) with \(\mathcal{C}\)
\begin{equation}
\label{eq:fw-kh-optimisation-problem} \min_{g \in \mathcal{M}} J(g) =
\min_{g \in \mathcal{M}}\frac{1}{2}\norm{g - \mu_{\rho_{X}}}_{\rkhs{H}}^{2}
\end{equation} where \(\mu_{\rho_X}\) is the mean embedding of \(\rho_X\)
into \(\rkhs{H}\), leading to the FW update equations
\begin{align} \tilde{g}_{t} & = \argmin_{g \in \mathcal{M}} \scal{g_{t-1} -
\mu_{X}}{g}_{\rkhs{H}} \label{al:fw-update-step-1}\\ g_{t} & = (1 -
\rho_{t-1}) g_{t-1} + \rho_{t-1} \tilde{g}_{t} \label{al:fw-update-step-2}
\end{align} and using \(\rho_{t} = \frac{1}{t+1}\) we recover the KH algorithm.
Note that for each \(t\) in \ref{al:fw-update-step-1} the chosen
\(\tilde{g}_{t}\) is assured to lie on the extremum according to
\ref{fact:optimal-element-is-extremum}. This shows that as \(X\) is a finite
set of \(n\) points, running FW produces a sequence \((x_{t})_{t=1}^{n}\) where
\(x_{t}\) is associated with the corner of \(\mathcal{M}\) such that
\(\tilde{g}_{t} = \phi(x_{t})\) which means that when \(X = \poolsetx_{0}\) we
can use the FW algorithm to choose points. In the utility framework, we let
\(\upsilon\) be the function
\begin{equation}
\label{eq:fw-utility-function} \upsilon(x) = -\scal{g_{t-1} -
\mu_{\rho_{\poolsetx_{0}}}}{\phi(x)}_{\rkhs{H}},
\end{equation} and using this we can create an algorithm by letting
\(\querstrat\) be the querying strategy that has utility function
\ref{eq:fw-utility-function} of \ref{alg:active-learning}. We call this family
of algorithms Frank-Wolfe Active Learning (FWAL), where we indicate the
algorithm by specifying \(\rho_{t}\).

\subsection{Convergence rates for Frank-Wolfe}
\label{sec:org9640d5f} While KH had a convergence rate of \(O(\frac{1}{t})\), FW
has different rates depending on the step-size chosen at each \(t\). In fact, FW
can have much faster convergence if we use line-search as opposed to a fixed
step-size. Let the constant
\begin{equation}
\label{eq:d-radius-of-biggest-interior-ball} d = \argmin_{g \in \relbound
\mathcal{M}} \norm{g - \mu_{\rho_{\X}}}_{\rkhs{H}}
\end{equation} where \(\relbound \mathcal{M}\) is the
\hyperref[def:relative-boundary]{relative boundary} of \(\mathcal{M}\). Note
that this depends on \(\mathcal{M}\) and we can have arbitrarily small \(d\).
The following table denotes upper bounds on the rate of convergence rates of
\(J(g_{t}) = \frac{1}{2} \norm{g_{t} - \mu_{\rho_{\X}}}^{2}_{\rkhs{H}}\)
using FW (any), FW-kh (\(\rho_{t} =
\frac{1}{t+1})\), FW-line search (hereafter \emph{FW-ls}), where \(g_{t}\) is
the output from the algorithms after \(t\) steps and sufficient (but not
necessary) conditions for this to hold is laid out in the following table

\begin{table}[htbp]
\caption{\label{tbl:FW-convergence-table} Upper bound on convergence rates using
Frank Wolfe} \centering
\begin{tabular}{lll} \hline Algorithm & Convergence (upper bound) & Additional
Condition to \ref{as:fw-kernel-herding}\\ \hline FW (any) & \(4\frac{R^{2}}{t}\)
& None\\ FW-kh & \(\frac{2R^{4}}{d^{2}t^{2}}\) & \(d > 0\)\\ FW-ls & \(R^{2}
\exp(-\frac{d^{2}t}{R^{2}})\) & \(d > 0\)\\ \hline
\end{tabular}
\end{table}

Finally, \cite{bach12_equiv_between_herdin_condit_gradien_algor} prove the following two propositions
\begin{proposition}
\label{prop:finite-dim-fw-has-speedup} Assume that \(\rkhs{H}\) is
finite-dimensional, that \(\X\) is a compact topological measure space with a
continuous kernel function \(K\) and that the distribution \(\rho_{\X}\) has
full support on \(\X\). Then \(d > 0\).
\end{proposition}

\begin{proposition}
\label{prop:infinite-dim-fw-has-probably-not-speedup} Assume that \(\rkhs{H}\)
is infinite-dimensional, that \(\X\) is a compact subspace of \(\R^D\) with a
continuous kernel function \(K\) on \(\X \times \X\). Then \(d = 0\).
\end{proposition}

