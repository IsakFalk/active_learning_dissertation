\documentclass[11pt,mres,a4paper,oneside]{ucl_thesis}

%%%%%%%%%%%%%
% Packages  %
%%%%%%%%%%%%%
\usepackage{etex}
\usepackage{placeins}

% Load my own package as well
\usepackage{isak}
% Load to get outlines right
\usepackage{bookmark}

% allows us to load csv files directly
\usepackage{csvsimple}

% This package means empty pages (pages with no text) won't get stuff
%  like chapter names at the top of the page. It's mostly cosmetic.
\usepackage{emptypage}

% The float package improves LaTeX's handling of floats,
%  and also adds the option to *force* nLaTeX to put the float
%  HERE, with the [H] option to the float environment.


%  Use these two packages together --
% they define symbols
%  for e.g. units that you can usein both text and math mode.
\usepackage{gensymb}

\usepackage{textcomp}

% The setspace package lets you use 1.5-sized or double line spacing.
\usepackage{setspace}
\setstretch{1.5}

% That just does body text -- if you want to expand *everything*,
%  including footnotes and tables, use this instead:
%\renewcommand{\baselinestretch}{1.5}

% The linenumbers command from the lineno package adds line numbers
%  alongside your text that can be useful for discussing edits 
%  in drafts.
% Remove or comment out the command for proper versions.
%\usepackage[modulo]{lineno}
% \linenumbers 


% The multirow package adds the option to make cells span 
%  rows in tables.
\usepackage{multirow}

% Create subfigures
\usepackage{subcaption}

% The natbib package allows book-type citations commonly used in
%  longer works, and less commonly in science articles (IME).
% e.g. (Saucer et al., 1993) rather than [1]
% More details are here: http://merkel.zoneo.net/Latex/natbib.php
\usepackage{natbib}

% The bibentry package (along with the \nobibliography* command)
%  allows putting full reference lines inline.
%  See: 
%   http://tex.stackexchange.com/questions/2905/how-can-i-list-references-from-bibtex-file-in-line-with-commentary
\usepackage{bibentry}

% The isorot package allows you to put things sideways 
%  (or indeed, at any angle) on a page.
% This can be useful for wide graphs or other figures.
%\usepackage{isorot}

% The caption package adds more options for caption formatting.
% This set-up makes hanging labels, makes the caption text smaller
%  than the body text, and makes the label bold.
% Highly recommended.
% \usepackage[format=hang,font=small,labelfont=bf]{caption}

% and specially defined macros
% We define all the macros here so we can change them later

\onehalfspacing

\newcommand{\rkhs}[1]{\ensuremath{\mathcal{#1}}}
\newcommand{\mat}[1]{\ensuremath{\mathbf{#1}}}
\newcommand{\dist}{\ensuremath{\rho}}
\newcommand{\iid}{iid}
\newcommand{\algo}{\ensuremath{\mathcal{A}}}
\newcommand{\risk}[2]{\ensuremath{\err{#1}{#2}}}
\newcommand{\emprisk}[2]{\ensuremath{\errh{#1}{#2}}}
\newcommand{\trainset}{\ensuremath{S}}
\newcommand{\poolset}{\ensuremath{P}}
\newcommand{\unsampset}{\ensuremath{U}}
\newcommand{\queryset}{\ensuremath{Q}}
\newcommand{\trainsetx}{\ensuremath{S^x}}
\newcommand{\poolsetx}{\ensuremath{P^x}}
\newcommand{\unsampsetx}{\ensuremath{U^x}}
\newcommand{\oracle}{\ensuremath{\mathcal{O}}}
\newcommand{\querstrat}{\ensuremath{\mathcal{Q}}}

%%%%%%%%%%%%%%%%%%%%%
% Proof in appendix %
%%%%%%%%%%%%%%%%%%%%%

% \usepackage{environ}

% \makeatletter
% \providecommand{\@fourthoffour}[4]{#4}
% % We define an addition for the theorem-like environments; when
% % \newtheorem{thm}{Theorem} is declared, the macro \thm expands
% % to {...}{...}{...}{Theorem} and with \@fourthoffour we access
% % to it; then we make available \@currentlabel (the theorem number)
% % also outside the environment.  
% \newcommand\fixstatement[2][\proofname\space of]{%
%   \ifcsname thmt@original@#2\endcsname
%     % the theorem has been declared with \declaretheorem
%     \AtEndEnvironment{#2}{%
%       \xdef\pat@label{\expandafter\expandafter\expandafter
%         \@fourthoffour\csname thmt@original@#2\endcsname\space\@currentlabel}%
%       \xdef\pat@proofof{\@nameuse{pat@proofof@#2}}%
%     }%
%   \else
%     % the theorem has been declared with \newtheorem
%     \AtEndEnvironment{#2}{%
%       \xdef\pat@label{\expandafter\expandafter\expandafter
%         \@fourthoffour\csname #1\endcsname\space\@currentlabel}%
%       \xdef\pat@proofof{\@nameuse{pat@proofof@#2}}%
%     }%
%   \fi
%   \@namedef{pat@proofof@#2}{#1}%
% }

% % We allocate a block of 1000 token registers; in this way \prooftoks
% % is 1000 and we can access the following registers of the block by
% % \prooftoks+n (0<n<1000); we'll use a dedicated counter for it
% % that is stepped at every proof
% \globtoksblk\prooftoks{1000}
% \newcounter{proofcount}

% % We gather the contents of the proof as argument to \proofatend
% % and then we store
% % "\begin{proof}[Proof of <theoremname> <theoremnumber>]#1\end{proof}"
% % in the next token register of the allocated block
% \NewEnviron{proofatend}{%
%   \edef\next{%
%     \noexpand\begin{proof}[\pat@proofof\space\pat@label]%
%     \unexpanded\expandafter{\BODY}}%
%   \global\toks\numexpr\prooftoks+\value{proofcount}\relax=\expandafter{\next\end{proof}}
%   \stepcounter{proofcount}}

% % \printproofs simply loops over the used token registers of the
% % block, freeing their contents
% \def\printproofs{%
%   \count@=\z@
%   \loop
%     \the\toks\numexpr\prooftoks+\count@\relax
%     \ifnum\count@<\value{proofcount}%
%     \advance\count@\@ne
%   \repeat}
% \makeatother

% \fixstatement{theorem}
% \fixstatement{lemma}
% \fixstatement{proposition}
% \fixstatement{corollary}

%%%%%%%%%%%%%%%%%%%%%%
% Links and Metadata %
%%%%%%%%%%%%%%%%%%%%%%

\usepackage{bibentry}
\makeatletter\let\saved@bibitem\@bibitem\makeatother
% \usepackage[pdftex,hidelinks]{hyperref}
\makeatletter\let\@bibitem\saved@bibitem\makeatother
\makeatletter
\AtBeginDocument{
  \hypersetup{
    pdfsubject={Machine Learning},
    pdfkeywords={machine learning, SLT, MMD, active learning},
    pdfauthor={Author},
    pdftitle={Title},
  }
}
\makeatother

%%%%%%%%%%%%%%%%%%
% Float Settings %
%%%%%%%%%%%%%%%%%%

% General parameters, for ALL pages:
\renewcommand{\topfraction}{0.9}	% max fraction of floats at top
\renewcommand{\bottomfraction}{0.8}	% max fraction of floats at bottom

% Parameters for TEXT pages (not float pages):
\setcounter{topnumber}{2}
\setcounter{bottomnumber}{2}
\setcounter{totalnumber}{4}     % 2 may work better
\setcounter{dbltopnumber}{2}    % for 2-column pages
\renewcommand{\dbltopfraction}{0.9}	% fit big float above 2-col. text
\renewcommand{\textfraction}{0.07}	% allow minimal text w. figs

% Parameters for FLOAT pages (not text pages):
\renewcommand{\floatpagefraction}{0.7}	% require fuller float pages
% N.B.: floatpagefraction MUST be less than topfraction !!
\renewcommand{\dblfloatpagefraction}{0.7}	% require fuller float pages

%%%%%%%%%%%%%%%%
% Bibliography %
%%%%%%%%%%%%%%%%

\bibliographystyle{abbrv}

%%%%%%%%%%%%%%%%
% TOC settings %
%%%%%%%%%%%%%%%%

\setcounter{secnumdepth}{3}
\setcounter{tocdepth}{3}

%%%%%%%%%%%%%%%%%%%%%%%%%%%%%%%%%%%%%%%%%%%%%%%%%%%%%
% Fix algorithm and hyperref interaction of counter %
%%%%%%%%%%%%%%%%%%%%%%%%%%%%%%%%%%%%%%%%%%%%%%%%%%%%%

\makeatletter
\newcounter{algorithmicH}% New algorithmic-like hyperref counter
\let\oldalgorithmic\algorithmic
\renewcommand{\algorithmic}{%
  \stepcounter{algorithmicH}% Step counter
  \oldalgorithmic}% Do what was always done with algorithmic environment
\newcommand{\theHALG@line}{ALG@line.\thealgorithmicH.\arabic{ALG@line}}
\makeatother

\begin{document}

\nobibliography*

% I may change the way this is done in a future version, 
%  but given that some people needed it, if you need a different degree title 
%  (e.g. Master of Science, Master in Science, Master of Arts, etc)
%  uncomment the following 3 lines and set as appropriate (this *has* to be before \maketitle)
% \makeatletter
% \renewcommand {\@degree@string} {Master of Things}
% \makeatother

\title{Active Learning Regression by Mitigating Domain Drift}
\author{John Isak Texas Falk}
\department{Department of Computer Science}

\maketitle
\makedeclaration

\begin{abstract} % 300 word limit
  % Motivation / Purpose
  Active learning enables an algorithm to choose a subset training set of a dataset to
  label while retaining most of the performance as if training on the full dataset.
  When labelling data is costly but finding instances cheap, this can lead to
  massive cost reduction over supervised learning.
  % Problem
  While active learning for classification has been explored both theoretically and practically, there
  is much less work done on regression.
  % Method
  In this work we focus on noiseless regression where we devise an algorithm to
  pick instances to label and train a Kernel Ridge Regression on this data.
  Central to this is the Frank Wolfe algorithm which greedily optimises the Maximum Mean
  Discrepancy between the full dataset and train set. This leads to an algorithm
  that optimises an upper bound on the empirical risk of the estimator trained
  on the built train set. We try to control the true generalisation error but
  current results from convergence of Frank Wolfe means we can't say whether or
  not this improves upon random sampling.
  % Results
  We apply the algorithm to a wide range of datasets and compare it to leverage
  score sampling. We show that the algorithm performs competitevely with other
  algorithms and the benefit of using active learning for regression. We also
  show that it dominates other methods in the agnostic setting of both
  regression and classification.
  % Conclusion
  We show that using Frank Wolfe to mitigate domain drift is competitive and
  works well in practice, with learning curves that show greater performance
  than random sampling on a wide range of datasets in both agnostic and
  realisable setting.
\end{abstract}

\begin{impactstatement}

% 	UCL theses now have to include an impact statement. \textit{(I think for REF reasons?)} The following text is the description from the guide linked from the formatting and submission website of what that involves. (Link to the guide: {\scriptsize \url{http://www.grad.ucl.ac.uk/essinfo/docs/Impact-Statement-Guidance-Notes-for-Research-Students-and-Supervisors.pdf}})

% \begin{quote}
% The statement should describe, in no more than 500 words, how the expertise, knowledge, analysis,
% discovery or insight presented in your thesis could be put to a beneficial use. Consider benefits both
% inside and outside academia and the ways in which these benefits could be brought about.

% The benefits inside academia could be to the discipline and future scholarship, research methods or
% methodology, the curriculum; they might be within your research area and potentially within other
% research areas.

% The benefits outside academia could occur to commercial activity, social enterprise, professional
% practice, clinical use, public health, public policy design, public service delivery, laws, public
% discourse, culture, the quality of the environment or quality of life.

% The impact could occur locally, regionally, nationally or internationally, to individuals, communities or
% organisations and could be immediate or occur incrementally, in the context of a broader field of
% research, over many years, decades or longer.

% Impact could be brought about through disseminating outputs (either in scholarly journals or
% elsewhere such as specialist or mainstream media), education, public engagement, translational
% research, commercial and social enterprise activity, engaging with public policy makers and public
% service delivery practitioners, influencing ministers, collaborating with academics and non-academics
% etc.

% Further information including a searchable list of hundreds of examples of UCL impact outside of
% academia please see \url{https://www.ucl.ac.uk/impact/}. For thousands more examples, please see
% \url{http://results.ref.ac.uk/Results/SelectUoa}.
% \end{quote}

The knowledge and algorithms in this dissertation have the potential to have
industrial use in any setting where labelling is costly, or when there is a
fixed budget of how many instances it is possible to label. This is a setting
which occur in many different areas, to name a few, medical healthcare, process
engineering, natural language processing and data mining of documents. As
this contribution is derived from theoretical considerations, it can be used
both locally and internationally, by anyone anywhere.

From an academic point of view, the information and knowledge in this
dissertation will act as a springboard for future researchers and students, and
add to the existing pool of knowledge in the field of machine learning. The hope
it that this will foster further ideas on how to improve the field of active
learning for regression, building on results derived here and extending it to
more general and different settings.
\end{impactstatement}

\begin{acknowledgements}
  To my parents, thank you.
  \newline
  \newline
  To Carlo and Massi, Cheers!
\end{acknowledgements}

\setcounter{tocdepth}{2} 
% Setting this higher means you get contents entries for
%  more minor section headers.

\tableofcontents
\listoffigures
\listoftables


\chapter{Introduction}
\label{ch:introduction}

\section{Background}
\label{sec:background}
A cornerstone of machine learning is the paradigm of supervised learning. In
supervised learning the goal is to learn an input / output (also called label)
relationship depicted by \(f : \X \to \Y\), where \(f\) represents a
relationship mapping from the input space \(\X\) the output space \(\Y\). We
learn this \(f\) from a collection of \(n\) input-output pairs \((x, y) \in \X
\times \Y\) sampled from the underlying phenomenon giving rise to \(f\).
However, in supervised learning the algorithm does not have any way of choosing how this
dataset is created. This puts some restrictions on its use in practice as certain
algorithms are too computationally expensive to train on full datasets or the
case when instances are cheap to collect, but expensive to label.

The second setting occurs frequently in practice. Today data is being created in
abundance due to human use of social media, email and devices
\cite{desjardins19_how}, and due to data being collected from industrial processes
\cite{qin14_proces_data_analy_era_big_data}. Much of this data is unstructured and
in most cases the output we want to predict is not available directly. An
example of this is the text data being generated by the web; platforms such as
Twitter produces massive amounts of free text which is not labelled at all. For
example, when trying to do sentiment analysis on sentences from Twitter, it is
necessary to heuristically assign a label to an instance or send it to an expert
for labelling \cite{pak10_twitt}. This begs the question if we can design
strategies for what instances we should label first in order to learn patterns
faster, in comparison to labelling instances at random.

Active learning \cite{cohn94_improv_gener_with_activ_learn} is a machine learning
paradigm which generalises supervised learning by allowing the algorithm in some
sense to decide what instances to query for the corresponding output. The hope
is that this will lead to close-to-optimal performance in much fewer labels
compared to a supervised learning algorithm trained on a dataset of the same
size. In this way, it is a question of the information content of the dataset.
By choosing instances carefully we may create a dataset that is richer in
information with respect to the pattern we want to learn than a dataset of same
size, given to us from nature.

As such, active learning is similar to semi-supervised learning
\cite{chapelle09_semi_super_learn_o} in that it tries to make efficient use of all
available data. However, where semi-supervised learning uses the unlabelled data
to improve performance by inferring aspects of the marginal distribution, active
learning builds a dataset by querying the unlabelled data and getting the
corresponding output. It is possible to combine the two \cite{zhu03_combin}.

As a concrete example, imagine the following problem: We are on \([0,
1] \subseteq \R\) and we know that there exist some true relationship
\(f(x) = \mathbf{1}(w^{\ast} \leq x)\) where \(w^{\ast} \in (0, 1)\),
but we do not observe the output directly but have to ask for it, an operation we
call \emph{querying} an instance. 

Nature draws instances \(x \sim \rho_{\X}\) where \(\rho_{\X}\) is some
distribution over the domain \(\X = [0, 1]\). Our goal is to find a good guess
\(h_{w}(x) = \mathbf{1}(w \leq x)\) so that we incur a small loss,
\(\Pr_{\rho_{\X}}(h_{w}(x) \neq f(x))\), in as few queries of labels as
possible. Since we know the form of the relationship, we also know that if we
have queried an instance \(x_{1}\) which leads to the pair \((x_{1}, f(x_{1}))\)
where \(f(x_{1}) = 0\) (which means that \(x_{1} \leq w^{\ast}\)) then we also
know that for any other \(x_{2} \leq x_{1} \leq w^{\ast}\), \(f(x_{2}) = 0\).
This means that \textbf{querying any instance smaller than} \(x_{1}\) \textbf{is unnecessary}.

It can be shown that for a set of \(O(\frac{1}{\epsilon})\) unlabelled
instances, an active learning algorithm have to query only \(O(\log
\frac{1}{\epsilon})\) instances in order to find a \(h_{w}\) such that \(\Pr_{\rho_{\X}}(h_{w}(x)
\neq f(x)) < \epsilon\) with high probability. Compared to supervised
learning which requires \(O(\frac{1}{\epsilon})\) this is an exponential
speedup \cite{dasgupta06_coars,dasgupta05_analy}!

In practice, it is often not so simple. Active learning has been shown to not
yield better performance than supervised learning when considering worst case
scenarios\footnote{Which is done in statistical learning theory.} which hints
that only when we can identify and exploit structure of the problem, such as the
knowledge of the threshold function in the example above, can we hope to improve
over normal supervised learning. In the end this has come down to identifying
conditions on the data-generating distribution and the target class which
enables active learning to improve over supervised learning
\cite{balkan15_activ_learn_moder_learn_theor}.

\section{Research Focus}
The theoretical foundations of active learning for binary classification is
fairly well-understood, with results which are essentially always at least as
good as supervised learning \cite{balcan10_true_sampl_compl_activ_learn}. However,
results for regression is more sparse and less well-developed. For regression,
the assumption is often that the bias term of the error decomposition is
negligible, see for example \cite{cohn96_activ_learn_with_statis_model}. There has
been attempts to control for this directly, but relies on assumptions that
basically reduces regression to classification
\cite{willett06_faster_rates_regres_activ_learn}. Thus a natural direction is to
consider active learning for regression based on statistical learning theory and
investigate how the bias term can be understood and controlled.

In general, statistical learning theory considers how to minimise the risk,
which is a much harder problem than minimising the empirical risk. This stems
from the problem of generalisation and the fact that good performance on the
train set does not necessarily imply good performance on the test set. However,
given that we know the data-generating distrubution, active learning can be
shown to be closely related to quadrature \cite{briol15_frank_wolfe_bayes} and
optimal experimental design \cite{fedorov10_optim_exper_desig}. The main
difference is the setting and assumptions, for quadrature the distribution is in
general known and optimal experimental design assumes very
stringent conditions on the problem at hand.

In this sense we are interested in active learning procedures that are motivated
by statistical learning theory and grounded on error bounds through the use of
controlling the sampling and approximation error. Ideally we would like to find
necessary assumptions for active learning to yield performance improvements over
supervised learning, that is, an active learning algorithm with statistical
guarantees that performs better than supervised learning in probability.
Specifically, in this dissertation we aim to do the following 

\begin{itemize}
\item Review the literature on active learning for regression in a theoretical
  setting including necessary results from RKHS theory and optimisation
\item Specify and find necessary conditions for enabling error bounds of the risk,
  which we term the objective
\item Derive an active learning algorithm for regression based on optimising objective
\item Empirically evaluate and compare this algorithm to other active learning
  algorithms for regression
\end{itemize}

\section{Research Aims and Individual Research Objectives}
The aim of this dissertation is to critically analyse and summarise the
literature on active learning in the setting of regression and provide
conditions and assumption necessary for a theoretically grounded active learning
algorithm with guarantees to be possible. We do this by leaning on the
well-developed field of RKHS theory, which is in some sense an ideal setting for
statistical learning due to various theoretical properties such as the
reproducing property and the representer theorem.

We further aim to derive bounds on the quantity of interest, a modified excess
risk for active learning, and provide a framework for comparing active learning
algorithms to supervised learning algorithms by framing active learning as a
generalisation of supervised learning. In addition to this, we will make
connections with other fields and use recent advances in machine learning, based
on the Frank Wolfe algorithm applied to RKHS's.

Finally we aim to produce an algorithm which satisfy theoretical guarantees
which is shown to be empirically competitive with other similar algorithms. This
will be done through a rigorous investigation of how the algorithm perform in
practice by evaluating the algorithm on regression and finally classification
tasks, displaying the theoretical performance in practice and show that when
applied to settings where the theoretical assumptions do not hold, it's still
competitive.

\section{Value of this Research}
The contribution of this research will be the following:
\begin{itemize}
\item Produce a review of the existing methods and approaches to active learning in
  regression which is currently lacking to the extent that binary classification
  active learning is a synonym with active learning
\item Investigate and develop an area of active learning that is underdeveloped
  compared to binary classification
\item Synthesise methods from statistical learning theory, kernel theory and
  optimisation in order to produce novel algorithms for active learning
\item Identify failure cases and set a basis for future work to build upon
\end{itemize}
\chapter{Literature Review}
\label{ch:lit_rev}

In this chapter, we review and compile the work done on active learning for
regression based on RKHS theory. We divide this into three parts: the first
focusing on the theoretical foundation of RKHS theory, the second active
learning theory and previous work using RKHS to do active learning for
regression, and finally we give an overview of Kernel Herding (hereafter
\textit{KH}) and its link to the Frank Wolfe (hereafter \textit{FW}) algorithm
for optimising the maximum mean discrepancy efficiently.

\section{Reproducing Kernel Hilbert Space Theory}
\label{sec:rkhs-theory}

\subsection{A brief history of the what and the why of Reproducing Kernel
Hilbert Spaces}
\label{sec:a-brief-hisory-RKHS} Kernels have a long history and the field was
arguably started with the work of \cite{aronszajn50_theor_reprod_kernel}. Since
then it has found use, to name a few, in geostatistics in the form of Kriging
\cite{cressie90_origin_krigin}, machine learning with support vector machines
\cite{evgeniou99_suppor} and gaussian processes \cite{williams96_gauss}.

A first question might be why we are interested in RKHS's when in many other
areas of mathematics and the sciences we consider \(L^{2}\) spaces. One of the
first obstacles is that the space of \(L^{2}(\X, \rho_{\X})\) is actually not a
space of functions, but equivalence classes of functions which agree except on a
set of measure zero with respect to \(\rho_{\X}\). In particular for specific
choices of kernel functions, for any compact subset \(\mathcal{Z}\) of the input
space \(\X\), the set of \(\overline{\textrm{span}(K_x, x \in \mathcal{Z})}\) is
dense in \(\cont{}(\mathcal{Z})\), the space of all continuous functions on
\(\mathcal{Z}\) with respect to the maximum norm, showing the power of RKHS's.

A second reason is that by choosing an RKHS as the hypothesis space turns the
regularised empirical risk minimisation objective, when using a convex loss,
into a nice convex program through the use of the representer theorem
\cite["Large" Reproducing Kernel Hilbert
Spaces]{gretton18_advan_topic_machin_learn}. This also allows us to turn
optimisation over a space of functions to a dual problem over a
finite-dimiensional vector space, reducing it to the domain of linear algebra
and giving us access to the tools therein.

However, there are drawbacks of using kernels for learning. It is well-known
that most kernel algorithms require inversion of the kernel matrix, including
kernel ridge regression, which scales cubically in the number of datapoints
\cite{saunders98_ridge_regres_learn_algor_dual_variab} making it prohibitively
expensive except for small datasets. Luckily this goes very well with active
learning since we are actually interested in making the size of the dataset
small while retaining close-to-optimal performance. This means that we are not
really giving up scalability as we are already making the aim to label as few
instances in the original dataset as possible. Hence RKHS is an ideal space to
work with as it gives us nice theoretical properties, a well-developed theory
and when used in active learning, few of the drawbacks usually encountered.

\subsection{Reproducing Kernel Hilbert Spaces} The field of RKHS theory is
well-established and there has been much written about it, we will follow the
course on RKHS theory taught at UCL 2018-2019
\cite{gretton18_advan_topic_machin_learn} together with introductions of
\cite{manton15_primer_reprod_kernel_hilber_spaces,fasshauer11_posit_defin_kernel}.

A Hilbert space is a tuple \((\Hc, \langle \cdot, \cdot \rangle)\) where \(\Hc\)
is a vector space and \(\langle \cdot, \cdot \rangle\) is an inner product
defined on some set \(\X\) such that this space is complete with respect to the
metric \(\|\cdot\|\) induced by the inner product. An RKHS is a specific type of
Hilbert space with the following property:

\begin{definition}
\label{def:RKHS} Let \(\X\) be a non-empty set and let \(\rkhs{H}\) be a Hilbert
space of functions \(f: \X \to \R\), then \(\rkhs{H}\) is said to be an RKHS if
for any \(x \in \X\) the evaluation functional \(\delta_x: f \mapsto f(x)\) is
continuous or equivalently bounded.
\end{definition}

The above definition is often too abstract in practice and we will show how to
equivalently construct RKHs's through the use of specific concepts of
\emph{kernels, reproducing kernel} and \emph{positive definite functions},
starting with the definition a kernel,
\begin{definition}
\label{def:kernel} Let \(\X\) be a non-empty set and \(K\) a function \(K: \X
\times \X \to \R\). Then \(K\) is called a \textit{kernel} on \(\X\) if there
exists some Hilbert space \(\rkhs{H}\) and a \textit{feature map} \(\phi : \X
\to \rkhs{H}\) such that for any \(x, x' \in \X\), \(K(x, x') =
\scal{\phi(x)}{\phi(x')}_{\rkhs{H}}\).
\end{definition}

Secondly the concept of a \emph{reproducing kernel},
\begin{definition}
\label{def:reproducing-kernel} Let \(\X\) be a non-empty set and let
\(\rkhs{H}\) be a Hilbert space of functions \(f: \X \to \R\). A function \(K:
\X \times \X \to \R\) is called a \textit{reproducing kernel} of \(\rkhs{H}\) if
it satisfies the following properties
\begin{align} \forall x \in \X, & \quad K_x = K(\cdot, x) \in
\rkhs{H}, \label{al:reproducer-in-rkhs}\\ \forall x \in \X, & \: \forall f \in
\rkhs{H}, \quad \scal{f}{K_x}_{\rkhs{H}} =
f(x), \label{al:reproducing-property-of-rkhs}
\end{align} that is, all functions \(K_x\) are in the space and they have the so
called \textit{reproducing property}.
\end{definition} In particular, we have that
\begin{corollary}
  \label{cor:reproducing-property-kxy} For any \(x, x' \in \X\)
  \begin{equation*} K(x, x') = \scal{K(\cdot, x)}{K(\cdot, x')}_{\rkhs{H}}.
  \end{equation*}
\end{corollary} In fact, every reproducing kernel is a kernel
\begin{corollary}
 \label{cor:reproducing-kernel-is-a-kernel} Every reproducing kernel is a kernel
with explicit feature map \(\phi : x \mapsto K(\cdot, x) \in \rkhs{H}\).
\end{corollary}

Finally, we have the following theorems which state that an RKHS is
characterised by its reproducing kernel function and it is unique:
\begin{theorem}
\label{th:rep-kernel-is-unique} If it exists, a reproducing kernel is unique.
\end{theorem}

\begin{theorem}
\label{th:rep-kernel-defines-rkhs} \(\rkhs{H}\) is an RKHS if and only if it has
a reproducing kernel.
\end{theorem}

A reproducing kernel is useful since it enables us to use machinery from linear
algebra to bound quantities. For example, we can use cauchy-schwartz (from here
on denoted \emph{CS}) to bound \(f(x)\) using \(\kappa = \sup_{x \in \X}
\sqrt{K(x, x)}\),
\begin{example}
\label{ex:cs-rkhs-bound-on-fx} When \(f \in \rkhs{H}\), an RKHS with reproducing
kernel \(K\), then we have the following
  \begin{align*} \abs{f(x)} & = \abs{\scal{f}{K(\cdot, x)}_{\rkhs{H}}} \\ & \leq
\norm{f}_{\rkhs{H}} \norm{K(\cdot, x)}_{\rkhs{H}} \\ & \leq \norm{f}_{\rkhs{H}}
\sqrt{K(x, x)} \\ & \leq \norm{f}_{\rkhs{H}} \sup_{x \in \X} \sqrt{K(x, x)} \\ &
= \norm{f}_{\rkhs{H}} \cdot \kappa
  \end{align*}
\end{example} where we have used the CS inequality since \(\rkhs{H}\) is a
Hilbert space. Thus we have
\begin{equation}
\label{eq:cs-rkhs-bound-on-fx} f(x) \leq \norm{f}_{\rkhs{H}} \cdot \kappa,
\end{equation} which shows that if a function is in an RKHS it is enough to
control the norm \(\norm{f}_{\rkhs{H}}\) in order to control \(f\) at any point
\(x \in \X\).

The final part we need is that of \emph{positive semi-definiteness},
\begin{definition}
  \label{def:pos-semi-definite-function} Let \(\X\) be a non-empty set then \(K
: \X \times \X \to \R\) is a \textit{positive semi-definite} function if for any
\(n \geq 1\), for any \((\alpha_i)_{i=1}^n \in \R^n\), for any \((x_i)_{i=1}^n
\in \X^n\),
\begin{equation*} \sum_{i, j}^n \alpha_i \alpha_j K(x_i, x_j) \geq 0,
\end{equation*} and we say that \(K\) is \textit{positive definite} if for
mutually distinct \(x_i\), the inequality is strict for any vector
\((\alpha_i)_{i=1}^n \neq \mat{0}\).
\end{definition}

Every inner product is a positive semi-definite function, which implies that
\begin{corollary}
\label{cor:kernels-are-positive-semi-definite} Every kernel \(K\) is a positive
semi-definite function.
\end{corollary}

At this point we have shown that for any reproducing kernel \(K\) of an RKHS
\(\rkhs{H}\), it is also a kernel and furthermore is positive semi-definite,
finally we also have that every positive semi-definite function \(K\) is the
reproducing kernel for some RKHS \(\rkhs{H}\),
\begin{theorem}[\cite{aronszajn50_theor_reprod_kernel}]
\label{thm:moore-aronszajn-pos-def-function-is-rep-kernel} Let \(\X\) be a
non-empty set and let \(K: \X \times \X \to \R\) be a positive semi-definite
function. Then there exists a unique RKHS \(\rkhs{H} \subset \{f : \X \to \R, f
\: \textrm{is measureable}\}\) such that \(K\) is the reproducing kernel of this
space.
\end{theorem}

Looking back, we see that using an RKHS as the space of functions under
consideration in machine learning is useful as it enables us to use the
\emph{kernel trick}: if an algorithm is defined only in terms of elementary
operations over inner products between data vectors, we can replace each inner
product \(\scal{x}{x'}\) with a kernel \(K(x, x')\) in order to work implicitly
in a (potentially) infinite dimensional space.

There are many different kernels that can be used and can be combined in various
ways according to rules we won't go into here, yielding a sort of calculus of
kernels. We will only consider the so called Gaussian kernel,
\begin{definition}
  \label{def:gaussian-kernel} Let \(\X \subseteq \R^d\) for some \(d > 0\), the
gaussian kernel \(K_{\sigma} : \X \times \X \to \R\) has the form of
  \begin{equation}
    \label{eq:gaussian-kernel} K_{\sigma}(x, x') = \exp(-\frac{\norm{x -
x'}_2^2}{2\sigma^2}),
  \end{equation} where \(\sigma > 0\).
\end{definition}

The Gaussian kernel is ubiqituos in machine learning and other fields which use
RKHS's, in particular it has the following properties
\begin{theorem}
  \label{th:gaussian-kernel-properties} For any \(\sigma > 0\), the Gaussian
kernel is
  \begin{itemize}
  \item Characteristic
  \item Universal
  \item Stationary
  \end{itemize} .
\end{theorem}

\subsection{Kernel Mean Embedding} Kernel Mean Embedding (hereafter \emph{KME})
generalises the reproducer for \(x \in \X\), \(K_{x}\), to that of a
distribution \(\rho \in \prob(\X)\). We will follow the review of
\cite{muandet17_kernel_mean_embed_distr} and the notes of
\cite{gretton18_advan_topic_machin_learn}.

\begin{definition}
  \label{def:kernel-mean-embedding} Let \(\X\) be a non-empty set and let
\(\prob(X)\) be the space of distributions on a measurable space \((\X,
\Sigma)\). The kernel mean embedding of a distribution \(\rho \in \prob(X)\)
into a RKHS \(\rkhs{H}\) endowed with a reproducing kernel \(K: \X \times \X \to
\R\) is defined by a mapping
  \begin{equation}
    \label{eq:kernel-mean-embedding} \mu : \prob(\X) \to \rkhs{H}, \quad
\mu(\rho) = \int_{\X} K_{x} \dd \rho(x).
  \end{equation}
\end{definition} We will use the shorthand \(\mu_{\rho} \coloneqq \mu(\rho)\).
Note that this generalises the definition of \(K_{x}\) since a dirac delta
distribution \(\delta_{x'}\) gives us \(\mu(\delta_{x'}) = \int_{\X}K_{x} \dd
\delta_{x'} = K_{x'}\). In this way \(\mu\) can be seen as the analogue to
\(\phi\) when working with distributions rather than points.

The following lemma gives necessary condition for the element to be in
\(\rkhs{H}\). We reproduce the proof as it exposes techniques we will be using
later.

\begin{theorem} Let \(\rkhs{H}\) be an RKHS with reproducing kernel \(K: \X
\times \X \to \R\). If \(\E_{\rho}[\sqrt{K(X, X)}] < \infty\) and \(f \in
\rkhs{H}\), then \(\mu_{\rho} \in \rkhs{H}\) and \(\E_{\rho}[f(X)] =
\scal{f}{\mu_{\rho}}_{\rkhs{H}}\).
\end{theorem}

\begin{proof} Let \(L_{\rho}\) be the linear operator defined as \(L_{\rho}(f) =
\E_{\rho}[f(X)]\). Then
  \begin{align*} \abs{ L_{\rho}(f) } & = \abs{\E_{\rho}[f(X)]} \\ & \leq
\E_{\rho}[\abs{f(X)}] \\ & = \E_{\rho}[\abs{\scal{f}{K_{X}}}] \\ & \leq
\E_{\rho}[\sqrt{K(X, X)} \norm{f}_{\rkhs{H}}]
  \end{align*} where we first apply Jensen's inequality then the CS inequlity.
By Riesz representation theorem, there exists an element \(\lambda_{\rho} \in
\rkhs{H}\) such that \(L_{\rho}(f) = \scal{\lambda_{\rho}}{f}_{\rkhs{H}}\).
Letting \(f = K_{x}\) for some \(x \in \X\), then \(\lambda_{\rho}(x) =
L_{\rho}(K_{x}) = \int_{\X} K(x, x') \dd \rho(x')\) which shows that
\(\lambda_{\rho} = \mu_{\rho}\).
\end{proof}

Trivially we have the following corollary
\begin{corollary} If \(\kappa = \sup_{x \in \X}\sqrt{K(X, X)} < +\infty\) then
for any \(\rho \in \prob(\X)\), \(\mu_{\rho} \in \rkhs{H}\).
\end{corollary}

\begin{proof} Since \(\E_{\rho}[\sqrt{K(X, X)} \norm{f}_{\rkhs{H}}] \leq \kappa
\norm{f}_{\rkhs{H}}\) we are done.
\end{proof}

We finish by considering the KME of the empirical distribution of a dataset. Let
\(S = (x_{i})_{i=1}^{n}\) be a sequence of \(n\) datapoints from some set
\(\X\), then the empirical distribution of \(S\), call this \(\rho_{S}\), is
defined to be
\begin{equation}
\label{eq:empirical-distribution} \rho_{S} = \frac{1}{n}\sum_{i=1}^{n}
\delta_{x_{i}}
\end{equation} and the corresponding mean embedding is \(\mu_{\rho_{S}} =
\int_{S} K_{x} \dd \rho_{S} = \frac{1}{n} \sum_{i=1}^{n}K_{x_{i}}\). If \(S\) is
the set of \(n\) samples sampled \(\iid\) from some distribution \(\rho\), then
we say
\begin{equation}
\label{eq:empirical-kernel-mean-embedding} \hat{\mu}_{\rho} \coloneqq
\mu_{\rho_{S}}
\end{equation} as it is an \emph{empirical estimate} of the mean embedding
\(\mu_{\rho}\).

\subsection{Maximum Mean Discrepancy} Integral Probability Metrics (hereafter
\emph{IPM}) \cite{mueller97_integ_probab_metric_their_gener_class_funct} are
distance-like functions defined in the following way
\begin{definition}
\label{def:IPM} Let \(\Fc\) be a space of real-valued bounded measurable
functions on a non-empty set \(\X\). Given two probability measures \(\rho,
\xi\) with support of \(\X\), then
  \begin{equation}
    \label{eq:IPM} \mathrm{IPM}_{\Fc}(\xi, \rho) = \sup_{f \in \Fc}\left(
\int_{\X}f(x) \dd \xi(x) - \int_{\X}f(x) \dd \rho(x) \right).
  \end{equation}
\end{definition}

Any space \(\Fc\) defines an IPM, we will only consider the case when \(\Fc\) is
a unit ball in an RKHS, in which case we call this metric the Maximum Mean
Discrepancy (hereafter \emph{MMD}). It's simple to show that in this case the
MMD takes the following simple form,
\begin{theorem}
\label{th:mmd-is-norm-between-feature-maps} Let \(B\) be the unit ball of an
RKHS with reproducing kernel \(K\), then we can express the MMD as
\begin{equation}
\label{eq:mmd-is-norm-between-feature-maps} \MMD{\xi}{\rho}{B} = \norm{\mu_{\xi}
- \mu_{\rho}}_{\rkhs{H}}.
\end{equation}
\end{theorem} This leads to the following expression of the MMD as the
expectation over kernels,
\begin{equation}
\label{eq:mmd-expectation-over-kernels} \MMD{\xi}{\rho}{B}^{2} = \E_{X, X' \sim
\xi}[K(X, X')] - 2 \E_{X \sim \xi, Y \sim \rho}[K(X, Y)] + \E_{Y, Y' \sim
\rho}[K(Y, Y')],
\end{equation} where the notation \(\E_{X, X' \sim \xi}[K(X, X')]\) indicates
that we draw two independent identically distributed copies form \(\xi\). When
the distributions are discrete, this turns into sums over kernel evaluations,
\begin{corollary}
\label{cor:discrete-dists-MMD-sum} When \(\xi = \sum_{i=1}^{n}\alpha_{i}
\delta_{x_{i}}\) and \(\rho = \sum_{j=1}^{m} \beta_{j} \delta_{y_{j}}\), where
we collect \((\alpha_{i})_{i=1}^{n}, (\beta_{j})_{j=1}^{m}\) into vectors
\(\mat{\alpha}, \mat{\beta}\), then we have
  \begin{align} \MMD{\xi}{\rho}{B}^{2} & = \sum_{i, i'}^{n} \alpha_i
\alpha_{i'}K(x_i, x_{i'}) - 2 \sum_{i, j}^{n, m} \alpha_i \beta_j K(x_i, y_j) +
\sum_{j, j'}^{n} \beta_j \beta_{j'}K(y_j, y_{j'}) \\ & = \mat{\alpha}^T
\mat{K}_{nn} \mat{\alpha} - 2 \mat{\alpha}^T \mat{K}_{nm} \mat{\beta} +
\mat{\beta}^T \mat{K}_{mm} \mat{\beta},
  \end{align} where \(\mat{K}_{nn} \in \R^{n \times n}, \mat{K}_{nm} \in \R^{n
\times m}, \mat{K}_{mm} \in \R^{m \times m}\) are matrices such that
\((\mat{K}_{nn})_{lt} = K(x_l, x_t), (\mat{K}_{nm})_{lt} = K(x_l, y_t),
(\mat{K}_{mm})_{lt} = K(y_l, y_t)\).
\end{corollary}

When the kernel is characteristic, the MMD is a true metric
\begin{theorem}
\label{th:when-characteristic-kernel-mmd-is-metric} When \(K\) is a
characteristic kernel, MMD is a metric on distributions in \(\prob(\X)\).
\end{theorem}

\section{Active Learning} We introduce the necessary concepts of Active Learning
by first introducing Supervised Learning (hereafter \emph{SL}) under the
Statistical Learning Theory (hereafter \emph{SLT}) model following the book
\cite{shalev-shwartz14_under} and the notes from the Learning Theory part of the
course in Advanced Topics in Machine Learning given at UCL 2019
\cite{ciliberto18_advan_topic_machin_learn}. After this we review active
learning as a field, looking at the historical developments and sub-divisions.
We will then focus on the so called \emph{pool-based active learning}, showing
how it relates to supervised learning and critically analyse the works using
RKHS theory. We lean on the comprehensive introduction of
\cite{settles12_activ_learn}.

\subsection{Statistical Learning Theory and Supervised Learning} SLT aims to
introduce a framework grounded on probability theory that answers such questions
as \emph{what does it mean for an algorithm to learn} and \emph{How do we design
good learning algorithms} \cite[Lecture
1]{ciliberto18_advan_topic_machin_learn}.

The SLT framework introduces the following concepts,

\begin{description}
\item[{Domain set}] An arbitrary set, \(\X\). The set of objects that we wish to
find the output for. We also refer to this set as the \emph{input space} and
call a point \(x \in \X\) an \emph{instance} or \emph{input}. We assume that
\(\X \subseteq \R^{D}\) for some \(D \in \N\).
\item[{Co-domain set}] An arbitrary set, \(\Y\). Our goal is to find a good
mapping from \(\X\) to \(\Y\) in the sense that we should be able to predict the
output in \(\Y\) for an arbitrary instance \(x \in \X\), hence why it's called
the \emph{output space} or \emph{target space}, occasionally we will call \(y\)
a \emph{label}. We will focus on regression, assuming that \(\Y \subseteq \R\).
\item[{Train set}] \(S = (x_{i}, y_{i})_{i=1}^{n}\) is a finite sequence of
pair, where each pair \((x_{i}, y_{i}) \in \X \times \Y\).
\item[{Hypothesis space}] A space of functions \(\mathcal{H} \subseteq \{h : \X
\to \Y, \: h \: \mathrm{measurable}\}\) meant to represent all of the possible
input-output rules that we consider. Note that we have used \(\mathcal{H}\) for
both an RKHS and hypothesis space, in practice this doesn't matter as we will
always use an RKHS as the hypothesis space.
\item[{Learning algorithm}] A function that takes as input a train set and maps
  to a hypothesis space,
  \begin{equation}
    \label{eq:sl-algorithm}
    \algo : \cup_{i=1}^{\infty} (\X \times \Y)^{n} \to \mathcal{H}.
  \end{equation}
\item[{Data generating distribution}] We assume there exists some distribution
\(\rho\) with support on \(\X \times \Y\) such that \(\rho(x, y) = \rho(y |
x)\rho_{\X}(x)\). The train set \(S = (x_{i}, y_{i})_{i=1}^{n}\) is sampled
\(\iid\) from \(\rho\), that is, each pair \((x_{i}, y_{i}) \in S\) is sampled
independently from \(\rho\) and we write \(S \sim \rho^{n}\). We do not have
access to \(\rho\) directly, but only through \(S\).
\item[{Measure of performance}] We choose a \emph{loss function}, \(\ell : \Y
\times \Y \to \R_{+}\) that quantifies the loss of predicting \(\hat{y}\) when
the output is \(y\) through \(\ell(y, y')\). The goal of SLT is then to minimize
the \emph{risk}, which is a function of an estimator \(h\),
\begin{equation}
\label{eq:def-risk} \risk{}{h} = \E_{\rho}[\ell(h(x), y)],
\end{equation} and we will be writing \(\err{\rho}{h}\) to indicate the risk
with respect to what measure.
\end{description}

The goal of SLT is to find an algorithm \(\algo\) that with high probability
over the sampled train set \(S\) is close in performance to the optimal
estimator in the hypothesis space. This is formalised by the concept of
\emph{excess risk},
\begin{definition}
  \label{def:excess-risk} Given the above, we define the excess risk to be the
quantity
  \begin{equation}
    \label{eq:excess-risk} \err{\rho}{h} - \err{\rho}{h^{\ast}},
  \end{equation} where \(h^{\ast} \in \argmin_{h' \in \rkhs{H}}\err{\rho}{h'}\).
\end{definition} The excess risk measures the additional risk that we take on by
using \(h\) rather than the optimal estimator in the hypothesis space. In
practice, the estimator \(h\) will be random since it will be the output of an
algorithm \(\algo\), which maps from a train set \(S\) to \(h\)\footnote{When
the function \(h\) is the output of an algorithm based on a train set \(S\) of
size \(n\) we denote \(\hat{h}_{n} \coloneqq \algo(S)\). If the size of \(S\) is
obvious we suppress the \(n\) and simply write \(\hat{h}\).}, making the excess
risk itself random due to the random sampling of \(S \sim \rho^{n}\).

In order to control the excess risk in probability, statements are made of the
following form
\begin{example}
  \label{ex:pac-like-excess-risk} With probability larger than \(1 - \delta\)
taken over the sampling of the train set \(S \sim \rho^n\), for any \(\delta \in
[0, 1]\), we have that the excess risk is bounded,
  \begin{equation*} \err{\rho}{\hat{h}_n} - \err{\rho}{h^{\ast}} \leq \epsilon
  \end{equation*}
\end{example} and these bounds are derived using error decomposition and
controlling for the error in probabiliy using assumptions on the hypothesis
class, data generating distribution etc. One famous example of this is Empirical
Risk Minimization (hereafter \emph{ERM})\cite{vapnik92_princ} where the
algorithm \(\algo\) is defined as the minimizer of the empirical risk, which in
our setting reduces to the following definition,
\begin{definition}
  \label{def:erm} Given an input and an output space \(\X, \Y\), an RKHS
\(\rkhs{H}\) acting as our hypothesis space, with loss \(\ell\) and a dataset
\(S = (x_i, y_i)_{i=1}^n\), Empirical Risk Minimization is an algorithm
  \begin{equation} \algo(S) = \argmin_{h \in \mathcal{H}} \frac{1}{n}
\sum_{i=1}^n \ell(h(x_i), y_i).
  \end{equation}
\end{definition} In this work, we will focus on Regularised Empirical Risk
Minimization (herafter \emph{RERM}) specialised to our setting,
\begin{definition}
  \label{def:rerm} Given input and output space \(\X, \Y\), an RKHS \(\rkhs{H}\)
acting as our hypothesis space, with loss \(\ell\) and a dataset \(S = (x_i,
y_i)_{i=1}^n\), Regularised Empirical Risk Minimization is an algorithm
  \begin{equation} \algo(S) = \argmin_{h \in \mathcal{H}} \frac{1}{n}
\sum_{i=1}^n \ell(h(x_i), y_i) + \lambda \norm{h}_{\rkhs{H}}^2.
  \end{equation}
\end{definition}

Kernel Ridge Regression (hereafter \emph{KRR}) is a generalisation of Ridge
Regression \cite{hoerl70_ridge_regres} where all of the inner products are
lifted to the feature space through the use of the kernel function. KRR can be
seen as the solution to the RERM when we have an RKHS \(\rkhs{H}\) and mean
squared error \(\ell(y, y') = (y - y')^{2}\) with \(\X \subseteq \R^{D}\) and
\(\Y \subseteq \R\), so that
\begin{equation}
  \label{eq:krr-rerm-form} \algo(S) = \argmin_{h \in \mathcal{H}}
\frac{1}{n}\sum_{i=1}^n (h(x_i) - y_i)^2 + \lambda \norm{h}_{\rkhs{H}}^2.
\end{equation}

It can be shown that KRR has the following solution
\begin{equation}
\label{eq:krr-equation} \algo(S)(x) = \mat{k}_{x}^{T} \mat{\alpha}_{\ast} =
\mat{k}_{x}^{T} (\mat{K}_{nn} + \lambda n \mat{I}_{n})^{-1}\mat{Y},
\end{equation} where
\begin{equation}
\label{eq:k-one-x} \mat{k}_{x} =
\begin{bmatrix} K(x, x_{1}) \\ \vdots \\ K(x, x_{n}), 
\end{bmatrix}
\end{equation} and
\begin{equation}
  \label{eq:y-mat} \mat{Y} =
  \begin{bmatrix} y_1 \\ \vdots \\ y_n, 
  \end{bmatrix}
\end{equation}
.

We judge the quality of an algorithm with respect to its \emph{error bound}. An
error bound gives us probabilistic guarantees on the magnitude of the excess
risk as defined in \ref{def:excess-risk},
\begin{definition}
  \label{def:error-bound-sl} A supervised learning error bound \(\epsilon(n,
\delta)\) is a function depending on an algorithm \(\algo\) that given a
\(\delta \in [0, 1]\) assures that the excess risk will be less than
\(\epsilon\) with probability greater than \(1 - \delta\). Or as a mathematical
expression,
  \begin{equation} \Pr_{S \sim \rho^n}(\err{}{\algo(S)} - \err{}{h^{\ast}} \leq
\epsilon(n, \delta)) \geq 1 - \delta.
  \end{equation}
\end{definition}

\subsection{Active Learning Theory Review} Active learning generalises the
setting of supervised learning by introducing an aspect of cost to acquire
labels of instances. According to Settles
\begin{quote} Active learning is any form of learning in which the learning
program has some control over the inputs on which it trains
\cite{settles12_activ_learn}.
\end{quote} In the SL setting we are \emph{given} a train set \(S\) consisting
of instance and output-pairs sampled from \(\rho\). Compare this to a setting
where we are only given a large set of \emph{instances} but attaining the
outputs of an instances is costly, it seems possible that we could attain an
acceptable level of accuracy or risk by just asking for a few outputs rather
than all of them. The latter example can be formalised as the setting of Active
Learning and was hinted at in the \hyperref[sec:background]{Background} section.

From literature there are proposed criteria for what a strategy for choosing
instances to label should consider and allows us to classify algorithms
according to how they implement each of these.
\cite{wu18_pool_based_sequen_activ_learn_regres} argue that a strategy should
take into account the \emph{informativeness}, \emph{representativeness} and
\emph{diversity} of the instances chosen. This corresponds well with how the
objective for choosing what instance to label often decompose into terms that
can be interpreted as encouraging informative instances but discouraging instances similar to instances already labeled (for
example \cite[Equation 4]{guo08_discr} and \cite[Equation
7]{chattopadhyay13_batch_mode_activ_sampl_based}).

It is possible to define an active learning algorithm by specifying an
\emph{active learning tuple} \((\algo, \querstrat)\) where \(\algo\) is the
algorithm of SL as defined in \ref{eq:sl-algorithm} and \(\querstrat\) is the strategy for how to choose instances
to query. The various setting of active learning then depends on what we need
\(\querstrat\) to do and according to what objective we want to have good
performance on. This leads to the subsettings of active learning.

Active learning can be split up into three subsettings, being \emph{Query
Synthesis}, \emph{Stream-Based Selective Sampling} and \emph{Pool-based Active
Learning} \cite{settles12_activ_learn}. These can be further divided into
sequential and batch-mode active learning where the first considers instances
one-by-one while the second instead presents the learner with sets of instances
\cite{guo08_discr}. The dominating stream of the two is that of sequential
learning due to the simplicity of deriving results but also due to the
combinatorial aspect of choosing subsets from a mother set leading to explosion
in computational complexity unless using approximations. Furthermore, these
approximations can sometimes be shown to be close-to-optimal.
\cite{krause08_near_optim_sensor_placem_gauss_proces} shows that in an active
learning setting where the goal is to optimally place sensors in predefined
locations and casting this as a problem of maximising the Mutual Information
between the sensor locations and the rest of the locations, there is a
polynomial-time greedy procedure that is within \((1 - 1/e)\) of optimum of the
NP-hard combinatorial objective problem. All this indicates that the
batch-setting is underexplored as the benefits are small over sequential active
learning.

All active learning theory from here on will be considered in the sequential
setting, and note that all batch-mode active learning can be rendered sequential
by making the set of size 1. Active learning then operates in a loop where \(t\)
is the time step at the start of the iteration, and the active learning
algorithm gets an input to the querying strategy \(\querstrat\) after which
\(\querstrat\) chooses what to do depending on the current setting, thereafter
the algorithm is free to predict or do whatever needed for \(\querstrat\) to
choose the next point at time \(t+1\). The reason for the flexibility is that in
general the instance chosen at \(t+1\) may depend on the output and predicted
output of the trained model up to time \(t\), that is the full history up until \(t\).

\subsection{Query synthesis} When the algorithm is allowed to synthesis
instances \emph{de novo} so that the instance chosen by \(\querstrat\) at time
\(t\) may be any point in \(\X\) we call this query synthesis, or learning with
membership queries \cite{angluin88_queries_concep_learn}. One prominent issue
with this is that we do not have access to \(\rho_{\X}\) at all and thus cannot
hope to minimise any risk unless this is supposed to be known a-priori.

\cite{cohn96_activ_learn_with_statis_model} does query synthesis by assuming
that \(\rho_{\X}\) is uniform to make the problem tractable, learning the
dynamics of an idealised robot arm. This can be a fruitful direction to take if
the actual goal is about learning a concept in itself rather than aiming for low
risk. However, there are failure cases when the instances are actually from some
underlying distribution \(\rho_{\X}\), which is expanded upon in
\cite{baum92_query} where they show that using humans to label synthesised
images of MNIST lead to strange artifacts as the algorithm does not take into
account the fact that the true marginal distribution on image only has support
for a small subspace of all possible images of digits. This makes it problematic
since we are trying to reduce the risk and need access to samples from
\(\rho_{\X}\) in order to learn about the marginal distribution.

\subsection{Stream-Based active learning} This is also called selective sampling
\cite{cohn94_improv_gener_with_activ_learn} due to the querying strategy
deciding at each time step whether to query an instance, which arrive in an
on-line fashion, or discard it and query a later instance. The assumption is
that the source of the instances produce these virtually for free so cost only
enters through querying. The example of learning a threshold concept in
\hyperref[sec:background]{Background} can be cast in this setting, where we reject any instance
shown to us to be less than the biggest instance currently labeled \(0\). In
this way, the querying strategy is a function \(\querstrat: \X \to \{0, 1\}\)
where \(0\) represents a decision to not label and \(1\) to label the current
instance streamed.

Approaches of when to query often corresponds to the criteria of an instance
being informative, representative and to add to the diversity of the already
queried instances so far. The most straightforward approach is to define a
measure of utility of information content and bias the sampling towards
instances with high information content. A second approach is that of using a
version space \cite{tong01_suppor_vector_machin_activ_learn}.

This setting contains many theoretically grounded algorithms, for example
\cite{dasgupta08}. They also point out some common problems with general active
learning algorithms; they rely upon computations which are not feasible in
practice or requires solving problems which are intractable in practice.

\subsection{Pool-based active learning} When given a set of datapoints \(S \sim
\rho^{n}\) we may assume that we know the instances and that the labels exist,
but are hidden from us until we query them for the corresponding outputs. When
the querying comes with a cost we want to only query as many instances as needed
to reach some level of performance before stopping. This is the pool-based
active learning setting \cite{lewis94} and it differs from query synthesis in
that the query strategy \(\querstrat\) depends on the dataset \(S\) of which it
only has access to the instances and labels from previously queried instances to
pick which instance to query next.

Pool-based active learning has a lot of overlap with the information retrieval
and data mining communities, due to the straightforward representation of
databases with instances such as documents as a pool set where labels need to be
produced by human experts
\cite{lewis94,tong01_suppor_vector_machin_activ_learn}. \cite{settles08} go into
how various utility based ways to choose what instance to label. Most of these
depend on the output of the current best hypothesis (or in the case of Query By
Committee, a set of hypotheses), which we will see is unnecessary in the setting
we will consider.

One limitation of pool-based active learning is that if the pool of samples is
big, the querying strategy theoretically need to calculate the utility of each
instance and then pick the instance which maximises the utility. In practice,
this is less of a concern as it's possible to approximate this scheme by for
example first subsample a smaller unlabeled dataset and label the most
informative instance in this. This poses a problem for us though, as unless we
can show that making this approximation leads to guarantees in probability, it's
not simple to see how this can be overcome.

We now define the necessary components for Pool-based active learning for
regression. For a set of input-output pairs \(\poolset = (x_{i},
y_{i})_{i=1}^{n}\) we denote the set of only instance as \(\poolsetx =
(x_{i})_{i=1}^{n}\). Pool-based active learning operates in a loop and we let
\(t\) denote the current time-step, incremented at the end of each loop. We have
the following

\begin{description}
\item[{Pool set}] A set of instance-output pairs sampled from \(\rho\) before
the start of the active learning loop, denoted by \(\poolset_{0} = (x_{i},
y_{i})_{i=1}^{n} \sim \rho^{n}\) and the corresponding set of only instances as
\(\poolsetx_{0}\). The pool set acts as the pool of available instances that
the algorithm can query at each \(t\), so that \(\poolsetx_{t}\) are the
instances not labeled at start of iteration \(t\). At time \(t\) the algorithm
only have access to \(\poolsetx_{t}\) and will ask for labels of an instance
from \(\poolsetx_{t}\).
\item[{Oracle}] The oracle is a function which takes as an input an instance and
samples from the conditional distribution \(\rho(Y | X = x)\). We will assume no
output noise in our derivation, so the conditional distribution is
deterministic. Formally we define \(\oracle: \X \to \Y\) where \(\oracle(x) =
f(x)\) where \(f(x)\) is the true relationship.
\item[{Querying strategy}] The strategy of how we decide what instance to label
at time \(t\) which we denote \(\querstrat\). We will make the assumption that
\(\querstrat\) chooses instances at \(t\) through a non-adaptive utility
function\footnote{This simply means that the querying strategy is independent of
the history of outputs from the queried instances.} \(\upsilon: \X \to \R\),
which implicitly depends on the instances already queried although we omit this
for notational ease, and that \(\querstrat\) chooses an instance \(x_{t} \in
\argmax_{x \in \poolsetx_{t}} \nu(x)\). We call the chosen instance \(x^{q}_{t}\) and send it
to the oracle to label giving us \(y^{q}_{t} = \oracle(x^{q}_{t})\).
\item[{Train set}] This is the analogue of the train set used in supervised
learning. We denote this by \(\trainset_{t}\) and the corresponding set of
instances as \(\trainsetx_{t}\). The reason for including \(t\) as a subscript
is that \(\trainset_{t}\) is not static but is grown by one input-output pair at
the end of each iteration.
\end{description}

In this way a pool-based active learning algorithm is defined by a tuple
\((\algo, \querstrat)\) where \(\algo\) has the same role as in supervised
learning, outputting a hypothesis \(\hat{h}_{t}\) depending on the current train
set \(S_{t}\), but which is now built depending on the querying strategy
\(\querstrat\).
\begin{algorithm}
  \caption{Pool based active learning}\label{alg:active-learning}
  \begin{algorithmic}[1] \Procedure{ActiveLearning}{$\poolset_{0}$, $\algo$,
$\querstrat$} \State $\trainset_0 \gets \emptyset$ \State $n \gets
|\poolset_{0}|$ \State $t \gets 1$ \While{$t \leq n$} \State $x^q_t \gets
\querstrat(\poolsetx_{t})$ \State $y^q_t \gets \oracle(x^q_t)$ \State
$\trainset_{t} \gets \trainset_{t-1} \cup \{(x^q_t, y^q_t)\}$ \State $\poolset_t
\gets \poolset_t \setminus \{(x^q_t, y^q_t)\}$ \State $t \gets t + 1$ \EndWhile
\EndProcedure
  \end{algorithmic}
\end{algorithm}

The performance of an active learning algorithm will be considered to be its
\emph{error bound} which is defined similarly as in \ref{def:error-bound-sl},
but we now also let this depend on the size of \(\trainset_{t}\) in addition to
the size original pool set
\begin{definition}
  \label{def:error-bound-al} An active learning error bound \(\epsilon(n, t,
\delta)\) is a function depending on an algorithm \(\algo\) and a querying
strategy \(\querstrat\) that given a \(\delta \in [0, 1]\) assures that the
excess risk will be less than \(\epsilon\) with probability greater than \(1 -
\delta\). Or as a mathematical expression, let \(\trainset_{t}\) be the train
set at the end of time \(t\),
  \begin{equation}
  \label{eq:error-bound-al} \Pr_{S \sim \rho^n}(\err{}{\algo(S_{t})} -
\err{}{h^{\ast}} \leq \epsilon(n, t, \delta)) \geq 1 - \delta.
  \end{equation}
\end{definition} This gives us a way to compare active learning algorithms to
each other, and note that this also gives us a way to compare supervised
learning to active learning by specifying \((\algo, \querstrat)\) and comparing
the active learning error bound \(\epsilon(n_{al}, t, \delta)\) to the
supervised learning bound \(\epsilon(n_{sl}, \delta)\) where \(n_{al} =
|\poolset_{0}|\) is the number of original samples in the pool and \(n_{sl} =
|\trainset|\) is the number of samples in the train set of supervised learning.

For active learning there has been several works based on generalisation bounds
\cite{ganti12_upal,xu19_towar_effic_evaluat_risk_via_herdin,gu12_towar} but
these are aimed at classification or graph-based prediction. For the setting of
regression, where a normal choice is to consider square loss and kernel ridge
regresion there is recently some promising results of
\cite{viering17_nuclear_discr_activ_learn} where they derive a bound of the form
\begin{equation}
\label{eq:viering-mmd-gen-bound} \err{\poolset_{0}}{h} \leq
\err{\trainset_{t}}{h} + \MMD{\poolsetx_{0}}{\trainsetx_{t}}{} + \eta_{MMD}
\end{equation} and propose to base the querying strategy of optimising the MMD
term. By making certain assumptions on the space for the MMD it can be shown
that \(\eta_{MMD} = 0\) and that the active learning bound decomposes into term
which may be interpreted as the empirical fit and the drift from the instance
set \(\trainsetx_{t}\) to the original set of instances \(\poolsetx_{0}\). This
makes it very clear that active learning in this setting is closely related to
domain adaptation and domain drift. \cite{cortes19_adapt_based_gener_discr} show
other measures in addition to MMD which may be used to control this domain drift
through upper bounding the empirical risk in similar ways, but MMD is an
appealing choice since it has an analytical form and analysing it comes down to
qualities of the kernel matrix.

From a domain adaptation perspective, active learning under no noise is
equivalent to the case of \emph{covariate shift}
\cite{gretton09_covar_shift_by_kernel_mean_match}.

\section{Herding and Conditional Gradient Methods in RKHS's}
\subsection{Kernel Herding}
Optimising the MMD term globally in the sense of
picking \(k\) instances out of \(\poolsetx_{0}\) can be shown to reduce to a
quadratic binary programming problem and is thus NP-hard in general
\cite{chaovalitwongse09_quadr_integ_progr}. In
\cite{chattopadhyay13_batch_mode_activ_sampl_based} they relax this problem in
two ways to choose instances, one through solving a convex quadratic programming
problem and one through solving a linear programming problem, both which can be
solved efficiently. While they consider classification, since the algorithm is
non-adaptive and based on controlling the same MMD term as in
\ref{eq:viering-mmd-gen-bound} it is directly comparable to what we are trying
to do. We will insted consider optimising this term through the use of
Frank-Wolfe \cite{frank56_algor_quadr_progr,jaggi13_revis_frank_wolfe} of which
herding can be shown to be a special case
\cite{bach12_equiv_between_herdin_condit_gradien_algor}.

Herding started as an approach to generate deterministic \emph{pseudo-samples}
from observed moments of a posited model, without needing to first find the
maximum likelihood estimates of the model parameters, but was limited in that
\(\X\) had to be finite \cite{welling09_herdin}. The Herding algorithm was later
generalised to arbitrary domains \(\X\) by
\cite{chen12_super_sampl_from_kernel_herdin} noting that we can let \(\mat{w}\),
the parameters of the model, be an element of a \textbf{finite-dimensional} RKHS
\(\rkhs{H}\) with \(\phi\) being the feature map of the corresponding
reproducing kernel \(K\). Given a probability distribution \(\rho_{\X}\) on
\(\X\), kernel herding is the following
\begin{algorithm}
  \caption{KernelHerding}\label{alg:kernel-herding}
  \begin{algorithmic}[1] \Procedure{KernelHerding}{$\rkhs{H}$, $\rho_{\X}$,
$\mat{w}_{0} \in \rkhs{H}$} \State $\mu_{\rho_{\X}} = \E_{X \sim
\rho_{\X}}[\phi(X)]$ \State $t \gets 1$ \While{$t < \infty$} \State $x_{t} \gets
\argmax_{x \in \X} \scal{\mat{w}_{t-1}}{\phi(x)}_{\rkhs{H}}$ \State $\mat{w}_{t}
\gets \mat{w}_{t-1} + \hat{\mu} - \phi(x_{t})$ \State $t \gets t + 1$ \EndWhile
\EndProcedure
  \end{algorithmic}
\end{algorithm}

While this is very similar to the original herding algorithm, they also show
that we can view the herding algorithm as a way of sequentially minimizing the
squared error
\begin{equation}
\label{eq:squared-error-herding} \norm{\mu_{\rho_{\X}} - \frac{1}{t}
\sum_{j=1}^{t} \phi(x_{j})}_{\rkhs{H}}^{2}.
\end{equation} If we look at the form of this, we can see that by
\ref{eq:mmd-is-norm-between-feature-maps} this is the same as
\begin{equation} \MMD{\rho_{\X}}{\hat{\rho}_{t}}{\rkhs{H}}^{2},
\end{equation} where \(\hat{\rho}_{t} = \frac{1}{t} \sum_{j=1}^{t}
\delta_{x_{j}}\). Furthermore, while Monte-Carlo sampling of \(\hat{\rho}_{t}\), that
is sampling \(\iid\) from \(\rho_{\X}\), decreases the squared MMD as
\(O(\frac{1}{t})\), KH improves upon this with \(O(\frac{1}{t^{2}})\) which can
be explained by the KH algorithm using \emph{negative} auto-correlations to
explore parts of the space where samples haven't been drawn from before, linking
it to Quasi Monte Carlo methods and Low Discrepancy Sequences
\cite{chen12_super_sampl_from_kernel_herdin,bach12_equiv_between_herdin_condit_gradien_algor}.
They state the following which relates how herding and KH can be used in active
learning,
\begin{corollary}[\citep{chen12_super_sampl_from_kernel_herdin}, Corollary 6] An
active learning algorithm selecting labels in accordance with the herding
algorithm has guaranteed rate of convergence in terms of \(O(\frac{1}{t})\).
Moreover, the submodular greedy algorithm of
\cite{krause08_near_optim_sensor_placem_gauss_proces} has therefor also at least
the same approximation rate since it is within a constant fraction \((1 -
e^{-1})\) of optimality.
\end{corollary}
which is in essence what we will be doing in this dissertation by building
\(S_t\) using KH and FW, but trying to extend
it to the generalisation error instead of considering the empirical excess risk.

\subsection{Frank-Wolfe} Frank Wolfe
\cite{frank56_algor_quadr_progr} is an algorithm for solving general constrained
convex optimization problems of the form
\begin{equation}
\label{eq:constrainted-convex-opt-problem} \min_{z \in \mathcal{C}} J(z),
\end{equation} where \(\mathcal{C}\) is some compact convex subset of a Hilbert
space \(\mathcal{Z}\) and \(J\) is assumed to be convex over \(\mathcal{C}\) and
continuously differentiable. FW sets itself apart as it only requires access to
the gradient \(\nabla J\) on \(\mathcal{C}\) and be able to find the minimiser
of quantities of the form \(\scal{s}{\nabla J(z)}_{\mathcal{Z}}\), and does not
require any projection steps while converging to \(J(z^{\ast}) = \min_{z
\in \mathcal{C}} J(z)\) as \(O(\frac{1}{t})\) given certain regularity
conditions \cite{jaggi13_revis_frank_wolfe}.

\begin{algorithm}
  \caption{FrankWolfe}\label{alg:frank-wolfe}
  \begin{algorithmic}[1] \Procedure{FrankWolfe}{$\mathcal{Z}$, $\mathcal{C}$, $J$,
$\rho_{t}$} \State $t \gets 1$ \While{$t \leq T$} \State $\tilde{g}_{t} \gets
\argmin_{z \in \mathcal{Z}} \scal{z}{\nabla J(x_{t-1})}_{\X}$ \State $g_{t}
\gets (1 - \rho_{t-1})g_{t-1} + \rho_{t-1}\tilde{g}_t$ \State $t \gets t + 1$
\EndWhile \EndProcedure
  \end{algorithmic}
\end{algorithm}

By rewriting the Kernel Herding algorithm,
\cite{bach12_equiv_between_herdin_condit_gradien_algor} showed that KH is a
subset of a family of Frank-Wolfe algorithms over an RKHS \(\rkhs{H}\). This is
the setting we will work in and we describe the assumptions they use which will
also be ours.

\begin{assumption}
\label{as:fw-kernel-herding} We have a set \(\X\) and a mapping \(\phi: \X \to
\rkhs{H}\), where \(\rkhs{H}\) is an RKHS with reproducing kernel \(K\). We
assume that the data is uniformly bounded in the feature space, which means that
for any \(x \in \X\), \(\norm{\phi(x)}_{\rkhs{H}} \leq R\) for some \(R > 0\).
\end{assumption}

We denote \(\mathcal{M}\) to be the marginal polytope, which is the
\hyperref[def:conv-hull]{convex hull} of all vectors \(\phi(x)\) for \(x \in X\)
a finite set \(X \subseteq \X\). For any \(f \in \rkhs{H}\) the following holds
\begin{fact}
\label{fact:optimal-element-is-extremum} The supremum of \(f(x)\) over
\(\mathcal{C}\) is attained by an extremum,
\begin{equation}
\label{eq:optimal-element-is-extremum} \sup_{x \in \X} f(x) = \sup_{g \in
\mathcal{M}} \scal{f}{g}_{\rkhs{H}}.
\end{equation}
\end{fact} the authors then show equivalence between KH and FW by considering
the optimization problem where we equate \(\rkhs{H}\) with \(\mathcal{Z}\) and
\(\mathcal{M}\) with \(\mathcal{C}\)
\begin{equation}
\label{eq:fw-kh-optimisation-problem} \min_{g \in \mathcal{M}} J(g) =
\min_{g \in \mathcal{M}}\frac{1}{2}\norm{g - \mu_{\rho_{X}}}_{\rkhs{H}}^{2}
\end{equation} where \(\mu_{\rho_X}\) is the mean embedding of \(\rho_X\)
into \(\rkhs{H}\), leading to the FW update equations
\begin{align} \tilde{g}_{t} & = \argmin_{g \in \mathcal{M}} \scal{g_{t-1} -
\mu_{X}}{g}_{\rkhs{H}} \label{al:fw-update-step-1}\\ g_{t} & = (1 -
\rho_{t-1}) g_{t-1} + \rho_{t-1} \tilde{g}_{t} \label{al:fw-update-step-2}
\end{align} and using \(\rho_{t} = \frac{1}{t+1}\) we recover the KH algorithm.
Note that for each \(t\) in \ref{al:fw-update-step-1} the chosen
\(\tilde{g}_{t}\) is assured to lie on the extremum according to
\ref{fact:optimal-element-is-extremum}. This shows that as \(X\) is a finite
set of \(n\) points, running FW produces a sequence \((x_{t})_{t=1}^{n}\) where
\(x_{t}\) is associated with the corner of \(\mathcal{M}\) such that
\(\tilde{g}_{t} = \phi(x_{t})\) which means that when \(X = \poolsetx_{0}\) we
can use the FW algorithm to choose points. In the utility framework, we let
\(\upsilon\) be the function
\begin{equation}
\label{eq:fw-utility-function} \upsilon(x) = -\scal{g_{t-1} -
\mu_{\rho_{\poolsetx_{0}}}}{\phi(x)}_{\rkhs{H}},
\end{equation} and using this we can create an algorithm by letting
\(\querstrat\) be the querying strategy that has utility function
\ref{eq:fw-utility-function} of \ref{alg:active-learning}. We call this family
of algorithms Frank-Wolfe Active Learning (FWAL), where we indicate the
algorithm by specifying \(\rho_{t}\).

\subsection{Convergence rates for Frank-Wolfe}
\label{sec:org9640d5f} While KH had a convergence rate of \(O(\frac{1}{t})\), FW
has different rates depending on the step-size chosen at each \(t\). In fact, FW
can have much faster convergence if we use line-search as opposed to a fixed
step-size. Let the constant
\begin{equation}
\label{eq:d-radius-of-biggest-interior-ball} d = \argmin_{g \in \relbound
\mathcal{M}} \norm{g - \mu_{\rho_{\X}}}_{\rkhs{H}}
\end{equation} where \(\relbound \mathcal{M}\) is the
\hyperref[def:relative-boundary]{relative boundary} of \(\mathcal{M}\). Note
that this depends on \(\mathcal{M}\) and we can have arbitrarily small \(d\).
The following table denotes upper bounds on the rate of convergence rates of
\(J(g_{t}) = \frac{1}{2} \norm{g_{t} - \mu_{\rho_{\X}}}^{2}_{\rkhs{H}}\)
using FW (any), FW-kh (\(\rho_{t} =
\frac{1}{t+1})\), FW-line search (hereafter \emph{FW-ls}), where \(g_{t}\) is
the output from the algorithms after \(t\) steps and sufficient (but not
necessary) conditions for this to hold is laid out in the following table

\begin{table}[htbp]
\caption{\label{tbl:FW-convergence-table} Upper bound on convergence rates using
Frank Wolfe} \centering
\begin{tabular}{lll} \hline Algorithm & Convergence (upper bound) & Additional
Condition to \ref{as:fw-kernel-herding}\\ \hline FW (any) & \(4\frac{R^{2}}{t}\)
& None\\ FW-kh & \(\frac{2R^{4}}{d^{2}t^{2}}\) & \(d > 0\)\\ FW-ls & \(R^{2}
\exp(-\frac{d^{2}t}{R^{2}})\) & \(d > 0\)\\ \hline
\end{tabular}
\end{table}

Finally, \cite{bach12_equiv_between_herdin_condit_gradien_algor} prove the following two propositions
\begin{proposition}
\label{prop:finite-dim-fw-has-speedup} Assume that \(\rkhs{H}\) is
finite-dimensional, that \(\X\) is a compact topological measure space with a
continuous kernel function \(K\) and that the distribution \(\rho_{\X}\) has
full support on \(\X\). Then \(d > 0\).
\end{proposition}

\begin{proposition}
\label{prop:infinite-dim-fw-has-probably-not-speedup} Assume that \(\rkhs{H}\)
is infinite-dimensional, that \(\X\) is a compact subspace of \(\R^D\) with a
continuous kernel function \(K\) on \(\X \times \X\). Then \(d = 0\).
\end{proposition}


\chapter{Active Learning using Frank Wolfe}
\label{ch:methodology}
In this chapter we propose an active learning algorithm based on the bound of
\citep{viering17_nuclear_discr_activ_learn} and use FW to optimise this bound. In
particular we use the tools laid out in the \hyperref[ch:lit-rev]{Literature Review}
in order to attempt to derive bounds on the excess active learning risk.
We show that FW gives us guarantees of reducing the empiricial risk as a
function of the size of the active learning dataset, but that it is inconclusive
given current results on rate of convergence if it improves on the
generalisation error compared to random sampling.

\section{Setting}
The setting largely follows that of \citep{viering17_nuclear_discr_activ_learn} and
\citep{bach12_equiv_between_herdin_condit_gradien_algor} as we fuse these two works and
lean on results of both. We first specify all of our assumptions for the bound.

\begin{assumption}
  \label{as:al-fw}
  We assume that \(\X \subseteq \R^{D}\) for some \(D \in \N\) and \(\Y \subseteq
  \R\). We assume there exists a deterministic labeling function \(f : \X \to
  \Y\) and that there is some unknown distribution \(\rho\) on \(\X \times \Y\).
  Furthermore, we assume that there is no output noise such that \(\rho(x, y) = \rho_{\X}(x)\mathbf{1}(f(x) =
  y)\). At the start we get a pool of \(n\) \(\iid\) samples from \(\rho\), which we
  call \(\poolset_{0} = (x_{i}, y_{i})_{i=1}^{n}\), and we start with an empty train
  set \(\trainset_{0}\).
\end{assumption}
We furthermore make the following assumptions
\begin{assumption}
  \label{as:domain-is-compact}
  We assume that \(\X\) is compact.
\end{assumption}
\begin{assumption}
  We assume that the output it bounded in magnitude,
  \label{as:output-bounded}
  \begin{equation*}
    \sup_{y \in \mathcal{Y}}|y| = M_y < \infty.
  \end{equation*}
\end{assumption}
\begin{assumption}
  \label{as:f-in-gamma-ball}
  The function \(f \in B_{\gamma}\) where \(B_{\gamma} \coloneqq B_{\gamma}(0) \subseteq \rkhs{H}\) and \(\rkhs{H}\) is an RKHS with kernel function \(K\).
\end{assumption}
and sometimes we will refer to the realisable setting
\begin{assumption}
  \label{as:realisable-setting}
  We are in realisable setting, our algorithm \(\algo\) is such that \(f \in
  \ran(\algo)\), the hypothesis space.
\end{assumption}

We will use the squared loss \(\ell(y, y') = (y - y')^{2}\). We will consider
the hypothesis space to be a ball in an RKHS \(\rkhs{H}\) with reproducing
kernel \(K\), where the radius is \(\gamma'\). If \(\gamma' \geq \gamma\) so that \(f\) is in this
hypothesis space then we are in the realisable setting. We introduce an auxiliary RKHS \(\rkhs{H}'\) which is such
that the reproducing kernel \(K'\) is \(K'(x, x') = K(x, x')^{2}\), which will
be associated with the functional space used for MMD calculations and in
extension KH and FW.

The functional space we will use for MMD will be \(B_{\eta} \subseteq
\rkhs{H}'\) for some \(\eta > 0\), and from \ref{as:domain-is-compact}, if we
assume that \(K, K'\) are continuous on \(\X \times \X\) we have that the norm
of the feature maps, \(\phi(x)\) and \(\phi'(x)\) for \(\rkhs{H}\) and
\(\rkhs{H}'\) respectively, are bounded and we denote \(\kappa = \sup_{x \in
\mathcal{X}}\sqrt{K(x, x)} = \sup_{x \in \mathcal{X}}\norm{\phi(x)}_{\rkhs{H}} <
\infty\), and \(\kappa' = \sup_{x \in \mathcal{X}}\sqrt{K'(x, x)} < \infty\).

\section{MMD Empirical Generalisation Bound}
We now state results from \citep{viering17_nuclear_discr_activ_learn} that will be
used.
\begin{theorem}[\citep{viering17_nuclear_discr_activ_learn}, Extension: Theorem 1]
  \label{th:mmd-emp-bound}
  Let \(\ell\) be any loss, and let \(g_{f, h}(x) = \ell(f(x), h(x))\), then for
  any function \(h \in H \subseteq \rkhs{H}\), \(H\) being an arbitrary subspace of \(\rkhs{H}\), any train set
  \(\trainset \subseteq \poolset_{0}\) and let \(\rho_{\trainsetx}\) be any distribution with
  support on \(\trainsetx\) with \(\rho_{S}\) the corresponding distribution on
  \((x, f(x))_{x \in \trainset}\), then for any \(H' \subseteq \rkhs{H}'\)
  \begin{equation*}
    \label{eq:MMD-gen-bound}
    \err{\poolset_{0}}{h} \leq \err{\rho_{\trainset}}{h} + \MMD{\poolsetx_{0}}{\rho_{\trainsetx}}{H'} + \eta_{MMD}
  \end{equation*}
\end{theorem}
where \(\eta_{MMD} = 2 \min_{\tilde{g} \in H'} \max_{h \in H, x \in \poolsetx_{0}} |g_{f, h}(x) - \tilde{g}(x)|\).

\begin{proof}
  This is essentially \citep[Proof of Theorem 1]{viering17_nuclear_discr_activ_learn}.
  Let \(g_{f, h}(x) = \ell(f(x), h(x))\) and let \(\tilde{g}\) be any function in
  \(H'\).

  Consider bounding the following quantity
  \begin{equation*}
    \abs{\err{\poolset_0}{h} - \err{\rho_{\trainset}}{h}}.
  \end{equation*}
  We decompose this as follows, letting \(w_{j} = \rho_{\trainsetx}(x_j)\) where
  \(x_j \in \trainsetx\) and \(n = |\poolset_0|, m = |\trainset|\),
  \begin{align*}
    \abs{\err{\poolset_0}{h} - \err{\rho_{\trainset}}{h}} = & \abs{\err{\poolset_0}{h} - \frac{1}{n} \sum_{i=1}^n\tilde{g}(x_i) + \frac{1}{n} \sum_{i=1}^n\tilde{g}(x_i) - \sum_{j=1}^m w_j \tilde{g}(x_j) + \sum_{j=1}^m w_j \tilde{g}(x_j) - \err{\rho_{\trainset}}{h}} \\
    \leq & \underbrace{\abs{\err{\poolset_0}{h} - \frac{1}{n} \sum_{i=1}^n\tilde{g}(x_i)}}_{(a)} + \underbrace{\abs{\frac{1}{n} \sum_{i=1}^n\tilde{g}(x_i) - \sum_{j=1}^m w_j \tilde{g}(x_j)}}_{(b)} + \underbrace{\abs{\err{\rho_{\trainset}}{h} - \sum_{j=1}^m w_j \tilde{g}(x_j)}}_{(c)}
  \end{align*}

  For \((a)\) we can bound this as
  \begin{equation*}
    \abs{\err{\poolset_0}{h} + \frac{1}{n} \sum_{i=1}^n\tilde{g}(x_i)} \leq \frac{1}{n} \sum_{i=1}^{n}\abs{g_{f, h}(x_{i}) - \tilde{g}(x_{i})} \leq \max_{x \in \poolsetx_0}\abs{g_{f, h}(x) - \tilde{g}(x)},
  \end{equation*}
  and for \((c)\) we can similarly bound this as
  \begin{align*}
    \abs{\err{\rho_{\trainset}}{h} - \sum_{j=1}^m w_j \tilde{g}(x_j)} & \leq \sum_{j=1}^{m} w_{j}\abs{g_{f, h}(x_{j}) - \tilde{g}(x_{j})} \leq \sum_{j=1}^{m}w_{j}\max_{x \in \trainsetx}\abs{g_{f, h}(x) - \tilde{g}(x)} \\
                                                                      & = \max_{x \in \trainsetx}\abs{g_{f, h}(x) - \tilde{g}(x)} \leq \max_{x \in \poolsetx_0}\abs{g_{f, h}(x) - \tilde{g}(x)},
  \end{align*}
  which follows from the fact that \(\trainsetx \subseteq \poolsetx_{0}\).

  For \((b)\) we can bound this as follows
  \begin{equation*}
    \abs{\frac{1}{n} \sum_{i=1}^n\tilde{g}(x_i) - \sum_{j=1}^m w_j \tilde{g}(x_j)} \leq \sup_{\tilde{g} \in H'}\abs{\frac{1}{n} \sum_{i=1}^n \tilde{g}(x_i) - \sum_{j=1}^m w_j \tilde{g}(x_j)} = \MMD{\poolsetx_0}{\rho_{\trainsetx}}{H'},
  \end{equation*}
  and altogether we have that
  \begin{align*}
    \abs{\err{\poolset_0}{h} - \err{\rho_{\trainset}}{h}} & \leq \MMD{\poolsetx_0}{\rho_{\trainsetx}}{H'} + 2\max_{x \in \poolsetx_0}\abs{g_{f, h}(x) - \tilde{g}(x)} \\
                                                          & \leq \MMD{\poolsetx_0}{\rho_{\trainsetx}}{H'} + 2\min_{\tilde{g} \in H'}\max_{h \in H, x \in \poolsetx_0}\abs{g_{f, h}(x) - \tilde{g}(x)}.
  \end{align*}

  Setting \(\rho_{\trainsetx} = \frac{1}{m}\sum_{j=1}^{m}\delta_{x_{j}}\) recovers the original proof.
\end{proof}

We note that \ref{th:mmd-emp-bound} works for empirical distributions over
\(\trainsetx\), but also reweighted distributions. This is useful as it allows
us to use FW algorithms that outputs a distribution over \(\trainsetx_{t}\) that
is non-uniform which includes any FW algorithm where \(\tau_t \neq
\frac{1}{1+t}\) which in particular includes the line-search version.

We have the following result
\begin{theorem}[\citep{viering17_nuclear_discr_activ_learn}, Theorem 2]
  \label{thm:zero_eta_MMD}
  Let \(\ell\) be the squared loss and assume \(f, h \in B_{\gamma}\). Let \(K, K'\)
  be the kernel functions of the RKHS's \(\rkhs{H}, \rkhs{H}'\) respectively. If
  \(K'(x, x') = K(x, x')^2\) and \(\eta \geq 4\gamma^2\) so that \(H =
  B_{\gamma}, H' = B_{\eta}\), then it is guaranteed
  that \(g(\cdot) = \ell(f(\cdot), h(\cdot)) \in B_{\eta}\) and thus \(\eta_{MMD}
  = 0\).
\end{theorem}
When using a Gaussian kernel, this reduces to the following corollary
\begin{corollary}
  \label{cor:gauss-kernel-squared-gives-eta-zero}
  Let \(f, h \in B_{\gamma}\) and let \(K(x, x') = \exp(- \frac{\|x - x'\|^2}{2\sigma^2})\), the gaussian kernel with
  bandwidth \(\sigma\). If \(K'(x, x') = \exp(- \frac{\|x - x'\|^2}{2\sigma'^2})\) with bandwidth \(\sigma' =
  \frac{\sigma}{\sqrt{2}}\) and \(\eta = 4\gamma^2\) then \(\eta_{MMD} = 0\).
\end{corollary}

\begin{proof}
  A sufficient condition given by \ref{thm:zero_eta_MMD} is that if \(K'(x, x') =
  K(x, x')^2\) and \(\eta = 4\gamma^2\) then \(\eta_{MMD} = 0\). As we are using
  gaussian kernels, this means that
  \begin{equation*}
    \exp(- \frac{\|x - x'\|^2}{2\sigma'^2}) = \exp(- \frac{\|x - x'\|^2}{2\sigma^2})^2 = \exp(- \frac{\|x - x'\|^2}{\sigma^2})
  \end{equation*}
  which means that \(\sigma' = \frac{\sigma}{\sqrt{2}}\).
\end{proof}

For \(\eta_{MMD}\) to be zero when using KRR we need the output of our algorithm to be
contained in the ball \(B_{\gamma}\), with the risk of falling outside the
realizable setting. While KRR and Ivanov Regression is linked
as KRR solves a lagrangian formulation of the Ivanov Regression problem, we need
to make sure that the output \(\hat{h} = \algo(S)\) when doing KRR is such that
\(\norm{\hat{h}}_{\rkhs{H}} \leq \gamma\). Following the proof of \citep[Lemma
1]{cortes14_domain_adapt_sampl_bias_correc} we have the following
\begin{lemma}
\label{lem:KRR-output-ball-bound} The output of the KRR algorithm with
regularisation parameter \(\lambda\), when applied
to \(S = \{(x_{i}, y_{i})\}_{i=1}^{n}\) is such that for \(\hat{h} = \algo(S)\) we have
\(\norm{\hat{h}}_{\rkhs{H}} \leq \frac{2 M_{y}}{\gamma}\) and if \(\lambda =
\frac{2 M_y \kappa}{\gamma}\) then the hypothesis space is contained within
\(B_{\gamma}\).
\end{lemma}
\begin{proof} Since \(\hat{h} = \algo(S)\) and this is the minimiser of the KRR
objective \(\err{S}{h} + \lambda \norm{h}^2_{\rkhs{H}}\), we see that
  \begin{equation*} \err{S}{\hat{h}} + \lambda \norm{\hat{h}}_{\rkhs{H}}^2 \leq
\err{S}{0}
  \end{equation*} and we can then rearrange this by
  \begin{align*} \lambda \norm{\hat{h}}_{\rkhs{H}}^2 & = \err{S}{0} -
\err{S}{\hat{h}} \\ & = \frac{1}{n} \sum_{i=1}^n (y_i^2 - (\hat{h}(x_i) -
y_i)^2) \\ & = \frac{1}{n} \sum_{i=1}^n (2 y_i \hat{h}(x_i) - \hat{h}(x_i)^2) \\
& \leq \frac{2}{n} \sum_{i=1}^n y_i \hat{h}(x_i) \\ & \leq \frac{2}{n}
\sum_{i=1}^n M_y \norm{\hat{h}}_{\rkhs{H}} \kappa \\ & = 2 M_y \kappa
\norm{\hat{h}}_{\rkhs{H}}.
  \end{align*}

  Assume that \(\hat{h} \neq 0\), otherwise this is an equality, then we can
cancel out the norm showing that
  \begin{equation*}
    \label{eq:KRR-output-ball-bound} \norm{\hat{h}}_{\rkhs{H}} \leq \frac{2 M_y
\kappa}{\lambda}.
  \end{equation*} Setting \(\gamma = \frac{2 M_y \kappa}{\lambda}\), we see that
if \(\lambda = \frac{2 M_y \kappa}{\gamma}\) then the output of KRR will be in
\(B_{\gamma}\).
\end{proof}

Lemma \ref{lem:KRR-output-ball-bound} assures us that if we pick \(\lambda\)
large enough, then the output will be contained within the ball \(B_{\gamma}\)
so that \ref{cor:gauss-kernel-squared-gives-eta-zero} holds and we can assume
\(\eta_{MMD} = 0\) but that also now \(B_{\gamma'} \subseteq B_{\gamma}\).

\section{Frank Wolfe Active Learning algorithm}
Considering the bound of the empirical risk over the sampled set
\(\poolset_{0}\) in \ref{th:mmd-emp-bound} we see that there are three terms to
control, the bias term \(\err{\rho_{\trainset_{t}}}{h}\), the term representing
domain drift, \(\MMD{\poolsetx_{0}}{\rho_{\trainsetx_{t}}}{H'}\), and the leftover term
\(\eta_{MMD}\) which becomes \(0\) according to
\ref{cor:gauss-kernel-squared-gives-eta-zero} by setting \(H' =
B_{4\gamma^{2}}\) when \(H = B_{\gamma}\) and \(\sigma' = \frac{\sigma}{\sqrt{2}}\) where we use gaussian kernels \(K, K'\).

We focus on the domain drift term and propose to optimise this using FW. Outside
of the degenerate case this enables us to do active learning in an online
fashion by running the FW algorithm on
\(\MMD{\poolsetx_{0}}{\rho_{\trainsetx_{t}}}{H'}\) which will give a convergence
of at least \(O(\frac{1}{t})\) compared to \(O_P(\frac{1}{\sqrt{t}})\) if we were
to sample points randomly.

We now state our first active learning algorithm which uses the active learning
tuple \((\algo, \querstrat)\) where \(\algo\) is KRR \ref{eq:krr-equation} with regularisation
parameter \(\lambda\) and \(\querstrat\) is the querying strategy that runs FW on
\(\poolsetx_{0}\) and chooses the point \(x_{t}\) corresponding to
\(\tilde{g}_{t}\) in \ref{al:fw-update-step-1}. The algorithm follows the recipy of
\ref{alg:active-learning} with algorithm \(\algo\) being KRR.

Note that in general we can create ad-hoc active learning algorithms by letting
\(\querstrat\) be FW with any \(\tau_t\) scheme and \(\algo\) any algorithm. In
this way we can create a family of FW-based active learning algorithms. We
introduce the following active learning algorithm for classification, let
\(\querstrat\) be FW and let \(\algo\) be the KRR applied to classification
through application of the method expanded upon in \citep{ciliberto16}. We use
this to empirically investigate FW for classification in the
\hyperref[ch:experiments]{Experiments} section.

\section{Attempting to Bound the Excess Risk}
With inspiration of error decomposition of the excess risk in
supervised learning we try to bound the excess risk
\err{\rho}{\hat{h}} - \err{\rho}{f}
where as before \(\hat{h} = \algo(\trainset)\). We work with
Ivanov Regression instead of Kernel Ridge Regression, where the algorithm is
the following
\begin{equation*}
  \label{eq:eq:ivanov-kr}
  \algo(\trainset) = \argmin_{h \in B_{\gamma} \subseteq \rkhs{H}} \err{S}{h} = \argmin_{h \in B_{\gamma}} \frac{1}{m}  \sum_{j=1}^{m}(h(x_{j}) - y_{j})^{2},
\end{equation*}
as this can easily be extended to the KRR case by decomposing the generalisation
error further.

\subsection{Error Decomposition}
\label{sec:error-decomposition}
We decompose the excess risk as follows
\begin{align*}
  \err{\rho}{\hat{h}} - \err{\rho}{f} & = \underbrace{\err{\rho}{\hat{h}} - \err{\trainset}{\hat{h}}}_{(a)} + \underbrace{\err{\trainset}{\hat{h}} - \err{\trainset}{h_{\gamma}}}_{\leq 0} + \underbrace{\err{\trainset}{h_{\gamma}} - \err{\rho}{h_{\gamma}}}_{(b)} + \underbrace{\err{\rho}{h_{\gamma}} - \err{\rho}{f}}_{\text{approximation error}},
\end{align*}
where \(h_{\gamma} = \argmin_{h \in B_{\gamma}} \err{\rho}{h}\) and
\(\hat{h} = \algo(\trainset)\).

We then try to control each term, first we have that \((a)\) can be decomposed
as 
\begin{equation*}
  \err{\rho}{\hat{h}} - \err{\trainset}{\hat{h}} = \underbrace{\err{\rho}{\hat{h}} - \err{\poolset_0}{\hat{h}}}_{\text{generalisation error}} + \underbrace{\err{\poolset_0}{\hat{h}} - \err{\trainset}{\hat{h}}}_{\text{empirical drift generalisation error}},
\end{equation*}

where we have called the second quantity \emph{empirical drift generalisation error} to highlight the fact that we are training on \(\trainset\) which is a
biased sample drifting from \(\poolset_0\).

Secondly, we can decompose \((b)\) as follows,
\begin{equation*}
  \err{\trainset}{h_{\gamma}} - \err{\rho}{h_{\gamma}} = \underbrace{\err{\trainset}{h_{\gamma}} - \err{\poolset_0}{h_{\gamma}}}_{\text{empirical drift generalisation error}} +\underbrace{\err{\poolset_0}{h_{\gamma}} - \err{\rho}{h_{\gamma}}}_{\text{generalisation error}}.
\end{equation*}

We now investigate each of these expressions in turn. We start with the
generalisation error for both.

Assume that we have an arbitrary \(h \in B_{\gamma}\) and consider the following
\begin{align*}
  \abs{\err{\poolset_0}{h} - \err{\rho}{h}} = & \abs{\frac{1}{n} \sum_{i=1}^{n}(h(x_i) - y_i)^2 - \E_{\rho}[(h(x) - y)^2]} \\
                                            = & \abs{\frac{1}{n} \sum_{i=1}(h(x_i)^2 - 2h(x_i)y_i + y_i^2) - \E_{\rho}[h(x)^2 - 2h(x)y + y^2]} \\
                                            = & \abs{\frac{1}{n} \sum_{i=1}^nh(x_i)^2 - \E_{\rho_{\X}}[h(x)^2] + 2(\E_{\rho}[h(x)y] - \frac{1}{n} \sum_{i=1}^n h(x_i)y_i) + \frac{1}{n} \sum_{i=1}^n y_i^2 - \E_{\rho_{\Y}}[y^2]} \\
                                            \leq & \underbrace{\abs{\frac{1}{n} \sum_{i=1}^nh(x_i)^2 - \E_{\rho_{\X}}[h(x)^2]}}_{(i)} + 2 \underbrace{\abs{\frac{1}{n} \sum_{i=1}^n h(x_i)y_i - \E_{\rho}[h(x)y]}}_{(ii)} + \underbrace{\abs{\frac{1}{n} \sum_{i=1}^n y_i^2 - \E_{\rho_{\Y}}[y^2]}}_{(iii)}.
\end{align*}
Using the reproducing property of \ref{def:reproducing-kernel} we can express
\(f(x) = \scal{f}{K_{x}}_{\rkhs{H}}\) for any \(x \in \X\). Consider each term
individually

\begin{description}
\item[{(i)}] 
\end{description}
\begin{align*}
  \abs{\frac{1}{n} \sum_{i=1}^nh(x_i)^2 - \E_{\rho_{\X}}[h(x)^2]} & = \abs{\frac{1}{n} \sum_{i=1}^n \scal{h}{\phi(x_i)}_{\rkhs{H}}^2  - \E_{\rho_{\X}}[\scal{ h}{ \phi(x) }_{\rkhs{H}}^2]} \\
                                                                  & = \abs{\frac{1}{n} \sum_{i=1}^n \scal{h}{\scal{h}{\phi(x_i)}_{\rkhs{H}} \phi(x_i)}_{\rkhs{H}}  - \E_{\rho_{\X}}[\scal{h}{\scal{h}{\phi(x)}_{\rkhs{H}}\phi(x)}_{\rkhs{H}}]} \\
                                                                  & = \abs{\frac{1}{n} \sum_{i=1}^n \scal{h}{(\phi(x_i) \otimes \phi(x_i)) h}_{\rkhs{H}} - \E_{\rho_{\X}}[\scal{h}{(\phi(x) \otimes \phi(x)) h}_{\rkhs{H}}]} \\
                                                                  & = \abs{\scal{h}{\frac{1}{n} \sum_{i=1}^n(\phi(x_i) \otimes \phi(x_i)) h}_{\rkhs{H}} - \scal{h}{\E_{\rho_{\X}}[(\phi(x) \otimes \phi(x))] h}_{\rkhs{H}}} \\
                                                                  & = \abs{\scal{h}{(\hat{C}_{\poolsetx_{0}} - C_{\rho_{\X}})h}_{\rkhs{H}}} \\
                                                                  & \leq \norm{h}_{\rkhs{H}}^2 \norm{\hat{C}_{\poolsetx_{0}} - C_{\rho_{\X}}}_{op},
\end{align*}         
where \(C_{\rho_{\X}} = \E_{\rho_{\X}}[(\phi(x) \otimes \phi(x))]\) and
\(\hat{C}_{\poolsetx_{0}} = \frac{1}{n} \sum_{i=1}^n(\phi(x_i) \otimes
\phi(x_i))\), are the covariate and empirical covariate operator in \(\rkhs{H}\)
for \(\rho_{\X}\), where \(\otimes: \rkhs{H} \times \rkhs{H} \to \rkhs{H},
\otimes: (a, b) \mapsto c\) such that \(c(h) = a\scal{h}{b}_{\rkhs{H}}\) is the
tensor product. Since \(C_{\rho_{\X}}, \hat{C}_{\poolsetx_{0}} \in
HS(\rkhs{H})\), the set of Hilbert-Schmidt operators from \(\rkhs{H}\) to
\(\rkhs{H}\), which follows as \(\kappa < \infty\) together with an application
of the Riez representation theorem, the operator norm can be bounded by the HS
(Frobenius) norm
\begin{equation}
  \label{eq:op_leq_HS} \norm{\hat{C}_{\poolsetx_{0}} - C_{\rho_{\X}}}_{op} \leq
\norm{\hat{C}_{\poolsetx_{0}} - C_{\rho_{\X}}}_{HS}.
\end{equation}

We will now use the following lemma
\begin{lemma}[\citep{zwald06}, Lemma 1]
  \label{lem:bound-cov-operator-rkhs-hs}
  Suppose that \(\kappa^{2} \leq M\). Given any \(\rho_{\X}\) let \(\{x_{i}\}_{i=1}^{n}\) be a set sampled \(\iid\) from \(\rho_{\X}\). If \(C,
  \hat{C}_{n}\) are the true and empirical covariance operators, then with
  probability greater than \(1 - \delta\),
  \begin{equation*}
    \label{eq:bound-cov-operator-rkhs-hs}
    \norm{\hat{C}_{n} - C}_{HS} \leq \frac{2M}{\sqrt{n}}\left( 1 + \sqrt{\frac{\log(1/\delta)}{2}} \right).
  \end{equation*}
\end{lemma}
using \ref{lem:bound-cov-operator-rkhs-hs} together with \ref{eq:op_leq_HS} we have
\begin{corollary}
  \label{cor:bound-cov-op-diff-in-prob}
  With probability greater than \(1 - \delta\),
  \begin{equation}
    \label{eq:bound-cov-op-diff-in-prob}
    \norm{\hat{C}_{\poolsetx_{0}} - C_{\rho_{\X}}}_{op} \leq \frac{2\kappa^{2}}{\sqrt{n}}\left( 1 + \sqrt{\frac{\log(1/\delta)}{2}} \right)
  \end{equation}
\end{corollary}

\begin{description}
\item[{(ii)}] 
\end{description}
\begin{align*}
  \abs{\frac{1}{n} \sum_{i=1}^n h(x_i)y_i - \E_{\rho}[h(x)y]} & = \abs{\frac{1}{n} \sum_{i=1}^n \scal{ h}{ \phi(x_i) } y_i - \E_{\rho}[\scal{ h}{ \phi(x) } y]} \\
                                                              & = \abs{\scal{ h}{ \frac{1}{n} \sum_{i=1}^n \phi(x_i) y_i - \E_{\rho}[\phi(x) y] }}
\end{align*}

Let \(z_{\poolset_{0}} = \frac{1}{n} \sum_{i=1}^n \phi(x_i) y_i\) and
\(z_{\rho} = \E_{\rho}[\phi(x) y]\). Applying the CS inequality, this leads to
\begin{equation*}
  \abs{\scal{h}{\frac{1}{n} \sum_{i=1}^n \phi(x_i) y_i - \E_{\rho}[\phi(x) y] }} \leq \norm{h}_{\rkhs{H}}\norm{z_{\poolset_0} - z_{\rho}}_{\rkhs{H}}
\end{equation*}

The following lemma will let us control the norm of \(z_{\poolset_{0}} - z_{\rho}\),
\begin{lemma}[\citep{smale07_learn_theor_estim_via_integ}, Lemma 2]
  \label{lem:RKHS_hoeffding}
  let \(\rkhs{H}\) be a Hilbert space and let \(\xi\) be a random variable
  on \((\X \times \Y, \rho)\), with values in \(\rkhs{H}\). Assume
  \(\norm{\xi}_{\rkhs{H}} \leq M < \infty\) almost surely. Denote
  \(\sigma^2(\xi) = \mathbb{E}[\norm{\xi}_{\rkhs{H}}^2]\). Let
  \(\{z_i\}_{i=1}^n\) be independent random draws of \(\rho\). For any \(0 <
  \delta < 1\), with probability \(1 - \delta\) over the draws,
  \begin{equation*}
    \label{eq:RKHS_hoeffding_bound}
    \norm{\frac{1}{n}\sum_{i=1}^n\xi(z_i) - \E[\xi(z)]}_{\rkhs{H}} \leq \frac{2 M \log(2/\delta)}{n} + \sqrt{\frac{2 \sigma^2(\xi) \log(2/\delta)}{n}}
  \end{equation*}
\end{lemma}

We will use \ref{lem:RKHS_hoeffding} in order to bound the quantity
\(\norm{z_{\poolset_0} - z_{\rho}}_{\rkhs{H}}\) in probability. \(z_{i} = y_{i}
\phi(x_{i})\) is an element of \(\rkhs{H}\). These \(z\)'s will correspond to
the \(\xi\)'s of lemma \ref{lem:RKHS_hoeffding}. We can bound them through
\begin{align*}
  \norm{z}_{\rkhs{H}} & \leq \norm{y \phi(x)}_{\rkhs{H}} \\
                      & \leq \norm{y}\norm{\phi(x)}_{\rkhs{H}} \\
                      & \leq M_y\sqrt{\scal{\phi(x)}{\phi(x)}_{\rkhs{H}}} \\
                      & \leq M_y\sqrt{K(x, x)} \\
                      & \leq M_y \sup_{x \in \X} \sqrt{K(x, x)} \\
                      & \leq M_y \kappa,
\end{align*}
hence we can identify \(M\) in the lemma with \(M_{y}\kappa\).

We also have that
\begin{align*}
  \sigma^2 & = \E[\norm{y \phi(x)}_{\rkhs{H}}^2] \\
           & = \E[\abs{y}^2 \scal{\phi(x)}{\phi(x)}_{\rkhs{H}}] \\
           & = \E[\abs{y}^2 K(x, x)] \\
           & \leq M^2_y \kappa^2 \\
           & < \infty.
\end{align*}
Finally as \(\poolset_{0}\) is a set of input-output pairs sampled \(\iid\) from \(\rho\)
we have that the \(z_{i} = y_{i}\phi(x_{i})\) are sampled \(\iid\) as well. Thus
all conditions of the lemma \ref{lem:RKHS_hoeffding} are satisfied.

Thus we have that for any \(\delta \in (0, 1)\), with  probability \(1 -
\delta\) the following holds
\begin{equation}
  \label{eq:zs_RKHS_hoeffding}
  \norm{z_{\poolset_0} - z_{\rho}}_{\rkhs{H}} \leq \frac{2 M_y \kappa \log(2/\delta)}{n} + \sqrt{\frac{2 M_y^2 \kappa^2 \log(2/\delta)}{n}} = M_y \kappa \left(\frac{2\log(2/\delta)}{n} + \sqrt{\frac{2\log(2/\delta)}{n}} \right).
\end{equation}

\begin{description}
\item[{(iii)}] 
\end{description}
\begin{equation*}
  \abs{\frac{1}{n} \sum_{i=1}^n y_i^2 - \E_{\rho_{\Y}}[y^2]}
\end{equation*}

We can bound this using assumption \ref{as:output-bounded} which says that any
\(y \in \Y\) is such that \(y \leq M_y\). First we introduce Hoeffding's inequality,
\begin{theorem}[\citep{ciliberto18_advan_topic_machin_learn}, Hoeffdings Inequality]
  \label{th:univariate-hoeffdings-ineq}
  Let \(X_{1}, \dots, X_{n}\) be independent random variables such that \(X_{i}
  \in [a_{i}, b_{i}]\) almost surely. Let \(\overline{X} = \frac{1}{n}
  \sum_{i=1}^n X_i\). Then
  \begin{equation*}
    \label{eq:univariate-hoeffdings-ineq}
    \Pr(\abs{\overline{X} - \E[\overline{X}]} \geq \epsilon) \leq 2 \exp(-\frac{2n^2\epsilon^2}{\sum_{i=1}^n(b_i - a_i)^2}).
  \end{equation*}
\end{theorem}
Applying \ref{th:univariate-hoeffdings-ineq} to the random variables
\(\{y_{i}\}_{i=1}^{n}\) which are sampled \(\iid\) and using that \(\abs{y}^{2}
\leq M_{y}^{2}\) we have that for any \(\delta \in (0, 1)\), with probability at
least \(1 - \delta\) over the samples the following holds
\begin{equation}
  \label{eq:y2s_hoeffding}
  \abs{\frac{1}{n} \sum_{i=1}^n y_i^2 - \E_{\rho_{y}}[y^2]} \leq \sqrt{\frac{M_y^4 \log(2/\delta)}{2n}} = M_y^2\sqrt{\frac{\log(2/\delta)}{2n}}.
\end{equation}

Combining \((i), (ii)\) and \((iii)\) we proceed to bound this uniformly using
the assumption that \(h \in B_{\gamma}\),
\begin{align*}
  \sup_{\norm{h}_{\rkhs{H}} \leq \gamma} \abs{\err{\poolset_0}{h} - \err{\rho}{h}}  & \leq \sup_{\norm{h}_{\rkhs{H}} \leq \gamma} \norm{h}_{\rkhs{H}}^2\norm{\hat{C}_{\poolsetx_0} - C_{\rho_{\X}}}_{op}  + \sup_{\norm{h}_{\rkhs{H}} \leq \gamma} \norm{h}_{\rkhs{H}} \norm{z_{\poolsetx_0} - z_{\rho}}_{\rkhs{H}} \\
                                                                                    & + \abs{\frac{1}{n} \sum_{i=1}^n y_i^2 - \E_{\rho_{\Y}}[y^2]} \\
                                                                                    & = \gamma^2 \norm{\hat{C}_{\poolsetx_0} - C_{\rho_{\X}}}_{op} + \gamma \norm{z_{\poolsetx_0} - z_{\rho}}_{\rkhs{H}} + \abs{\frac{1}{n} \sum_{i=1}^n y_i^2 - \E_{\rho(y)}[y^2]}.
\end{align*}
Combining this with results
\ref{eq:bound-cov-op-diff-in-prob}, \ref{eq:zs_RKHS_hoeffding},
\ref{eq:y2s_hoeffding} and using \ref{lem:combining_slt_bounds} we have that with probability \(1 - 3\delta\) over \(\poolset_0\)
\begin{equation}
\label{eq:uniform-bound-gen-error-over-ball}
\begin{split}
  \sup_{\|h\|_{\rkhs{H}} \leq \gamma} \abs{\err{\poolset_0}{h} -\err{}{h}} \leq
  \frac{2 \kappa^2 \gamma^2}{\sqrt{n}}  \left(1 + \sqrt{\frac{\log(1 / \delta)}{2}} \right) & + \frac{2 M_y \gamma \kappa \log(2/\delta)}{n}  \\
  & + \sqrt{\frac{2 M_y^2 \kappa^2 \log(2/\delta)}{n}} + M_y^2\sqrt{\frac{\log(2/\delta)}{2n}}.
\end{split}
\end{equation}

Since \(\hat{h}, h_{\gamma} \in B_{\gamma}\) we have bounded both of the
generalisation terms in \((a), (b)\).

\subsection{Bounding in Probability}
From \ref{eq:MMD-gen-bound} we have that for \(S \subseteq P_{0}\) and
letting \(\rho_{\trainsetx} = \frac{1}{m} \sum_{x \in \trainsetx}\delta_{x}\)
, both of the empirical drift generalisation errors in \((a), (b)\) can be
bounded above by \(\MMD{\poolsetx_{0}}{\trainsetx}{B_{\eta}} + \eta_{MMD}\),
and from \ref{cor:gauss-kernel-squared-gives-eta-zero} if we let \(\eta =
4\gamma^{2}\) and assume that \(f \in B_{\gamma}\) then \(\eta_{MMD} = 0\). We will choose \(K'\) such that
\ref{cor:gauss-kernel-squared-gives-eta-zero} holds.

Combining this we have that
\begin{theorem}
  \label{thm:al-non-generalisation-bound}
  Given that \(f \in B_{\gamma}\), \(\eta = 4\gamma^{2}\) and for any \(x, x' \in
  \X\), \(K'(x, x') = K(x, x')^2\) the following bound holds for \(\poolset_{0} \sim \rho^{n}\) and any \(S \subseteq \poolset_0\),
  \begin{equation*}
    \label{eq:al-non-generalisation-bound}
    \err{\rho}{\hat{h}} - \err{\rho}{f} \leq 2\sup_{\|h\| \leq \gamma}\abs{\err{\rho}{h} - \err{\poolset_0}{h}} + 2 \MMD{\poolset_0}{\trainset}{B_\eta}.
  \end{equation*}
\end{theorem}

\begin{proof}
  This is simply an application of the results developed in the
  \hyperref[sec:error-decomposition]{Error Decomposition} section. We have the following
  \begin{align*}
    \err{\rho}{\hat{h}} - \err{\rho}{f} & = \err{\rho}{\hat{h}} - \err{\trainset}{\hat{h}} + \underbrace{\err{\trainset}{\hat{h}} - \err{\trainset}{h_{\gamma}}}_{\leq 0} + \err{\trainset}{h_{\gamma}} - \err{\rho}{h_{\gamma}} + \err{\rho}{h_{\gamma}} - \err{\rho}{f} \\ 
                                        & \leq \err{\rho}{\hat{h}} - \err{\trainset}{\hat{h}} + \err{\trainset}{h_{\gamma}} - \err{\rho}{h_{\gamma}} + \err{\rho}{h_{\gamma}} - \err{\rho}{f} \\
                                        & \leq \abs{\err{\rho}{\hat{h}} - \err{\trainset}{\hat{h}} + \err{\trainset}{h_{\gamma}} - \err{\rho}{h_{\gamma}} + \err{\rho}{h_{\gamma}} - \err{\rho}{f}} \\
                                        & \leq \abs{\err{\rho}{\hat{h}} - \err{\trainset}{\hat{h}}} + \abs{\err{\trainset}{h_{\gamma}} - \err{\rho}{h_{\gamma}}} + \underbrace{\abs{\err{\rho}{h_{\gamma}} - \err{\rho}{f}}}_{= 0} \\
                                        & \leq 2 \left( \sup_{\|h\| \leq \gamma}\abs{\err{\rho}{h} - \err{\poolset_0}{h}} + \MMD{\poolset_0}{\trainset}{B_\eta} \right),
  \end{align*}
  where in the final line we have used the bounds on the generalisation and empirical drift generalisation error.
\end{proof}
From the first result in the table \ref{tbl:FW-convergence-table} we can bound the
quantity \(\MMD{\poolsetx_{0}}{\trainsetx}{B_{\eta}}\) as follows, let
\(\trainset_{t}\) be the set built from \(\poolset_{0}\) by running FW on
\(\frac{1}{2}\norm{\mu_{\poolsetx_0} - \mu_{\trainsetx_{t}}}_{\rkhs{H}'}^2\) for
\(t\) iterations, then
\begin{align*}
  \label{align:fw-worst-bound-on-mmd}
  \MMD{\poolsetx_{0}}{\trainsetx_{t}}{B_{\eta}} & = \sqrt{2\eta\frac{1}{2}\norm{\mu_{\poolsetx_0} - \mu_{\trainsetx_{t}}}_{\rkhs{H}'}^2} \\
                                                & \leq \sqrt{2\eta \frac{4R^2}{t}}.
\end{align*}
When using the Gaussian kernel, \(R = 1\) and the bound reduces to
\(\MMD{\poolsetx_{0}}{\trainsetx_{t}}{B_{\eta}} = 2\sqrt{\frac{2\eta}{t}}\). For
the Gaussian kernel case we have the following, by using
\ref{eq:uniform-bound-gen-error-over-ball} to bound the term \(\sup_{\|h\| \leq \gamma}\abs{\err{\rho}{h} - \err{\poolset_0}{h}}\),
\begin{theorem}
  \label{al:worst-fw-generalisation-bound-no-additional-assumptions}
  In addition to \cref{as:al-fw,as:domain-is-compact,as:output-bounded,as:f-in-gamma-ball,as:realisable-setting}, assume that \(\eta = 4\gamma^{2}\), \(K'(x, x') = K(x,
  x')^{2}\) and that \(\trainset_t\) is built using FW, \(\hat{h} = \algo(\trainset_t)\) is Ivanov
  Kernel Regression. Then with probability greater than \(1 - 3\delta\) the following bound holds
  \begin{align*}
    \err{\rho}{\hat{h}} - \err{\rho}{f} & \leq \frac{4 \gamma^2}{\sqrt{n}}\left(1 + \sqrt{\frac{\log(1/\delta)}{2}} \right) + \frac{4 M_y \gamma \log(2/\delta)}{n} + 2\sqrt{\frac{2 M_y^2 \log(2/\delta)}{n}} + 2M_y^2\sqrt{\frac{\log(2/\delta)}{2n}} \\
                                        & + 4\sqrt{\frac{2\eta}{t}}
  \end{align*}
\end{theorem}

While this is a bound, it is not an improvement over Monte-Carlo (hereafter \textit{MC}) sampling.
Consider a sampling strategy that uniformly at random picks instances in
\(\poolsetx_0\), this is the same as uniformly sampling from
\(\poolsetx_0\) without replacement. Since the set \(\trainset_{t}\) is now an
empirical \(\iid\) sample from \(\poolset_{0}\) we have that
\(\MMD{\poolsetx_{0}}{\trainsetx_{t}}{B_{\eta}} = O_{P}(\frac{1}{\sqrt{t}})\)
\citep{tolstikhin17_minim_estim_kernel_mean_embed} which is of the same order as
using FW. This is the baseline we would like to improve upon.

\subsection{Lower Bounding the Radius}
In order to have a generalisation bound which is an improvement over MC sampling
we would have to show that in probablity we can lower bound the quantity
\ref{eq:d-radius-of-biggest-interior-ball} so that we would have a convergence
rate of the second entry of table \ref{tbl:FW-convergence-table}. We proceed as
follows, let \(A = \{\mat{\alpha} \in \R^n | \mat{\alpha} \succeq 0, \norm{\mat{\alpha}}_1 = 1,
\norm{\mat{\alpha}}_0 < n\}\), the faces of the simplex
\(\triangle_{n-1}\), then the radius \(d\) of
\ref{eq:d-radius-of-biggest-interior-ball} as a function of \(\poolsetx_{0}\)
can be bounded as follows
\begin{align*}
  d(\poolsetx_0)^2 & = \min_{\varphi \in \partial_{relint} \mathcal{M}}\norm{\mu_{\poolsetx_0} - \varphi}^2_{\rkhs{H}'} \\
                   & = \min_{A}\norm{\frac{1}{n}\sum_{i=1}^n\phi(x_i) - \sum_{i=1}^n \alpha_{i} \phi(x_{i})}^2_{\rkhs{H}'} \\
                   & = \min_{A} \norm{\mat{K}_{nn}^{1/2}(\mathbf{1}/n - \mat{\alpha})}_{2}^2 \\
                   & \geq \lambda_{min}(\mat{K}_{nn}) \min_{A} \norm{\mathbf{1}/n - \mat{\alpha}}_{2}^2.
\end{align*}

We first consider the term \(\min_{A} \norm{\mathbf{1}/n -
\mat{\alpha}}_{2}^2\). Note that this is just the minimum of the distances from
the barycenter \(\mathbf{1}/n\) onto any of the faces of \(\triangle_{n-1}\).
Now, the result will be the same up to permutation
of what entries in \(\mat{\alpha}\) that is set to zero, so without loss of
generality we will denote \(\mat{\alpha}^{k} = (\alpha_{1}, \dots, \alpha_{k},
0, \dots, 0)^{T} \in A\). Let \(F_{k}\) be the face generated by \(k\)
non-zero vectors \(\mat{\alpha}^{k}\), then \(F_{k} \subseteq F_{l}\) when \(k
< l\). Thus, we can write, letting \([T] = \{1, \dots, T\}\),
\begin{equation}
  \label{eq:d-optim-problem}
  \min_{F_{k}, k \in [n-1]} \norm{\mathbf{1}/n - \mat{\alpha}}_{2}^2 = \min_{F_{n-1}}\norm{\mathbf{1}/n - \mat{\alpha}}_{2}^2
\end{equation}
where we used the fact that for any \(k < n\), \(F_{k} \subseteq F_{n-1}\) and
so we only have to minimise over \(F_{n-1}\) which is a convex set.

We claim that the optimum over the \hyperref[def:aff-hull]{affine hull} of \(F_{n-1}\) of
\ref{eq:d-optim-problem} is attained on \(F_{n-1}\), that is
\(\norm{\mathbf{1}/n - \alpha^{n-1}}_{2}^{2}\) can be optimised over \(A_{n-1}
\coloneqq \affhull(F_{n-1})\). To see this consider the family of vectors
\(\beta^{n-1} = (\beta_{1}, \dots, \beta_{n-1}, 0)^{T}\) where each \(\beta_{i}
\in \R\) and \(\sum_{i=1}^{n-1}\beta_{i} = 1\), then we see that
\begin{equation}
  \label{eq:minimal-d-affhull-opt-problem}
  \min_{\mat{\beta} \in A_{n-1}}\norm{\mathbf{1}/n - \mat{\beta}}^{2}_{2} = \min_{\mat{\beta} \in A_{n-1}} \sum_{i=1}^{n-1} (n^{-1} - \beta_{i})^{2} + n^{-2}.
\end{equation}

We can solve this using the technique of Lagrange multipliers. The corresponding
Lagrangian to
\ref{eq:minimal-d-affhull-opt-problem} is
\begin{equation*}
  \lagr(\mat{\beta}, \gamma) = \sum_{i=1}^{n-1} (n^{-1} - \beta_{i})^{2} + n^{-2} + \gamma (\sum_{i=1}^{n-1} \beta_i - 1)
\end{equation*}
and using the KKT conditions, we can replace the primal variables
\(\mat{\beta}\). Consider for any \(i \in [n-1]\),
\begin{align*}
  \pdv{\lagr}{\beta_{i}} & = 2(\beta_{i} - n^{-1}) + \gamma \\
                         & = 0
\end{align*}
which means that we can set \(\beta_{i} = n^{-1} - \frac{1}{2}\gamma\).
Substituting back the new expression for the \(\beta_i\)'s we then have the
following
\begin{align*}
  \lagr(\gamma) & = \sum_{i=1}^{n-1} (\frac{1}{2} \gamma)^{2} + \gamma (\sum_{i=1}^{n-1} (n^{-1} - \frac{\gamma}{2}) - 1) \\
                & = \frac{1}{4}(n-1)\gamma^{2} + \gamma((1 - n^{-1}) - \frac{n-1}{2}\gamma - 1) \\
                & = \frac{1}{4}(n-1)\gamma^{2} - \frac{1}{2}(n-1)\gamma^{2} - n^{-1} \gamma \\
                & = -\frac{1}{4}(n-1)\gamma^{2} - n^{-1} \gamma.
\end{align*}
This expression has the maximum at \(\gamma^{\ast} = -\frac{2}{n(n-1)}\) which
gives that each \(\beta_{i} = n^{-1} + n^{-1}(n-1)^{-1} = \frac{1}{n-1}\). It is
clear that this solution is in \(F_{n-1}\) and hence we have solved the problem.
Substituting this in for the \(\mat{\alpha}\) in
\ref{eq:d-optim-problem}, we have that
\begin{align*}
  \min_{F_{n-1}}\norm{\mathbf{1}/n - \mat{\alpha}}_{2}^2 & = \sum_{i=1}^{n-1}(n^{-1} - (n-1)^{-1})^{2} + n^{-2} \\
                                                         & = (n-1)^{-1}n^{-2} + n^{-2} \\
                                                         & = n^{-2}(1 + (n-1)^{-1}) \\
                                                         & = \frac{1}{n(n-1)} \\
                                                         & > \frac{1}{n^{2}}
\end{align*}
We have thus shown that we can bound \(d\) as
\begin{equation*}
  \label{eq:lower-bound-on-d}
  d(\poolsetx_{0}) \geq \frac{\sqrt{\lambda_{min}(\mat{K}_{nn})}}{n}.
\end{equation*}

Unfortunately this bound is vacuous as for this to be of relevance to us we
require \(\frac{\sqrt{\lambda_{min}(\mat{K}_{nn})}}{n} \geq c > 0\) for some
scalar \(c\) in probability over \(\poolset_{0}\). This would mean that
\(\lambda_{min}(\mat{K}_{nn}) = \Omega(n^{2})\) in probability which does not go
well with well-established assumptions on decay of the eigenvalues of the kernel
matrix, for example that \(\lambda_{min}(\mat{K}_{nn}) \leq \trace(\mat{K}_{nn})
= \Theta(n)\) according to \citep{bach13_sharp}.

\section{Concluding Remarks}
At this point we have done the following; investigated the empirical excess risk
bound of \ref{eq:MMD-gen-bound} and considered optimising the MMD term
according to current theory on FW and KH which shows that choosing instances greedily
with respect to the MMD yields convergence of the term on the order of
\(O(\frac{1}{t})\) which is a significant improvement over random sampling which
converges on the order of \(O_P(\frac{1}{\sqrt{t}})\). Inspired by this we consider
the true excess risk and derive a bound of it in high probability by using SLT
error decomposition techniques. We derive an upper bound on this that features
an MMD term which we can optimise using FW, but current theory does not say if
using FW in this case leads to speedups as we need to consider the sampled
pool set as random. Nevertheless, it has been shown that FW dominates random
(MC) sampling empirically \citep{bach12_equiv_between_herdin_condit_gradien_algor},
and this is validated in the \hyperref[ch:experiments]{Experiments} section. We
see this as an opportunity for future research to discover conditions that would
give us theoretical guarantees of FW convergence dominating that of random sampling.

\chapter{Experiments}
\label{ch:experiments}

\section{Setup}

As before, we will use a gaussian kernel \(K_{\sigma}(x, x') =
\exp(-\frac{\norm{x - x'}_2^2}{2\sigma^2})\). We split the experiments into
\textit{agnostic} and \textit{realisable}. All datasets were normalised in both
input and output over the whole dataset. We produce learning curves on the
test set in order to analyse the generalisation performance of the active
learning algorithms.

For agnostic we do the following: \(D\) is the total dataset Due
to computational feasibility, we have a threshold \(n_{sub}\), if the size of
\(D\) is greater than this, we randomly subsample \(n_{sub}\) datapoints from
\(D\) and let \(D\) be this smaller dataset. \(D\) is then split into two sets,
\(\trainset_{train+test}\) (80\%) and \(\trainset_{val}\) (20\%). We use
\(\trainset_{val}\) to do 5-fold cross validation \((kCV)\) and get optimal
hyperparamters \(\sigma_{opt}, \lambda_{opt}\) for algorithm \(\algo\) which will
be KRR.

The dataset \(\trainset_{train+test}\) is split into 5 folds \([S_1, \dots,
S_5]\) and for each \(i \in \{1, \dots, 5\}\) a train set
is chosen by letting \(\trainset_{train} = S_i\), and test set
\(\trainset_{test} = \trainset_{train+test} \setminus \trainset_{train}\). The
train set will correspond to \(\poolset_{0}\). \(\trainset_{0} =
\emptyset\) and for the first iteration, \(t=1\), we add the first instance
\(x_{1} \in \poolset_{0}\) so that \(\trainset_{1} = \{(x_{1}, y_{1})\}\) and
after this we use the querying strategy \(\querstrat\) at each \(t > 1\) to
choose what instance to query\footnote{This is necessary as the algorithms
  require an instance for the updates to be well-defined.}. At each time \(t\),
after querying an instance, we fit the algorithm \(\algo\) to the current train
set \(\trainset_{t}\) giving us a hypothesis \(\hat{h}_{t}\) with which we get
the current loss \(l_{t} = \err{\trainset_{test}}{\hat{h}_{t}}\) where the loss
will be the squared loss for regression (MSE) and the zero-one loss for
classification (Accuracy).

For the realisable setting, we proceed almost in the same way. The only
difference is that after potentially subsampling \(D\), we get the optimal
hyperparameters doing 5-fold \(kCV\) on the dataset \(D\), fit a ground-truth
hypothesis using KRR with these hyperparameters, which we'll call
\(h_{realisable}\). Finally we replace the outputs of \(D = \{(x_i,
y_i)_{i=1}^n\}\) with \(h_{realisable}(x_i)\). This will ensure that the
assumptions of no noise and realisability are satisfied. The rest is the same as
in the agnostic setting.

We compare the FW (KH, step-size \(\rho_t = \frac{1}{1+t}\)) algorithm with
random subsampling (MC) and Leverage Score sampling (Levscore (deterministic)) \cite{rudi18}, where we
calculate the leverage scores and pick the instances in descending order of the
leverage scores (we choose the \(\lambda\) of the leverage scores to be the same
as \(\lambda_{opt}\) gotten through \(kCV\)).

\section{Datasets}

We evaluate the algorithms on 5 regression datasets and 2 classification
datasets. These were sourced from the UCI dataset repository \cite{dua17_uci_machin_learn_repos} and through the
sklearn dataset API \cite{dua17_uci_machin_learn_repos}. Some of the datasets
were changed to remove non-continuous variables. The datasets are the following

\section{Analysis}

In this section we will analyse a subset of the learning curves produced. All of
the learning curves can be found in the appendix for the
\hyperref[figs:agnostic-regression-learning-curves]{agnostic regression},
\hyperref[figs:realisable-regression-learning-curves]{realisable regression} and
\hyperref[figs:agnostic-classification-learning-curves]{agnostic
  classification}. The rest of the learning curves are similar and do not retract
from the analysis in any significant way.

Since there is no standardised way on how to evaluate the performance of active
learning algorithms against each other empirically, we will analyse the curves
directly.

\begin{figure}[h] \centering
  \label{fig:learning-curve-boston-agnostic-experiments}
  \includegraphics[width=0.8\textwidth]{../active_learning_code/reports/figures/learning_curves_k_fold-boston.png}
  \caption{Learning curve (bold line is mean with shaded region being \(\pm 1\)
    standard deviation over the 5 folds) for agnostic regression on Boston dataset
    comparing FW (KH) with MC and Levscore. The \(y\)-axis is MSE (Mean Squared
    Error), \(x\)-axis is \(t\) (number of datapoints in current active learning
    train set). A good algorithm will have a trajectory that goes down quickly with
    \(t\). MC is baseline.}
  \label{fig:mesh1}
\end{figure}

\begin{figure}[h] \centering
  \label{fig:learning-curve-boston-realisable-experiments}
  \includegraphics[width=0.8\textwidth]{../active_learning_code/reports/figures/learning_curves_k_fold_realisable-boston.png}
  \caption{Learning curve (bold line is mean with shaded region being \(\pm 1\)
    standard deviation over the 5 folds) for realisable regression on Boston dataset
    comparing FW (KH) with MC and Levscore. The \(y\)-axis is MSE (Mean Squared
    Error), \(x\)-axis is \(t\) (number of datapoints in current active learning
    train set). A good algorithm will have a trajectory that goes down quickly with
    \(t\). MC is baseline.}
  \label{fig:mesh1}
\end{figure}


For agnostic regression (here the
\hyperref[fig:learning-curve-boston-agnostic-experiments]{Boston dataset}), in
general we see that FW (KH) performs well, going down quickly in the start and
outperforming MC (in mean) at all \(t\). Levscore performs poorly for the first
50 \(t\) which can be explained by the fact that it does not take into account
the already queried instances, only focusing on those with high leverage. This
means that while it focuses on \textit{informativeness} of the instances chosen,
it does not implement a way to choose \textit{representative} instances.
Nevertheless it performs well towards the end of the plot. MC goes down slowly
which is what you would expect, since it does not care about informativeness but
only about representativeness. FW (KH) arguably care about both as it chooses
instances in such a way that the set \(\trainsetx_t\) resembles \(\poolsetx_0\)
which means that it will focus early on building a train set that contains the
instances that can explain the statistical properties of \(\poolsetx_0\). The
analysis for the realisable setting (as in here for
\hyperref[fig:learning-curve-boston-realisable-experiments]{Boston dataset}) is
essentially the same, although we see that the curves goes towards 0 which is to
be expected since now the function \(f\) is in the hypothesis class which is not
true for the agnostic case.

\begin{figure}[h] \centering
  \label{fig:learning-curve-mnist-agnostic-experiments}
  \includegraphics[width=0.8\textwidth]{../active_learning_code/reports/figures/learning_curves_k_fold-mnist.png}
  \caption{Learning curve (bold line is mean with shaded region being \(\pm 1\)
    standard deviation over the 5 folds) for agnostic classification on mnist
    comparing FW (KH) with MC and Levscore. The \(y\)-axis is Accuracy, \(x\)-axis
    is \(t\) (number of datapoints in current active learning train set). A good
    algorithm will have a trajectory that goes up quickly with \(t\). MC is
    baseline.}
\end{figure}

For agnostic classification (here the
\hyperref[fig:learning-curve-mnist-agnostic-experiments]{mnist dataset}) we see
that FW (KH) does a very good job at choosing good instances. The red curve
outperforms MC and Levscore essentially everywhere except for in the very
beginning. Levscores poor performance can be explained that it is mainly a
method to choose instances that has high leverage for regression, but this does
not apply when doing classification. An explanation for the good performance of
FW (KH) is that it is practically \textit{mode seeking} as can be seen in
\hyperref[fig:kh-is-mode-seeking]{this figure}.

\begin{figure}[h]
  \centering
  \label{fig:kh-is-mode-seeking}
  \includegraphics[width=1.0\textwidth]{../active_learning_code/reports/figures/kh_is_mode_seeking_mog.png}
  \caption{Comparison of the first 9 instances chosen by FW (KH), MC and
    Levscore. The data was generated by sampling 50 datapoints from 9 gaussian
    distributions with means from the set \(\{(i, j)| i, j \in \{-1, 0, 1\}\}\)  and covariance matrix being the
    identity matrix scaled by 0.01. The kernel matrix was created by using a
    gaussian kernel with \(\sigma^{2} = 0.02\) which was fine-tuned to exhibit
    this phenomenon.}
\end{figure}
\chapter{Conclusions and Future Directions}
\label{ch:conclusions}

% Should respond to
% Research objectives
% Summary of Findings
% and Resulting Conclusions
% + Recommendations

In this dissertation we have done the following; introduced the field of active
learning, where we produced an overview of the field in general before diving
deep into the current work on active learning for regression using RKHS theory,
statistical learning theory and optimisation in the form of Kernel Herding and
Frank Wolfe. Since the work married ideas from different fields we also provided
a review of the necessary concepts of RKHS theory from kernels to MMD and the
necessary concepts for supervised learning from a statistical learning point of view.

From this we identified work in the field of active learning
regression upon which we base our work and relate it to. Using an error bound
which split up into a data-fitting term and a domain drift term in the form of
the MMD between the original sampled dataset and the built train set, we apply
FW to this in order to optimise it, which guarantees a theoretical convergence
which dominates random sampling.

We decompose the excess risk of the estimator of the active learning
algorithm and derived an upper bound on this which holds with high probability.
Given current theoretical results, we show that this do not yield an
improvement over random sampling, however this might change in the future, since
the conditions given for FW to converge faster are sufficient, but not
necessary.

We investigate the performance empirically and see that FW in the form of KH
does well when compared to the random sampling baseline (MC) and against
Leverage Scores (LS-rand) which is a commonly used method in regression for subsampling a
dataset or doing dimensionality reduction. We show that this holds both for the
realisable regression case where we can control the complexity term in the empirical bound
by making it zero, but also empirically in the agnostic regression and classification
setting. For classification this can potentially be explained by KH being mode seeking.

We conclude that KH and in general FW is a competitive way of choosing which
instances to label for active learning and that it has good properties whenever
there is an MMD bound to optimise. The family of FW algorithms have a convergence rate which is
provably faster over random sampling which enables us to reduce the empirical
risk fast, and work well in practice on both regression and classification. This
is as far as we are aware the only place where the theoretical properties of FW
has been used to show in theory how FW can give guarantees when optimising this
empirical error bound for active learning.

There are various avenues for further research. It would be fruitful to
investigate the setting of output noise and extend the framework to see how FW
can be leveraged. In addition to this it would also be interesting and helpful to
consider losses other than the squared error loss which would require taking the
output of the algorithm into account, making it adaptive. Finally, I believe
there is a potential direction of using dimensionality reduction together with
active learning akin to what is done in \cite{rudi18} in order to reduce the
complexity of the training and prediction of KRR, meaning we could apply the
active learning algorithm to bigger datasets without the need to subsample. In
this way it would be reasonable to look at random projection techniques.


\chapter{Appendix}

\section{Proofs}
\label{sec:appendix-proofs}
% \printproofs

\section{Convex Analysis}
We introduce the necessary concepts from convex analysis used in the analysis of
convergence of KH and FW. We will follow \cite{boyd04_convex}.

\begin{definition}
  \label{def:conv-hull}
  The Convex Hull of a set \(\mathcal{C}\) is denoted by \(\convhull(C)\) and is defined as
  \begin{equation}
    \label{eq:conv-hull}
    \convhull(\mathcal{C}) = \{ \sum_{i \in I}^n \theta_{i} x_i | \theta_i \geq 0, \forall i \in I, \sum_{i \in I} \theta_i = 1 \}
  \end{equation}
  where \(I\) is an index set of elements in \(\mathcal{C}\).
\end{definition}

\begin{definition}
  \label{def:aff-hull}
  The Affine Hull of a set \(\mathcal{C}\) is denoted by \(\affhull(C)\) and is defined as
  \begin{equation}
    \label{eq:aff-hull}
    \affhull(\mathcal{C}) = \{ \sum_{i \in I}^n \theta_{i} x_i | \theta_i \in \R, \forall i \in I, \sum_{i \in I} \theta_i = 1 \}
  \end{equation}
  where \(I\) is an index set of elements in \(\mathcal{C}\).
\end{definition}

\begin{definition}
  \label{def:relative-interior}
  The Relative Interior of a set \(\mathcal{C}\) is denoted by \(\relint(C)\) and is defined as
  \begin{equation}
    \label{eq:relative-interior}
    \relint(\mathcal{C}) = \{x \in \mathcal{C} | B_{r}(x) \cup \affhull(\mathcal{C}) \subseteq \mathcal{C}, \text{ for some } r \geq 0\}
  \end{equation}
\end{definition}

\begin{definition}
  \label{def:relative-boundary}
  The Relative boundary of a set \(\mathcal{C}\) is denoted by \(\relbound \mathcal{C}\) and is defined as
  \begin{equation}
    \label{eq:relative-boundary}
    \relbound \mathcal{C} = \clos(\mathcal{C}) \setminus \relint(\mathcal{C})
  \end{equation}
  where \(\clos(\mathcal{C})\) is the closure of \(\mathcal{C}\).
\end{definition}

\section{Experiments}
\subsection{Procedures}
\FloatBarrier

\begin{algorithm}[htb]
  \label{alg:experiment-agnostic}
  \caption{Experimental procedure (Agnostic)}
  \begin{algorithmic}[1]
    \Procedure{Experiment}{$D$, $\algo$, $\querstrat$, $\ell$, $k_{cv}$,
      $k_{\ell}$, $n_{sub}$, $n_{val}$} \Comment{\(D\) is the dataset, \(\algo\)
      is the algorithm, \(\querstrat\) is the querying strategy, \(k_{cv}\) is
      the number of folds in \(kCV\), \(k_{l}\) is the number of folds over
      \(D_{train+test}\) that we run to get \(k_{l}\) learning curves,
      \(n_{sub}\) is threshold for subsampling and \(n_{val}\) is the size of the
      validation split.}
    \State $\mat{L} \gets \mat{0}_{k, n}$ \Comment{\(k_l\) by \(n\) empty matrix for losses for timesteps \(t=1:n\) for folds \(i=1:k_l\)}
    \State $n_{D} \gets |D|$
    \If {$n_{D} > n_{sub}$} \Comment{If dataset too big, subsample}
    \State $D \gets \textrm{subsample}(D, n_{sub})$
    \State $n_{D} \gets |D|$
    \EndIf
    \State $S_{val} \gets D_{1:n_{val}}$
    \State $\sigma_{opt}, \lambda_{opt} \gets kCV(S_{val}, \algo, k_{cv})$ \Comment{Get optimal hyperparameters through k-fold CV}
    \State $S_{train+test} \gets S_{n_{val}+1:n}$
    \State $\sigma'_{opt} \gets \frac{\sigma_{opt}}{\sqrt{2}}$ \Comment{Set \(\sigma_{opt}'\) according to \ref{cor:gauss-kernel-squared-gives-eta-zero}}
    \State $[S_{1}, S_{2}, \dots, S_{k}] \gets S_{train+test}$ \Comment{Split train / test data in k folds}
    \For {$i = 1:k_{l}$} \Comment{Get learning curve for fold \(i\)}
    \State $\trainset_{test} \gets S_{i}$
    \State $\poolset_{0} \gets S_{train+test} \setminus \trainset_{test}$
    \State $n \gets |\poolset_{0}|$
    \For {$t = 1:n$} \Comment{Run active learning algorithm}
    \State $x_{t}^{q} \gets \querstrat(\poolsetx_{t-1})$
    \State $y_{t}^{q} \gets \oracle(x_{t}^{q})$
    \State $\trainset_{t} \gets \trainset_{t-1} \cup \{(x_{t}^{q}, y_{t}^{q})\}$
    \State $\poolset_{t} \gets \poolset_{t-1} \setminus \{(x_{t}^{q}, y_{t}^{q})\}$
    \State $\hat{h}_{t} \gets \algo(\trainset_{t})$
    \State $\mat{L}_{i, t} \gets \err{S_{test}}{\hat{h}_{t}}$ \Comment{Fill out loss matrix for current fold and step}
    \EndFor
    \EndFor
    \EndProcedure
  \end{algorithmic}
\end{algorithm}

\begin{algorithm}[htb]
  \label{alg:experiment-realisable}
  \caption{Experimental procedure (Realisable)}
  \begin{algorithmic}[1]
    \Procedure{Experiment}{$D$, $\algo$, $\querstrat$, $\ell$, $k_{cv}$,
      $k_{\ell}$, $n_{sub}$, $n_{val}$} \Comment{\(D\) is the dataset, \(\algo\)
      is the algorithm, \(\querstrat\) is the querying strategy, \(k_{cv}\) is
      the number of folds in \(kCV\), \(k_{l}\) is the number of folds over
      \(D_{train+test}\) that we run to get \(k_{l}\) learning curves,
      \(n_{sub}\) is threshold for subsampling and \(n_{val}\) is the size of the
      validation split.}
    \State $\mat{L} \gets \mat{0}_{k, n}$ \Comment{\(k_l\) by \(n\) empty matrix for losses for timesteps \(t=1:n\) for folds \(i=1:k_l\)}
    \State $n_{D} \gets |D|$
    \If {$n_{D} > n_{sub}$} \Comment{If dataset too big, subsample}
    \State $D \gets \textrm{subsample}(D, n_{sub})$
    \State $n_{D} \gets |D|$
    \EndIf
    \State $\sigma_{opt}, \lambda_{opt} \gets kCV(D, \algo, k_{cv})$ \Comment{Get optimal hyperparameters through k-fold CV}
    \State $h_{realisable} = \algo(D)$ \Comment{Fit new hypothesis and create
      new labels}
    \State $D \gets \{(x_i, h_{realisable}(x_i)\}_{i=1}^{n}$
    \State $S_{val} \gets D_{1:n_{val}}$
    \State $S_{train+test} \gets S_{n_{val}+1:n}$
    \State $\sigma'_{opt} \gets \frac{\sigma_{opt}}{\sqrt{2}}$ \Comment{Set \(\sigma_{opt}'\) according to \ref{cor:gauss-kernel-squared-gives-eta-zero}}
    \State $[S_{1}, S_{2}, \dots, S_{k}] \gets S_{train+test}$ \Comment{Split train / test data in k folds}
    \For {$i = 1:k_{l}$} \Comment{Get learning curve for fold \(i\)}
    \State $\trainset_{test} \gets S_{i}$
    \State $\poolset_{0} \gets S_{train+test} \setminus \trainset_{test}$
    \State $n \gets |\poolset_{0}|$
    \For {$t = 1:n$} \Comment{Run active learning algorithm}
    \State $x_{t}^{q} \gets \querstrat(\poolsetx_{t-1})$
    \State $y_{t}^{q} \gets \oracle(x_{t}^{q})$
    \State $\trainset_{t} \gets \trainset_{t-1} \cup \{(x_{t}^{q}, y_{t}^{q})\}$
    \State $\poolset_{t} \gets \poolset_{t-1} \setminus \{(x_{t}^{q}, y_{t}^{q})\}$
    \State $\hat{h}_{t} \gets \algo(\trainset_{t})$
    \State $\mat{L}_{i, t} \gets \err{S_{test}}{\hat{h}_{t}}$ \Comment{Fill out loss matrix for current fold and step}
    \EndFor
    \EndFor
    \EndProcedure
  \end{algorithmic}
\end{algorithm}

\FloatBarrier

\subsection{Plots}
\begin{figure}[htb]
  \label{figs:agnostic-regression-learning-curves}
  \centering
  \begin{subfigure}[b]{0.48\textwidth}
    \label{fig:learning-curve-agnostic-bike-sharing}
    \includegraphics[width=\textwidth]{../active_learning_code/reports/figures/learning_curves_k_fold-bike_sharing_day.png}
    \caption{bike sharing (day)}
  \end{subfigure}
  \begin{subfigure}[b]{0.48\textwidth}
    \label{fig:learning-curve-agnostic-boston}
    \includegraphics[width=\textwidth]{../active_learning_code/reports/figures/learning_curves_k_fold-boston.png}
    \caption{boston}
  \end{subfigure}
  \hspace{1.0cm}
  \begin{subfigure}[b]{0.48\textwidth}
    \label{fig:learning-curve-agnostic-concrete}
    \includegraphics[width=\textwidth]{../active_learning_code/reports/figures/learning_curves_k_fold-concrete.png}
    \caption{concrete}
  \end{subfigure}
  \begin{subfigure}[b]{0.48\textwidth}
    \label{fig:learning-curve-agnostic-red_whine}
    \includegraphics[width=\textwidth]{../active_learning_code/reports/figures/learning_curves_k_fold-red_wine.png}
    \caption{red wine}
  \end{subfigure}
  \hspace{1.0cm}
  \begin{subfigure}[b]{0.48\textwidth}
    \label{fig:learning-curve-agnostic-white_wine}
    \includegraphics[width=\textwidth]{../active_learning_code/reports/figures/learning_curves_k_fold-white_wine.png}
    \caption{white wine}
  \end{subfigure}
  \caption{Learning curves (bold line is mean with shaded region being \(\pm 1\) standard deviation over the 5 folds) for agnostic regression comparing FW (KH) with MC and
    Levscore. The \(y\)-axis is MSE (Mean Squared Error), \(x\)-axis is \(t\)
    (number of datapoints in current active learning train set). A good
    algorithm will have a trajectory that goes down quickly with \(t\). MC is baseline.}
\end{figure}

\begin{figure}[htb]
  \label{figs:realisable-regression-learning-curves}
  \centering
  \begin{subfigure}[b]{0.48\textwidth}
    \label{fig:learning-curve-realisable-bike-sharing}
    \includegraphics[width=\textwidth]{../active_learning_code/reports/figures/learning_curves_k_fold_realisable-bike_sharing_day.png}
    \caption{bike sharing (day)}
  \end{subfigure}
  \begin{subfigure}[b]{0.48\textwidth}
    \label{fig:learning-curve-realisable-boston}
    \includegraphics[width=\textwidth]{../active_learning_code/reports/figures/learning_curves_k_fold_realisable-boston.png}
    \caption{boston}
  \end{subfigure}
  \hspace{1.0cm}
  \begin{subfigure}[b]{0.48\textwidth}
    \label{fig:learning-curve-realisable-concrete}
    \includegraphics[width=\textwidth]{../active_learning_code/reports/figures/learning_curves_k_fold_realisable-concrete.png}
    \caption{concrete}
  \end{subfigure}
  \begin{subfigure}[b]{0.48\textwidth}
    \label{fig:learning-curve-realisable-red_whine}
    \includegraphics[width=\textwidth]{../active_learning_code/reports/figures/learning_curves_k_fold_realisable-red_wine.png}
    \caption{red wine}
  \end{subfigure}
  \hspace{1.0cm}
  \begin{subfigure}[b]{0.48\textwidth}
    \label{fig:learning-curve-realisable-white_wine}
    \includegraphics[width=\textwidth]{../active_learning_code/reports/figures/learning_curves_k_fold_realisable-white_wine.png}
    \caption{white wine}
  \end{subfigure}
  \caption{Learning curves (bold line is mean with shaded region being \(\pm 1\) standard deviation over the 5 folds) for regression comparing FW (KH) with MC and
    Levscore. The \(y\)-axis is MSE (Mean Squared Error), \(x\)-axis is \(t\)
    (number of datapoints in current active learning train set). A good
    algorithm will have a trajectory that goes down quickly with \(t\). MC is baseline.}
\end{figure}

\begin{figure}[htb]
  \label{figs:agnostic-classification-learning-curves}
  \centering
  \begin{subfigure}[b]{0.48\textwidth}
    \label{fig:learning-curve-agnostic-mnist}
    \includegraphics[width=\textwidth]{../active_learning_code/reports/figures/learning_curves_k_fold-mnist.png}
    \caption{mnist}
  \end{subfigure}
  \begin{subfigure}[b]{0.48\textwidth}
    \label{fig:learning-curve-agnostic-yeast}
    \includegraphics[width=\textwidth]{../active_learning_code/reports/figures/learning_curves_k_fold-yeast.png}
    \caption{yeast}
  \end{subfigure}
  \caption{Learning curves (bold line is mean with shaded region being \(\pm 1\) standard deviation over the 5 folds) for agnostic classification comparing FW (KH) with MC and
    Levscore. The \(y\)-axis is Accuracy, \(x\)-axis is \(t\)
    (number of datapoints in current active learning train set). A good
    algorithm will have a trajectory that goes up quickly with \(t\). MC is baseline.}
\end{figure}

\FloatBarrier

\subsection{Datasets and Hyperparameters}

\begin{table}[H]
  \centering
  \csvautobooktabular[
  table head=\toprule Dataset & $n$ & $d$ & $\lambda_{opt}$ & $\sigma_{opt}$ \\\midrule
  ]{../active_learning_code/reports/hyperparams/agnostic_regression_hyperparams.csv}
  \vspace{0.5cm}
  \caption{Table for Agnostic Regression containing dataset information and hyperparameters.
    First column is the name of the dataset, \(n\) is the size, \(d\) the number of
    dimensions, \(\lambda_{opt}\) and \(\sigma_{opt}\) the hyperparameters chosen
    for KRR using kCV}
\end{table}

\vspace{2cm}
\FloatBarrier

\begin{table}[H]
  \centering
  \csvautobooktabular[
  table head=\toprule Dataset & $n$ & $d$ & $\lambda_{opt}$ & $\sigma_{opt}$ \\\midrule
  ]{../active_learning_code/reports/hyperparams/realisable_regression_hyperparams.csv}
  \vspace{0.5cm}
  \caption{Table for Realisable Regression containing dataset information and hyperparameters.
    First column is the name of the dataset, \(n\) is the size, \(d\) the number of
    dimensions, \(\lambda_{opt}\) and \(\sigma_{opt}\) the hyperparameters chosen
    for KRR using kCV}
\end{table}

\vspace{2cm}
\FloatBarrier

\begin{table}[H]
  \centering
  \csvautobooktabular[
  table head=\toprule Dataset & $n$ & $d$ & $\lambda_{opt}$ & $\sigma_{opt}$ \\\midrule
  ]{../active_learning_code/reports/hyperparams/agnostic_classification_hyperparams.csv}
  \vspace{0.5cm}
  \caption{Table for Agnostic Classification containing dataset information and hyperparameters.
    First column is the name of the dataset, \(n\) is the size, \(d\) the number of
    dimensions, \(\lambda_{opt}\) and \(\sigma_{opt}\) the hyperparameters chosen
    for KRR using kCV}
\end{table}


% Actually generates your bibliography. The fact that \include is 
% the last thing before this ensures that it is on a clear page.
\bibliography{references}

% All done. \o/
\end{document}
