% I may change the way this is done in a future version, 
%  but given that some people needed it, if you need a different degree title 
%  (e.g. Master of Science, Master in Science, Master of Arts, etc)
%  uncomment the following 3 lines and set as appropriate (this *has* to be before \maketitle)
% \makeatletter
% \renewcommand {\@degree@string} {Master of Things}
% \makeatother

\title{Active Learning Regression by Mitigating Domain Drift}
\author{John Isak Texas Falk}
\department{Department of Computer Science}

\maketitle
\makedeclaration

\begin{abstract} % 300 word limit
  % Motivation / Purpose
  Active learning enables an algorithm to choose a subset training set of a dataset to
  label while retaining most of the performance as if training on the full dataset.
  When labelling data is costly but finding instances cheap, this can lead to
  massive cost reduction over supervised learning.
  % Problem
  While active learning for classification has been explored both theoretically and practically, there
  is much less work done on regression.
  % Method
  In this work we focus on noiseless regression where we devise an algorithm to
  pick instances to label and train a Kernel Ridge Regression on this data.
  Central to this is the Frank Wolfe algorithm which greedily optimises the Maximum Mean
  Discrepancy between the full dataset and train set. This leads to an algorithm
  that optimises an upper bound on the empirical risk of the estimator trained
  on the built train set. We try to control the true generalisation error but
  current results from convergence of Frank Wolfe means we can't say whether or
  not this improves upon random sampling.
  % Results
  We apply the algorithm to a wide range of datasets and compare it to leverage
  score sampling. We show that the algorithm performs competitevely with other
  algorithms and the benefit of using active learning for regression. We also
  show that it dominates other methods in the agnostic setting of both
  regression and classification.
  % Conclusion
  We show that using Frank Wolfe to mitigate domain drift is competitive and
  works well in practice, with learning curves that show greater performance
  than random sampling on a wide range of datasets in both agnostic and
  realisable setting.
\end{abstract}

\begin{impactstatement}

% 	UCL theses now have to include an impact statement. \textit{(I think for REF reasons?)} The following text is the description from the guide linked from the formatting and submission website of what that involves. (Link to the guide: {\scriptsize \url{http://www.grad.ucl.ac.uk/essinfo/docs/Impact-Statement-Guidance-Notes-for-Research-Students-and-Supervisors.pdf}})

% \begin{quote}
% The statement should describe, in no more than 500 words, how the expertise, knowledge, analysis,
% discovery or insight presented in your thesis could be put to a beneficial use. Consider benefits both
% inside and outside academia and the ways in which these benefits could be brought about.

% The benefits inside academia could be to the discipline and future scholarship, research methods or
% methodology, the curriculum; they might be within your research area and potentially within other
% research areas.

% The benefits outside academia could occur to commercial activity, social enterprise, professional
% practice, clinical use, public health, public policy design, public service delivery, laws, public
% discourse, culture, the quality of the environment or quality of life.

% The impact could occur locally, regionally, nationally or internationally, to individuals, communities or
% organisations and could be immediate or occur incrementally, in the context of a broader field of
% research, over many years, decades or longer.

% Impact could be brought about through disseminating outputs (either in scholarly journals or
% elsewhere such as specialist or mainstream media), education, public engagement, translational
% research, commercial and social enterprise activity, engaging with public policy makers and public
% service delivery practitioners, influencing ministers, collaborating with academics and non-academics
% etc.

% Further information including a searchable list of hundreds of examples of UCL impact outside of
% academia please see \url{https://www.ucl.ac.uk/impact/}. For thousands more examples, please see
% \url{http://results.ref.ac.uk/Results/SelectUoa}.
% \end{quote}

The knowledge and algorithms in this dissertation have the potential to have
industrial use in any setting where labelling is costly, or when there is a
fixed budget of how many instances it is possible to label. This is a setting
which occur in many different areas, to name a few, medical healthcare, process
engineering, natural language processing and data mining of documents. As
this contribution is derived from theoretical considerations, it can be used
both locally and internationally, by anyone anywhere.

From an academic point of view, the information and knowledge in this
dissertation will act as a springboard for future researchers and students, and
add to the existing pool of knowledge in the field of machine learning. The hope
it that this will foster further ideas on how to improve the field of active
learning for regression, building on results derived here and extending it to
more general and different settings.
\end{impactstatement}

\begin{acknowledgements}
  To my parents, thank you.
  \newline
  \newline
  To Carlo and Massi, Cheers!
\end{acknowledgements}

\setcounter{tocdepth}{2} 
% Setting this higher means you get contents entries for
%  more minor section headers.

\tableofcontents
\listoffigures
\listoftables

