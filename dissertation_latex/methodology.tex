\chapter{Methodology}
\label{ch:methodology}
In this chapter we propose an active learning algorithm based on the bound of
\cite{viering17_nuclear_discr_activ_learn} and use FW to optimise this bound. In
particular we use the tools laid out in the \hyperref[ch:lit-review]{Literature
  Review} in order to attempt to derive bounds on the excess active learning
risk. We show that FW gives us guarantees of reducing the empiricial risk but
that it's inconclusive given current results on rate of convergence if it
improves on the generalisation error compared to random sampling.

\section{Setting}
The setting largely follows that of \cite{viering17_nuclear_discr_activ_learn} and
\cite{beck04_condit_gradien_method_with_linear} as we fuse these two works and
lean on results of both. We first specify all of our assumptions for the bound.

\begin{assumption}
  \label{as:al-fw}
  We assume that \(\X \subseteq \R^{D}\) for some \(D \in \N\) and \(\Y \subseteq
  \R\). We assume there exists a deterministic labeling function \(f : \X \to
  \Y\) and that there is some unknown distribution \(\rho\) on \(\X \times \Y\).
  Furthermore, we assume that there is no output noise such that \(\rho(x, y) = \rho_{\X}(x)\mathbf{1}(f(x) =
  y)\). At the start we get a pool of \(\iid\) samples from \(\rho^{n}\), which we
  call \(\poolset_{0} = (x_{i}, y_{i})_{i=1}^{n}\), and we start with an empty train
  set \(\trainset_{0}\).
\end{assumption}
We furthermore make the following assumptions
\begin{assumption}
  \label{as:domain-is-compact}
  We assume that \(\X\) is compact.
\end{assumption}
\begin{assumption}
  We assume that the output it bounded in magnitude,
  \label{as:output-bounded}
  \begin{equation*}
    \sup_{y \in \mathcal{Y}}|y| = M_y < \infty.
  \end{equation*}
\end{assumption}
\begin{assumption}
  \label{as:f-in-gamma-ball}
  The function \(f \in B_{\gamma}\) where \(B_{\gamma} \coloneqq B_{\gamma}(0) \subseteq \rkhs{H}\) where \(\rkhs{H}\) is an RKHS with kernel function \(K\).
\end{assumption}
\begin{assumption}
  \label{as:realisable-setting}
  We are in realisable setting, our algorithm \(\algo\) is such that \(f \in \ran(\algo)\).
\end{assumption}

We will use the squared loss \(\ell(y, y') = (y - y')^{2}\). We will consider
the hypothesis space to be a ball in an RKHS \(\rkhs{H}\) with reproducing
kernel \(K\), where the radius \(\gamma < \gamma'\) so that \(f\) is in this
hypothesis space. We introduce an auxiliary RKHS \(\rkhs{H}'\) which is such
that the reproducing kernel \(K'\) is \(K'(x, x') = K(x, x')^{2}\), which will
be associated with the functional space used for MMD calculations and in
extension KH and FW.

We make the following assumptions, \(H' = B_{\eta} \subseteq \rkhs{H}'\), and
from \ref{as:domain-is-compact}, if we assume that \(K, K'\) are continuous on
\(\X \times \X\) we have that the norm in RKHS feature mappings, \(\phi(x)\) are bounded in the RKHS for both
\(\rkhs{H}, \rkhs{H}'\) and we denote \(\kappa = \sup_{x \in \mathcal{X}}\sqrt{K(x, x)} =
\sup_{x \in \mathcal{X}}\norm{\phi(x)}_{\rkhs{H}} < \infty\), and \(\kappa' = \sup_{x \in \mathcal{X}}\sqrt{K'(x, x)} < \infty\).

\section{MMD empirical generalisation bound}
We now state results from \cite{viering17_nuclear_discr_activ_learn} that will be
used.
\begin{theorem}[\cite{viering17_nuclear_discr_activ_learn}, Extension: Theorem 1]
  \label{th:mmd-emp-bound}
  Let \(\ell\) be any loss, and let \(g_{f, h}(x) = \ell(h(x), f(x))\), then for
  any function \(h \in H \subseteq \rkhs{H}\), \(H\) being an arbitrary subset of \(\rkhs{H}\), any train set
  \(\trainset \subseteq \poolset_{0}\) and let \(\rho_{\trainsetx}\) be any distribution with
  support on \(\trainsetx\) with \(\rho_{S}\) the corresponding distribution on
  \((x, f(x))_{x \in \trainset}\), then for any \(H' \subseteq \rkhs{H}'\)
  \begin{equation}
    \label{eq:MMD-gen-bound}
    \err{\poolset_{0}}{h} \leq \err{\rho_{\trainset}}{h} + \MMD{\poolsetx_{0}}{\trainsetx}{H'} + \eta_{MMD}
  \end{equation}
\end{theorem}
where \(\eta_{MMD} = 2 \min_{\tilde{g} \in H'} \max_{h \in H, x \in \poolsetx_{0}} |g_{f, h}(x) - \tilde{g}(x)|\).

\begin{proof}
  This is essentially \cite{viering17_nuclear_discr_activ_learn}[Proof of Theorem 1].
  Let \(g_{f, h}(x) = \ell(f(x), h(x))\) and let \(\tilde{g}\) be any function in
  \(H'\).

  Consider bounding the following quantity
  \begin{equation*}
    \abs{\err{\poolset_0}{h} - \err{\rho_{\trainset}}{h}}.
  \end{equation*}
  We decompose this as follows, letting \(w_{j} = \rho_{\trainsetx}(x_j)\) where
  \(x_j \in \trainsetx\) and \(n = |\poolset_0|, m = |\trainset|\),
  \begin{align*}
    \abs{\err{\poolset_0}{h} - \err{\rho_{\trainset}}{h}} & = \abs{\err{\poolset_0}{h} - \frac{1}{n} \sum_{i=1}^n\tilde{g}(x_i) + \frac{1}{n} \sum_{i=1}^n\tilde{g}(x_i) - \sum_{j=1}^m w_j \tilde{g}(x_j) + \sum_{j=1}^m w_j \tilde{g}(x_j) - \err{\rho_{\trainset}}{h}} \\
                                                          & \leq \underbrace{\abs{\err{\poolset_0}{h} - \frac{1}{n} \sum_{i=1}^n\tilde{g}(x_i)}}_{(a)} + \underbrace{\abs{\frac{1}{n} \sum_{i=1}^n\tilde{g}(x_i) - \sum_{j=1}^m w_j \tilde{g}(x_j)}}_{(b)} + \underbrace{\abs{\err{\rho_{\trainset}}{h} - \sum_{j=1}^m w_j \tilde{g}(x_j)}}_{(c)}.
  \end{align*}

  For \((a)\) we can bound this as
  \begin{equation*}
    \abs{\err{\poolset_0}{h} + \frac{1}{n} \sum_{i=1}^n\tilde{g}(x_i)} \leq \frac{1}{n} \sum_{i=1}^{n}\abs{g_{h, f}(x_{i}) - \tilde{g}(x_{i})} \leq \max_{x \in \poolsetx_0}\abs{g_{h, f}(x) - \tilde{g}(x)},
  \end{equation*}
  and for \((c)\) we can similarly bound this as
  \begin{align*}
    \abs{\err{\rho_{\trainset}}{h} - \sum_{j=1}^m w_j \tilde{g}(x_j)} & \leq \sum_{j=1}^{m} w_{j}\abs{g_{h, f}(x_{j}) - \tilde{g}(x_{j})} \leq \sum_{j=1}^{m}w_{j}\max_{x \in \trainsetx}\abs{g_{h, f}(x) - \tilde{g}(x)} \\
                                                                      & = \max_{x \in \trainsetx}\abs{g_{h, f}(x) - \tilde{g}(x)} \leq \max_{x \in \poolsetx_0}\abs{g_{h, f}(x) - \tilde{g}(x)},
  \end{align*}
  which follows from the fact that \(\trainsetx \subseteq \poolsetx_{0}\).

  For \((b)\) we can bound this as follows
  \begin{equation*}
    \abs{\frac{1}{n} \sum_{i=1}^n\tilde{g}(x_i) - \sum_{j=1}^m w_j \tilde{g}(x_j)} \leq \sup_{\tilde{g} \in H'}\abs{\frac{1}{n} \sum_{i=1}^n \tilde{g}(x_i) - \sum_{j=1}^m w_j \tilde{g}(x_j)} = \MMD{\rho_{\poolsetx_0}}{\rho_{\trainsetx}}{H'},
  \end{equation*}
  and altogether we have that
  \begin{align*}
    \abs{\err{\poolset_0}{h} - \err{\rho_{\trainset}}{h}} & \leq \MMD{\rho_{\poolsetx_0}}{\rho_{\trainsetx}}{H'} + 2\max_{x \in \poolsetx_0}\abs{g_{h, f}(x) - \tilde{g}(x)} \\
                                                          & \leq \MMD{\rho_{\poolsetx_0}}{\rho_{\trainsetx}}{H'} + 2\min_{\tilde{g} \in H'}\max_{h \in H, x \in \poolsetx_0}\abs{g_{h, f}(x) - \tilde{g}(x)}
  \end{align*}.

  Setting \(\rho_{\trainsetx} = \frac{1}{m}\sum_{j=1}^{m}\delta_{x_{j}}\) recovers the original proof.
\end{proof}

We note that \ref{th:mmd-emp-bound} works for empirical distributions, but also
reweighted distributions. This is useful as it allows us to use FW algorithms
that outputs a distribution over \(\trainsetx_{t}\) that is non-uniform, for
example when using FW line search which gives a weighted distribution.

We have the following result
\begin{theorem}[\cite{viering17_nuclear_discr_activ_learn}, Theorem 2]
  \label{thm:zero_eta_MMD}
  Let \(\ell\) be the squared loss and assume \(f \in B_{\gamma}\). Let \(K, K'\) be the
  kernel functions of the RKHS's \(\rkhs{H}, \rkhs{H}'\) respectively. If \(K'(x, x') =
  K(x, x')^2\) and \(\eta \geq 4\gamma^2\), then it is guaranteed that \(g(\cdot) =
  \ell(h(\cdot), f(\cdot)) \in H'\) and thus \(\eta_{MMD} = 0\).
\end{theorem}
When using a Gaussian kernel, this reduces to the following corollary
\begin{corollary}
  \label{cor:gauss-kernel-squared-gives-eta-zero}
  Let \(f \in B_{\gamma}\) and let \(K(x, x') = \exp(- \frac{\|x - x'\|^2}{2\sigma^2})\), the gaussian kernel with
  bandwidth \(\sigma\). If \(K'(x, x') = \exp(- \frac{\|x - x'\|^2}{2\sigma'^2})\) with bandwidth \(\sigma' =
  \frac{\sigma}{\sqrt{2}}\) and \(\eta = 4\gamma^2\) then \(\eta_{MMD} = 0\).
\end{corollary}

\begin{proof}
  A sufficient condition given by \ref{thm:zero_eta_MMD} is that if \(K'(x, x') =
  K(x, x')^2\) and \(\eta = 4\gamma^2\) then \(\eta_{MMD} = 0\). As we are using
  gaussian kernels, this means that
  \begin{equation*}
    \exp(- \frac{\|x - x'\|^2}{2\sigma'^2}) = \exp(- \frac{\|x - x'\|^2}{2\sigma^2})^2 = \exp(- \frac{\|x - x'\|^2}{\sigma^2})
  \end{equation*}
  which means that \(\sigma' = \frac{\sigma}{\sqrt{2}}\).
\end{proof}

For \(\eta_{MMD}\) to be zero, we need the output of our algorithm to be
contained in the ball \(B_{\gamma}\). While KRR and Ivanov Regression is linked
as KRR solves a lagrangian formulation of the Ivanov Regression problem, we need
to make sure that the output \(\hat{h} = \algo(S)\) when doing KRR is such that
\(\norm{\hat{h}}_{\rkhs{H}} \leq \gamma\). Following the proof of \cite[Lemma
1]{cortes14_domain_adapt_sampl_bias_correc} we have the following
\begin{lemma}
\label{lem:KRR-output-ball-bound} The output of the KRR algorithm when applied
to \(S = (x_{i}, y_{i})_{i=1}^{n}\) is such that for \(\hat{h} = \algo(S)\) we have
\(\norm{\hat{h}}_{\rkhs{H}} \leq \frac{2 M_{y}}{\gamma}\) and if \(\lambda =
\frac{2 M_y \kappa}{\gamma}\) then the hypothesis space is contained within
\(B_{\gamma}\).
\end{lemma}
\begin{proof} Since \(\hat{h} = \algo(S)\) and this is the minimiser of the KRR
objective, we see that
  \begin{equation*} \err{S}{\hat{h}} + \lambda \norm{\hat{h}}_{\rkhs{H}}^2 \leq
\err{S}{0}
  \end{equation*} and we can then rearrange this by
  \begin{align*} \lambda \norm{\hat{h}}_{\rkhs{H}}^2 & = \err{S}{0} -
\err{S}{\hat{h}} \\ & = \frac{1}{n} \sum_{i=1}^n (y_i^2 - (\hat{h}(x_i) -
y_i)^2) \\ & = \frac{1}{n} \sum_{i=1}^n (2 y_i \hat{h}(x_i) - \hat{h}(x_i)^2) \\
& \leq \frac{2}{n} \sum_{i=1}^n y_i \hat{h}(x_i) \\ & \leq \frac{2}{n}
\sum_{i=1}^n M_y \norm{\hat{h}}_{\rkhs{H}} \kappa \\ & = 2 M_y \kappa
\norm{\hat{h}}_{\rkhs{H}}.
  \end{align*}

  Assume that \(\hat{h} \neq 0\), otherwise this is an equality, then we can
cancel out the norm showing that
  \begin{equation}
    \label{eq:KRR-output-ball-bound} \norm{\hat{h}}_{\rkhs{H}} \leq \frac{2 M_y
\kappa}{\lambda}.
  \end{equation} Setting \(\gamma = \frac{2 M_y \kappa}{\lambda}\), we see that
if \(\lambda = \frac{2 M_y \kappa}{\gamma}\) then the output of KRR will be in
\(B_{\gamma}\).
\end{proof}

Lemma \ref{lem:KRR-output-ball-bound} assures us that ifo we pick \(\lambda\)
large enough, then the output will be contained within the ball \(B_{\gamma}\)
so that \ref{cor:gauss-kernel-squared-gives-eta-zero} holds and we can assume
\(\eta_{MMD} = 0\).


\section{Frank-Wolfe Active Learning algorithm}
Considering the bound of the empirical risk over the sampled set
\(\poolset_{0}\) in \ref{th:mmd-emp-bound} we see that there are three terms to
control, the bias term \(\err{\rho_{\trainset_{t}}}{h}\), the term representing
domain drift, \(\MMD{\poolsetx_{0}}{\trainsetx_{t}}{H'}\), and the leftover term
\(\eta_{MMD}\) which becomes \(0\) according to
\ref{cor:gauss-kernel-squared-gives-eta-zero} by setting \(H' = B_{4\gamma^{2}}\)
and \(\sigma' = \frac{\sigma}{\sqrt{2}}\). 

We focus on the domain drift term and propose to optimise this using FW. Outside
of the degenerate case this enables us to do active learning in an online
fashion by running the FW algorithm on
\(\MMD{\poolsetx_{0}}{\trainsetx_{t}}{H'}\) which will give a convergence of
\(O(\frac{1}{t})\) compared to \(O(\frac{1}{\sqrt{t}})\) if we were to sample
points randomly.

We now state our first active learning algorithm which uses the active learning
tuple \((\algo, \querstrat)\) where \(\algo\) is KRR \ref{eq:krr-equation} with regularisation
parameter \(\lambda\) and \(\querstrat\) is the querying strategy that runs FW on
\(\poolsetx_{0}\) and chooses the point \(x_{t}\) corresponding to
\(\tilde{g}_{t}\) in \ref{al:fw-update-step-1}. The algorithm follows the recipy of
\ref{alg:active-learning} and we call this FW-KRR for Frank-Wolfe Kernel Ridge
Regression.

Note that in general we can create ad-hoc active learning algorithms by letting
\(\querstrat\) being FW and \(\algo\) any algorithm. In this way we can create a
family of FW-based active learning algorithms. We introduce the following active
learning algorithm for classification (which we will call FW-KRRC), let
\(\querstrat\) be FW and let \(\algo\) be the KRR applied to classification
through application of the method expanded upon in \cite{ciliberto16}.

\section{Attempting to Bound the Generalisation Error}
With inspiration of error decomposition of the generalisation error in
supervised learning we try to bound the following quantity
\err{\rho}{\hat{h}} - \err{\rho}{f}
where as before \(\hat{h} = \algo(\trainset)\). We work with
Ivanov regularisation instead of Kernel Ridge Regression, where the algorithm is
the following
\begin{equation}
  \label{eq:eq:ivanov-kr}
  \algo(\trainset) = \argmin_{h \in B_{\gamma} \subseteq \rkhs{H}} \frac{1}{m}\sum_{j=1}^{m}(h(x_{j}) - y_{j})^{2},
\end{equation}
as this can easily be extended to the KRR case.

\subsection{Error decomposition}
\label{sec:error-decomposition}
We decompose the error as follows
\begin{align*}
  \err{\rho}{\hat{h}} - \err{\rho}{f} & = \underbrace{\err{\rho}{\hat{h}} - \err{\trainset}{\hat{h}}}_{(a)} + \underbrace{\err{\trainset}{\hat{h}} - \err{\trainset}{h_{\gamma}}}_{\leq 0} + \underbrace{\err{\trainset}{h_{\gamma}} - \err{\rho}{h_{\gamma}}}_{(b)} + \underbrace{\err{\rho}{h_{\gamma}} - \err{\rho}{f}}_{\text{Approximation error}},
\end{align*}
where \(h_{\gamma} = \argmin_{h \in \mathcal{H}_{\gamma}} \err{}{h}\) and
\(\hat{h} = \algo(\trainset)\).

We then try to control each term, first we have that \((a)\) can be decomposed
as 
\begin{equation*}
  \err{\rho}{\hat{h}} - \err{\trainset}{\hat{h}} = \underbrace{\err{\rho}{\hat{h}} - \err{\poolset_0}{\hat{h}}}_{\text{Generalization error}} + \underbrace{\err{\poolset_0}{\hat{h}} - \err{\trainset}{\hat{h}}}_{\text{Empirical drift generalization error}},
\end{equation*}

where we have called the second quantity \emph{Empirical drift generalisation error} to highlight the fact that we are training on \(\trainset\) which is a
biased sample drifting from \(\poolset_0\).

Secondly, we can decompose \((b)\) as follows,
\begin{equation*}
  \err{\trainset}{h_{\gamma}} - \err{\rho}{h_{\gamma}} = \underbrace{\err{\trainset}{h_{\gamma}} - \err{\poolset_0}{h_{\gamma}}}_{\text{Empirical drift generalization error}} +\underbrace{\err{\poolset_0}{h_{\gamma}} - \err{\rho}{h_{\gamma}}}_{\text{Generalization error}}.
\end{equation*}

We now investigate each of these expressions in turn. We start with the
generalisation error for both.

Assume that \(h \in B_{\gamma}\) is any function and consider the following
\begin{align*}
  \abs{\err{\poolset_0}{h} - \err{\rho}{h}} & = \abs{\frac{1}{n} \sum_{i=1}^{n}(h(x_i) - y_i)^2 - \E_{\rho}[(h(x) - y)^2]} \\
                                            & = \abs{\frac{1}{n} \sum_{i=1}(h(x_i)^2 - 2h(x_i)y_i + y_i^2) - \E_{\rho}[h(x)^2 - 2h(x)y + y^2]} \\
                                            & = \abs{\frac{1}{n} \sum_{i=1}^nh(x_i)^2 - \E_{\rho_{\X}}[h(x)^2] + 2(\E_{\rho}[h(x)y] - \frac{1}{n} \sum_{i=1}^n h(x_i)y_i) + \frac{1}{n} \sum_{i=1}^n y_i^2 - \E_{\rho_{\Y}}[y^2]} \\
                                            & \leq \underbrace{\abs{\frac{1}{n} \sum_{i=1}^nh(x_i)^2 - \E_{\rho_{\X}}[h(x)^2]}}_{(i)} + 2 \underbrace{\abs{\frac{1}{n} \sum_{i=1}^n h(x_i)y_i - \E_{\rho}[h(x)y]}}_{(ii)} + \underbrace{\abs{\frac{1}{n} \sum_{i=1}^n y_i^2 - \E_{\rho_{\Y}}[y^2]}}_{(iii)}.
\end{align*}
Using the reproducing property of \ref{def:reproducing-kernel} we can express
\(f(x) = \scal{f}{K_{x}}_{\rkhs{H}}\) for any \(x \in \X\). Consider each term
individually

\begin{description}
\item[{(i)}] 
\end{description}
\begin{align*}
  \abs{\frac{1}{n} \sum_{i=1}^nh(x_i)^2 - \E_{\rho_{\X}}[h(x)^2]} & = \abs{\frac{1}{n} \sum_{i=1}^n \scal{h}{\phi(x_i)}_{\rkhs{H}}^2  - \E_{\rho_{\X}}[\scal{ h}{ \phi(x) }_{\rkhs{H}}^2]} \\
                                                                  & = \abs{\frac{1}{n} \sum_{i=1}^n \scal{h}{\scal{h}{\phi(x_i)}_{\rkhs{H}} \phi(x_i)}_{\rkhs{H}}  - \E_{\rho_{\X}}[\scal{h}{\scal{h}{\phi(x)}_{\rkhs{H}}\phi(x)}_{\rkhs{H}}]} \\
                                                                  & = \abs{\frac{1}{n} \sum_{i=1}^n \scal{h}{(\phi(x_i) \otimes \phi(x_i)) h}_{\rkhs{H}} - \E_{\rho_{\X}}[\scal{h}{(\phi(x) \otimes \phi(x)) h}_{\rkhs{H}}]} \\
                                                                  & = \abs{\scal{h}{\frac{1}{n} \sum_{i=1}^n(\phi(x_i) \otimes \phi(x_i)) h}_{\rkhs{H}} - \scal{h}{\E_{\rho_{\X}}[(\phi(x) \otimes \phi(x))] h}_{\rkhs{H}}} \\
                                                                  & = \abs{\scal{h}{(\hat{C}_{\poolsetx_{0}} - C_{\rho_{\X}})h}_{\rkhs{H}}} \\
                                                                  & = \norm{h}_{\rkhs{H}}^2 \norm{\hat{C}_{\poolsetx_{0}} - C_{\rho_{\X}}}_{op},
\end{align*}         
where \(C_{\rho_{\X}}, \hat{C}_{\poolsetx_{0}}\), are the covariate and
empirical covariate operator in \(\rkhs{H}\) for \(\rho_{\X}\).
Since \(\hat{C}_{\poolsetx_{0}}, C_{\rho_{\X}} \in HS(\rkhs{H})\), which follows
as \(\kappa < \infty\), the operator norm can be bounded
by the HS (Frobenius) norm
\begin{equation}
  \label{eq:op_leq_HS}
  \norm{\hat{C}_{\poolsetx_{0}} - C_{\rho_{\X}}}_{op} \leq \norm{\hat{C}_{\poolsetx_{0}} - C_{\rho_{\X}}}_{HS}.
\end{equation}

We will now use the following lemma
\begin{lemma}[\cite{zwald06}, Lemma 1]
  \label{lem:bound-cov-operator-rkhs-hs}
  Suppose that \(\kappa^{2} < M\). Given any \(\rho_{\X}\) let \(\{x_{i}\}_{i=1}^{n}\) be a set sampled \(\iid\) from \(\rho_{\X}\). If \(C,
  \hat{C}_{n}\) are the true and empirical covariance operators, then with
  probability greater than \(1 - \delta\),
  \begin{equation}
    \label{eq:bound-cov-operator-rkhs-hs}
    \norm{\hat{C}_{n} - C}_{HS} \leq \frac{2M}{\sqrt{n}}\left( 1 + \sqrt{\frac{\log(1/\delta)}{2}} \right).
  \end{equation}
\end{lemma}
using \ref{lem:bound-cov-operator-rkhs-hs} together with \ref{eq:op_leq_HS} we have
\begin{corollary}
  \label{cor:bound-cov-op-diff-in-prob}
  With probability greater than \(1 - \delta\),
  \begin{equation}
    \label{eq:bound-cov-op-diff-in-prob}
    \norm{\hat{C}_{\poolsetx_{0}} - C_{\rho_{\X}}}_{op} \leq \frac{2\kappa^{2}}{\sqrt{n}}\left( 1 + \sqrt{\frac{\log(1/\delta)}{2}} \right)
  \end{equation}
\end{corollary}

\begin{description}
\item[{(ii)}] 
\end{description}
\begin{align*}
  \abs{\frac{1}{n} \sum_{i=1}^n h(x_i)y_i - \E_{\rho}[h(x)y]} & = \abs{\frac{1}{n} \sum_{i=1}^n \scal{ h}{ \phi(x_i) } y_i - \E_{\rho}[\scal{ h}{ \phi(x) } y]} \\
                                                              & = \abs{\scal{ h}{ \frac{1}{n} \sum_{i=1}^n \phi(x_i) y_i - \E_{\rho}[\phi(x) y] }}
\end{align*}

Let \(z_{\poolset_{0}} = \frac{1}{n} \sum_{i=1}^n \phi(x_i) y_i\) and
\(z_{\rho} = \E_{\rho}[\phi(x) y]\). Applying the CS inequality, this leads to
\begin{equation*}
  \abs{\scal{h}{\frac{1}{n} \sum_{i=1}^n \phi(x_i) y_i - \E_{\rho}[\phi(x) y] }} = \norm{h}_{\rkhs{H}}\norm{z_{\poolset_0} - z_{\rho}}_{\rkhs{H}}
\end{equation*}

The following lemma will let us control the norm of \(z_{\poolset_{0}} - z_{\rho}\),
\begin{lemma}[\cite{smale07_learn_theor_estim_via_integ}, Lemma 2]
  \label{lem:RKHS_hoeffding}
  let \(\rkhs{H}\) be a Hilbert space and let \(\xi\) be a random variable
  on \((\X \times \Y, \rho)\), with values in \(\rkhs{H}\). Assume
  \(\norm{\xi}_{\rkhs{H}} \leq M < \infty\) almost surely. Denote
  \(\sigma^2(\xi) = \mathbb{E}[\norm{\xi}_{\rkhs{H}}^2]\). Let
  \(\{z_i\}_{i=1}^n\) be independent random draws of \(\rho\). For any \(0 <
  \delta < 1\), with probability \(1 - \delta\) over the draws,
  \begin{equation}
    \label{eq:RKHS_hoeffding_bound}
    \norm{\frac{1}{n}\sum_{i=1}^n\xi(z_i) - \E[\xi(z)]}_{\rkhs{H}} \leq \frac{2 M \log(2/\delta)}{n} + \sqrt{\frac{2 \sigma^2(\xi) \log(2/\delta)}{n}}
  \end{equation}
\end{lemma}

We will use \ref{lem:RKHS_hoeffding} in  order to bound the quantity
\(\norm{z_{\trainset_0} - z_{\rho}}_{\rkhs{H}}\) in probability. 
By assumption \ref{as:output-bounded} \(\sup_{y \in \mathcal{Y}}|y| = M_y <
\infty\) which means that \(z_{i} = y_{i} \phi(x_{i})\) will be an element of
\(\rkhs{H}\). These \(z\)'s will correspond to the \(\xi\)'s of lemma \ref{lem:RKHS_hoeffding}. 
We can bound them through
\begin{align*}
  \norm{z}_{\rkhs{H}} & \leq \norm{y \phi(x)}_{\rkhs{H}} \\
                      & \leq \norm{y}\norm{\phi(x)}_{\rkhs{H}} \\
                      & \leq M_y\sqrt{\scal{\phi(x)}{\phi(x)}_{\rkhs{H}}} \\
                      & \leq M_y\sqrt{K(x, x)} \\
                      & \leq M_y \sup_{x \in \X} \sqrt{K(x, x)} \\
                      & \leq M_y \kappa,
\end{align*}
hence we can identify \(M\) in the lemma with \(M_{y}\kappa\).

We also have that
\begin{align*}
  \sigma^2 & = \E[\norm{y \phi(x)}_{\rkhs{H}}^2] \\
           & = \E[\abs{y}^2 \scal{\phi(x)}{\phi(x)}_{\rkhs{H}}] \\
           & = \E[\abs{y}^2 K(x, x)] \\
           & \leq M^2_y \kappa^2 \\
           & < \infty.
\end{align*}
Finally as \(\poolset_{0}\) is a set of input-output pairs sampled \(\iid\) from \(\rho\)
we have that the \(z_{i} = y_{i}\phi(x_{i})\) are sampled \(\iid\) as well. Thus
all conditions of the lemma \ref{lem:RKHS_hoeffding} are satisfied.

Thus we have that for any \(\delta \in (0, 1)\), with  probability \(1 -
\delta\) the following holds
\begin{equation}
  \label{eq:zs_RKHS_hoeffding}
  \norm{z_{\poolset_0} - z_{\rho}}_{\rkhs{H}} \leq \frac{2 M_y \kappa \log(2/\delta)}{n} + \sqrt{\frac{2 M_y^2 \kappa^2 \log(2/\delta)}{n}} = M_y \kappa \left(\frac{2\log(2/\delta)}{n} + \sqrt{\frac{2\log(2/\delta)}{n}} \right).
\end{equation}


\begin{description}
\item[{(iii)}] 
\end{description}
\begin{equation*}
  \abs{\frac{1}{n} \sum_{i=1}^n y_i^2 - \E_{\rho_{\Y}}[y^2]}
\end{equation*}

We can bound this using assumption \ref{as:output-bounded} which says that any
\(Y\) is such that \(y \leq M_y\). First we introduce Hoeffding's inequality,
\begin{theorem}[\cite{ciliberto18_advan_topic_machin_learn}, Hoeffdings Inequality]
  \label{th:univariate-hoeffdings-ineq}
  Let \(X_{1}, \dots, X_{n}\) be independent random variables such that \(X_{i}
  \in [a_{i}, b_{i}]\) with probability 1. Let \(\overline{X} = \frac{1}{n}
  \sum_{i=1}^n X_i\). Then
  \begin{equation}
    \label{eq:univariate-hoeffdings-ineq}
    \Pr(\abs{\overline{X} - \E[\overline{X}]} \geq \epsilon) \leq 2 \exp(-\frac{2n^2\epsilon^2}{\sum_{i=1}^n(b_i - a_i)^2}).
  \end{equation}
\end{theorem}
Applying \ref{th:univariate-hoeffdings-ineq} to the random variables
\(\{y_{i}\}_{i=1}^{n}\) which are sampled \(\iid\) and using that \(\abs{y}^{2}
\leq M_{y}^{2}\) we have that for any \(\delta \in (0, 1)\), with probability at
least \(1 - \delta\) over the samples the following holds
\begin{equation}
  \label{eq:y2s_hoeffding}
  \abs{\frac{1}{n} \sum_{i=1}^n y_i^2 - \E_{\rho_{y}}[y^2]} \leq \sqrt{\frac{M_y^4 \log(2/\delta)}{2n}} = M_y^2\sqrt{\frac{\log(2/\delta)}{2n}}.
\end{equation}

Combining \((i), (ii)\) and \((iii)\) we proceed to bound this uniformly using
the assumption that \(h \in B_{\gamma}\),
\begin{align*}
  \sup_{\norm{h}_{\rkhs{H}} \leq \gamma} \abs{\err{\poolset_0}{h} - \err{\rho}{h}} & \leq \sup_{\norm{h}_{\rkhs{H}} \leq \gamma}
                                                                                     \norm{h}_{\rkhs{H}}^2\norm{\hat{C}_{\trainsetx_0} - C_{\rho_{\X}}}_{op} +
                                                                                     \norm{h}_{\rkhs{H}} \norm{z_{\trainsetx_0} - z_{\rho}}_{\rkhs{H}} + \abs{\frac{1}{n}
                                                                                     \sum_{i=1}^n y_i^2 - \E_{\rho_{\Y}}[y^2]} \\
                                                                                   & = \gamma^2 \norm{\hat{C}_{\trainsetx_0}
                                                                                     - C_{\rho_{\X}}}_{op} + \gamma \norm{z_{\trainsetx_0} - z_{\rho}}_{\rkhs{H}} +
                                                                                     \abs{\frac{1}{n} \sum_{i=1}^n y_i^2 - \E_{\rho(y)}[y^2]}.
\end{align*}
Combining this with results
\ref{eq:bound-cov-op-diff-in-prob}, \ref{eq:zs_RKHS_hoeffding}, \ref{eq:y2s_hoeffding} we have that with probability \(1 - 3\delta\) over \(\poolset_0\) we have that
\begin{equation}
  \label{eq:uniform-bound-gen-error-over-ball}
  \sup_{\|h\|_{\rkhs{H}} \leq \gamma} \abs{\err{\poolset_0}{h} -\err{}{h}} \leq \frac{2 \kappa^2 \gamma^2}{\sqrt{n}}  \left(1 + \sqrt{\frac{\log \frac{1}{\delta}}{2}} \right) + \frac{2 M_y \gamma \kappa \log(2/\delta)}{n} + \sqrt{\frac{2 M_y^2 \kappa^2 \log(2/\delta)}{n}} + M_y^2\sqrt{\frac{\log(2/\delta)}{2n}}.
\end{equation}

Since \(\hat{h}, h_{\gamma} \in B_{\gamma}\) we have bounded both of the
generalisation terms in \((a), (b)\).

\subsection{Bounding in probability}
From \ref{eq:MMD-gen-bound} we have that for \(S \subseteq P_{0}\) and
letting \(\rho_{\trainsetx} = \frac{1}{m} \sum_{x \in \trainsetx}\delta_{x}\)
, both of the empirical drift generalisation errors in \((a), (b)\) can be
bounded above by \(\MMD{\poolsetx_{0}}{\trainsetx}{B_{\eta}} + \eta_{MMD}\),
and from \ref{cor:gauss-kernel-squared-gives-eta-zero} if we let \(\eta =
4\gamma^{2}\) and assume that \(f \in B_{\gamma}\) then \(\eta_{MMD} = 0\). We will choose \(K'\) such that
\ref{cor:gauss-kernel-squared-gives-eta-zero} holds.

Combining this we have that
\begin{theorem}
  \label{thm:al-non-generalisation-bound}
  Given that \(f \in B_{\gamma}\), \(\eta = 4\gamma^{2}\) and for any \(x, x' \in
  \X\), \(K'(x, x') = K(x, x')^2\) the following bound holds for \(\poolset_{0} \sim \rho^{n}\) and any \(S \subset \poolset_0\)
  \begin{equation}
    \label{eq:al-non-generalisation-bound}
    \err{\rho}{\hat{h}} - \err{\rho}{f} \leq 2\sup_{\|h\| \leq \gamma}\abs{\err{\rho}{h} - \err{\poolset_0}{h}} + 2 \MMD{\poolset_0}{\trainset}{B_\eta}.
  \end{equation}
\end{theorem}

\begin{proof}
  This is simply an application of the results developed in
  \hyperref{sec:error-decomposition}[Error Decomposition]. We have the following
  \begin{align*}
    \err{\rho}{\hat{h}} - \err{\rho}{f} & = \err{\rho}{\hat{h}} - \err{\trainset}{\hat{h}} + \underbrace{\err{\trainset}{\hat{h}} - \err{\trainset}{h_{\gamma}}}_{\leq 0} + \err{\trainset}{h_{\gamma}} - \err{\rho}{h_{\gamma}} + \err{\rho}{h_{\gamma}} - \err{\rho}{f} \\ 
                                        & \leq \err{\rho}{\hat{h}} - \err{\trainset}{\hat{h}} + \err{\trainset}{h_{\gamma}} - \err{\rho}{h_{\gamma}} + \err{\rho}{h_{\gamma}} - \err{\rho}{f} \\
                                        & \leq \abs{\err{\rho}{\hat{h}} - \err{\trainset}{\hat{h}} + \err{\trainset}{h_{\gamma}} - \err{\rho}{h_{\gamma}} + \err{\rho}{h_{\gamma}} - \err{\rho}{f}} \\
                                        & \leq \abs{\err{\rho}{\hat{h}} - \err{\trainset}{\hat{h}}} + \abs{\err{\trainset}{h_{\gamma}} - \err{\rho}{h_{\gamma}}} + \underbrace{\abs{\err{\rho}{h_{\gamma}} - \err{\rho}{f}}}_{= 0} \\
                                        & \leq 2 \left( \sup_{\|h\| \leq \gamma}\abs{\err{\rho}{h} - \err{\poolset_0}{h}} + \MMD{\poolset_0}{\trainset}{B_\eta} \right),
  \end{align*}
  where in the final line we have used the bounds on the generalisation and empirical drift generalisation error.
\end{proof}
From the first result in the table \ref{tbl:FW-convergence-table} we can bound the
quantity \(\MMD{\poolsetx_{0}}{\trainsetx}{B_{\eta}}\) as follows, let
\(\trainset_{t}\) be the set built from \(\poolset_{0}\) by running FW on
\(\frac{1}{2}\norm{\mu_{\poolsetx_0} - \mu_{\trainsetx_{t}}}_{\rkhs{H}'}^2\) for
\(t\) iterations, then
\begin{align*}
  \label{align:fw-worst-bound-on-mmd}
  \MMD{\poolsetx_{0}}{\trainsetx_{t}}{B_{\eta}} & = \sqrt{2\eta\frac{1}{2}\norm{\mu_{\poolsetx_0} - \mu_{\trainsetx_{t}}}_{\rkhs{H}'}^2} \\
                                                & \leq \sqrt{2\eta \frac{4R^2}{t}}.
\end{align*}
When using the Gaussian kernel, \(R = 1\) and the bound reduces to
\(\MMD{\poolsetx_{0}}{\trainsetx_{t}}{B_{\eta}} = 2\sqrt{\frac{2\eta}{t}}\). For
the Gaussian kernel case we have the following,
\begin{theorem}
  \label{al:worst-fw-generalisation-bound-no-additional-assumptions}
  In addition to \cref{as:al-fw,as:domain-is-compact,as:output-bounded,as:f-in-gamma-ball,as:realisable-setting}, assume that \(\eta = 4\gamma^{2}\), \(K'(x, x') = K(x,
  x')^{2}\) and that \(\trainset_t\) is built using FW, \(\hat{h} = \algo(\trainset_t)\) is Ivanov
  Kernel Regression. Then with probability greater than \(1 - (3 + 2)\delta) = (1 - 5\delta)\) the following bound holds
  \begin{align*}
    \err{\rho}{\hat{h}} - \err{\rho}{f} & \leq \frac{4 \gamma^2}{\sqrt{n}}\left(1 + \sqrt{\frac{\log(1/\delta)}{2}} \right) + \frac{4 M_y \gamma \log(2/\delta)}{n} + 2\sqrt{\frac{2 M_y^2 \log(2/\delta)}{n}} + 2M_y^2\sqrt{\frac{\log(2/\delta)}{2n}} \\
                                        & + 4\sqrt{\frac{2\eta}{t}}
  \end{align*}
\end{theorem}

While this is a bound, it is not an improvement over random (MC) sampling. Consider a
sampling strategy that uniformly at random picks datapoints in \(\poolset_0\),
this is the same as doing uniformly sampling from \(poolset_0\) without
replacement. Since the set \(\trainsetx_{t}\) is now an empirical \(\iid\)
sample from \(\trainsetx_{0}\) we have that \(\MMD{\poolsetx_{0}}{\trainsetx_{t}}{\rkhs{H}'} = O_{P}(\frac{1}{\sqrt{t}})\)
\cite{tolstikhin17_minim_estim_kernel_mean_embed} which is of the same order as
using FW. This is the baseline we would like to improve upon.

\subsection{Lower bounding the radius}
In order to have a generalisation bound which is an improvement over MC sampling
we would have to show that in probablity we can lower bound the quantity
\ref{eq:d-radius-of-biggest-interior-ball} so that we would have a convergence
rate of the second entry of table \ref{tbl:FW-convergence-table}. We proceed as
follows, let \(A = \{\mat{\alpha} \succeq 0, \norm{\mat{\alpha}}_1 = 1,
\norm{\mat{\alpha}}_0 < n\} \subseteq \R^{n}\), the faces of the simplex \(\triangle_{n-1}\),
then the radius \(d\) as a function of \(\poolsetx_{0}\) can be bounded as follows
\begin{align*}
  d(\poolsetx_0)^2 & = \min_{\varphi \in \partial_{relint} \mathcal{M}}\norm{\mu_{\poolsetx_0} - \varphi}^2_{\rkhs{H}} \\
                   & = \min_{A}\norm{\frac{1}{n}\sum_{i=1}^n\phi(x_i) - \sum_{i=1}^n \alpha_{i} \phi(x_{i})}^2_{\rkhs{H}} \\
                   & = \min_{A} \norm{\mat{K}_n^{1/2}(\mathbf{1}/n - \mat{\alpha})}_{2}^2 \\
                   & \geq \lambda_{min}(\mat{K}_{n}) \min_{A} \norm{\mathbf{1}/n - \mat{\alpha}}_{2}^2.
\end{align*}

We first consider the term \(\min_{A} \norm{\mathbf{1}/n -
  \mat{\alpha}}_{2}^2\). Note that this is just the minimum of the distances from
the barycenter \(\mathbf{1}/n\) onto any of the faces of \(\triangle_{n-1}\).
Now, the simplex is symmetric, so the result will be the same up to permutation
of what entries in \(\mat{\alpha}\) that is set to zero, so without loss of
generality we will denote \(\mat{\alpha}^{k} = (\alpha_{1}, \dots, \alpha_{k},
0, \dots, 0)^{T} \in A\). Let \(F_{k}\) be the face generated by \(k\)
non-zero vectors \(\mat{\alpha}^{k}\), then \(F_{k} \subseteq F_{l}\) when \(k
< l\). Thus, we can write
\begin{equation}
  \label{eq:d-optim-problem}
  \min_{F_{k}, k \in [n-1]} \norm{\mathbf{1}/n - \mat{\alpha}}_{2}^2 = \min_{F_{n-1}}\norm{\mathbf{1}/n - \mat{\alpha}}_{2}^2
\end{equation}
where we used the fact that for any \(k < n\), \(F_{k} \subseteq F_{n-1}\) and
so we only have to minimise over \(F_{n-1}\) which is a convex set.

We claim that the optimum over the affine hull of \(F_{n-1}\) of
\ref{eq:d-optim-problem} is attained on \(F_{n-1}\), that is
\(\norm{\mathbf{1}/n - \alpha^{n-1}}_{2}^{2}\) can be optimised over \(A_{n-1}
\coloneqq \affhull(F_{n-1})\). To see this consider the family of vectors
\(\beta^{n-1} = (\beta_{1}, \dots, \beta_{n-1}, 0)^{T}\) where each \(\beta_{i}
\in \R\) and \(\sum_{i=1}^{n-1}\beta_{i} = 1\), then we see that
\begin{equation}
  \label{eq:minimal-d-affhull-opt-problem}
  \min_{\mat{\beta} \in A_{n-1}}\norm{\mathbf{1}/n - \mat{\beta}}^{2}_{2} = \min_{\mat{\beta} \in A_{n-1}} \sum_{i=1}^{n-1} (n^{-1} - \beta_{i})^{2} + n^{-1}.
\end{equation}

We can solve this using lagrange multiplier. The corresponding lagrangian to
\ref{eq:minimal-d-affhull-opt-problem} is
\begin{equation*}
  \lagr(\mat{\beta}, \gamma) = \sum_{i=1}^{n-1} (n^{-1} - \beta_{i})^{2} + \gamma (\sum_{i=1}^{n-1} \beta_i - 1)
\end{equation*}
and using the KKT conditions, we can replace the primal variables
\(\mat{\beta}\). Consider,
\begin{align*}
  \pdv{\lagr}{\beta_{i}} & = 2(\beta_{i} - n^{-1}) + \gamma \\
                         & = 0
\end{align*}
which means that we can set \(\beta_{i} = n^{-1} - \frac{1}{2}\gamma\). We then
have the following
\begin{align*}
  \lagr(\gamma) & = \sum_{i=1}^{n-1} (\frac{1}{2} \gamma)^{2} + \gamma (\sum_{i=1}^{n-1} (n^{-1} - \gamma) - 1) \\
                & = \frac{1}{4}(n-1)\gamma^{2} + \gamma((1 - n^{-1}) - \frac{n-1}{2}\gamma - 1) \\
                & = \frac{1}{4}(n-1)\gamma^{2} - \frac{1}{2}(n-1)\gamma^{2} - n^{-1} \gamma \\
                & = -\frac{1}{4}(n-1)\gamma^{2} - n^{-1} \gamma.
\end{align*}
This expression has the maximum at \(\gamma^{\ast} = -\frac{2}{n(n-1)}\) which
gives that each \(\beta_{i} = n^{-1} + n^{-1}(n-1)^{-1} = \frac{1}{n-1}\). It's
clear that this solution is in \(F_{n-1}\) and hence we have solved the problem.
Substituting this in for the \(\mat{\alpha}\) in
\ref{eq:d-optim-problem}, we have that
\begin{align*}
  \min_{F_{n-1}}\norm{\mathbf{1}/n - \mat{\alpha}}_{2}^2 & = \sum_{i=1}^{n-1}(n^{-1} - (n-1)^{-1})^{2} + n^{-2} \\
                                                         & = (n-1)^{-1}n^{-2} + n^{-2} \\
                                                         & = n^{-2}(1 + (n-1)^{-1}) \\
                                                         & = \frac{1}{n(n-1)} \\
                                                         & > \frac{1}{n^{2}}
\end{align*}
We have thus shown that we can bound
\begin{equation}
  \label{eq:lower-bound-on-d}
  d(\poolsetx_{0}) \geq \frac{\sqrt{\lambda_{min}(\mat{K}_{n})}}{n}.
\end{equation}

Unfortunately this bound is vacuous as for this to be of relevance to us we
require \(\frac{\sqrt{\lambda_{min}(\mat{K}_{n})}}{n} \geq c > 0\) for some
scalar \(c\). This would mean that \(\lambda_{min}(\mat{K}_{n}) =
\Omega(n^{2})\) which does not go well with well-established assumptions on
decay of the eigenvalues of the kernel matrix, for example that
\(\trace(\mat{K}_{n}) = \Theta(n)\) according to \cite{bach13_sharp}.
